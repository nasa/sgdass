\documentstyle[11pt]{article}
%%
\input mace.tex
\newcommand{\pid}   {\par\hangindent=2em\hangafter=-9\noindent}
\newcommand{\ppid}  {\par\hangindent=4em\hangafter=-9\noindent}
\newcommand{\pppid} {\par\hangindent=6em\hangafter=-9\noindent}
\newcommand{\ppppid}{\par\hangindent=8em\hangafter=-9\noindent}
%\newcommand{\nthr}[1]{ \multicolumn{3}{ @{} l }{#1} }
\renewcommand{\epsilon}{\varepsilon}
\renewcommand{\phi}{\varphi}
\newcommand{\dint}{\int\hspace{-0.25em}\int}
\newcommand{\hh}{\hphantom}
\newcommand{\hpp}{\hphantom+}
\newcommand{\vex}{\vspace{1ex}}
\newcommand{\den}{\Frac{1}{c}\Frac{1}{1 + \Frac{1}{c}\lp \vec{V_{\earth}}(t_1)\, + \dmat{E}(e(t_1))\,\vec{r}_2\rp} }

\voffset    = -25mm
\hoffset    = -15mm
\textwidth  = 160mm
\textheight = 230mm
\tolerance  = 600
\hfuzz      = 0.3mm

\title {  \LARGE\bf Memo: Computation of a Doppler frequency shift for
                          observations in the Solar system}
\author{ {\large\sc L. Petrov} \\
         {\small Leonid.Petrov@lpetrov.net }
       }
\date{Draft of 2007.01.11}

\begin{document}


\maketitle

\section{Coarse expression}

  It is deduced in the elementary coarse of physics that the ratio of the
observed frequency $f_o$ of the electromagnetic wave to the frequency 
$f_e$ emitted by a body moving with the speed $\vec{v}_e$ with respect to the 
observer being at rest is
%
\beq
     \Frac{f_o}{f_e} = 
           \Frac{1 - \frac{1}{c} \, \vec{s}(t_e) \cdot \vec{v}_e(t_e)}
                { \sqrt{ 1 - \frac{v_e^2}{c^2}} }
\eeq{e:e1}

  Here $\vec{s}(t_e)$ is the unit vector of the direction to the emitter at the
moment of coordinate time $t_e$ when the light was emitted.

  If the velocity of the emitter is given in the coordinate system at which
the observer is not at rest, then the expression for the frequency ratio 
is changed to 
%
\beq
     \Frac{f_o}{f_e} = 
           \Frac{1 - \frac{1}{c} \, \vec{s}(t_e) \cdot \vec{v}_e(t_e)}
                {1 - \frac{1}{c} \, \vec{s}(t_o) \cdot \vec{v}_o(t_o)}
           \, 
           \Frac{\sqrt{ 1 - \frac{v_o^2}{c^2}}}
                {\sqrt{ 1 - \frac{v_e^2}{c^2}}}
\eeq{e:e2}
%
  where $t_o$ is the coordinate time of singal receving

  Gravitating bodies also cause the frequency changes. Taking this effect into
account and {\it assuming that the gravitating bodies are not moving}, modifies 
the previous equation this way:
%
\beq
     \Frac{f_o}{f_e} = 
           \Frac{1 - \frac{1}{c} \vec{s}(t_e) \cdot \vec{v}_e(t_e)}
                {1 - \frac{1}{c} \vec{s}(t_o) \cdot \vec{v}_o(t_o)}
           \:\: 
           \Frac{\sqrt{ 1 - \frac{v_o^2}{c^2}}}
                {\sqrt{ 1 - \frac{v_e^2}{c^2}}}
           \:\: 
           \Frac{1 - \dss\sum_i^n \Frac{G m_i}
                                  {c^2 \, |\vec{r}_i(t_i) - \vec{r}_e(t_e)|}}
                   {1 - \dss\sum_i^n \Frac{G m_i}
                                  {c^2 \, |\vec{r}_i(t_i) - \vec{r}_o(t_o)|}}
\eeq{e:e3}
%
   where summation is done over all gravitating bodies. Assumption that 
the gravitating bodies are not moving results in $t_i = t_o$. At the first 
approximation taking into account motion of gravitating bodies 
can be done in the form of setting $t_i$ to the retarded moment of time
$t'_i$, which is a solution of the gravitation null-cone equation:

\beq
      t_i = t_o - \Frac{1}{c} 
                   \biggl| \vec{r}_i(t_i) - \vec{r}_o(r_o) \biggr|
\eeq{e:e4}

   I do not have by hand estimates of errors of such an approximation.

\section{Refined expression}

  \cite{ks1999} derived a rigorous expression for the case when gravitating
bodies have an arbitrary motion. 

 [to be added in the future.]

\section{Computation of the Doppler frequency shift using the coarse expression}

  We can re-write expression \ref{e:e3} in this way:

\beq
  \Frac{f_o}{f_e} = \Frac{1-a_e}{1-a_o} \: 
                    \Frac{\sqrt{1-b^2_o}}{\sqrt{1-b^2_e}} \:
                    \Frac{1-c^2_e}{1-c^2_o} 
\eeq{e:e5}
%
  where 
\beq
   \begin{array}{lcl}
      a_k   & = & \frac{1}{c} \vec{s}(t_k) \cdot \vec{v}_k(t_k) \vex \\
      b_k   & = & \frac{|v_k|}{c}                                 \vex \\
      c^2_k & = & \dss\sum_i^n \Frac{G m_i}
                  {c^2 \, |\vec{r}_i(t_i) - \vec{r}_k(t_k)|}
   \end{array}
\eeqn

   Expand expression \ref{e:e5} discarding terms of $O(c^{-4})$:
%
\beq
  \begin{array}{l} 
  D = \Frac{f_o}{f_e} - 1 \enskip = \enskip
      \hphantom{+} a_o - a_e 
%
      \\ \hphantom{D = \Frac{f_o}{f_e} - 1 \enskip = \enskip }
      + a^2_o - a_e a_o - \frac{1}{2} b^2_o + \frac{1}{2} b^2_e -c^2_e + c^2_o 
%
      \vex \\ \hphantom{D = \Frac{f_o}{f_e} - 1 \enskip = \enskip }
      + a^3_o - a_e a^2_o - \frac{1}{2} a_o b^2_o + \frac{1}{2} a_o b^2_e 
      + \frac{1}{2} a_e b^2_o - \frac{1}{2} a_e b^2_e 
      - a_o c^2_e + a_e c^2_e + a_o c^2_o - a_e c^2_o
  \end{array}
\eeq{e:e6}

  According to \cite{bru91}, the velocity of the observer with repsect to 
the geoctenter, $\vec{v}_g$, is related to its velocity with respect to 
the barycenter, $\vec{v}_b$, as

\beq
    \vec{v}_b = \frac{ \sqrt{1 - \frac{V\earth^2}{c^2} } }
                     { 1 + \frac{\vec{v}_g \cdot \vec{V}\earth}{c^2} }
        \biggl\{ \vec{v}_g \: + \: \biggl[ \biggl( 
                \frac{1}{\sqrt{1 - \frac{V\earth^2}{c^2}}} - 1 \biggr)
                \, \Frac{\vec{v}_g \cdot \vec{V}\earth}{V\earth^2}
                \: + \: \frac{1}{\sqrt{1 -\frac{V\earth^2}{c^2}}} 
               \biggr] \vec{V}\earth
        \biggr\}
\eeq{e:e7}

  Expanding the expression \ref{e:e7} in series over 1/c retaining
terms of $1/c^2$ and adding contribution to the metric due to gravitational
potential, we get the following transformation:
%
\beq
    \vec{v}_o = \lp 1 - \lp \Frac{U\sun}{c^2} - L_b \rp - 
                        \frac{1}{2} \Frac{V\earth^2}{c^2} \,
          - \Frac{\vec{v}_g \cdot \vec{V}\earth}{c^2} \rp \vec{v}_g \: + \:
          \lp 1 - \Frac{1}{2 c^2} \vec{v}_g \cdot \vec{V}\earth \rp
          \vec{V}\earth \: + \:
          O(c^{-4})
\eeq{e:e8}
%
   where $L_b$ is the metric parameter, $L_b = \flo{1.55051976772}{-8}$, if the
TDB metric is used.

\begin{thebibliography}{1}

\bibitem[Brunmberg, 1999]{bru91}
   Brumberg, V.A., Essential relativistic mechanics, 274p, 1991.

\bibitem[Kopeikin and  Shaeffer, 1999]{ks1999}
   Kopeikin, S.M. and Shaeffer~G., Physical Review, D, vol. 50, 124002, 1999.

\end{thebibliography}

\end{document}
