% ************************************************************************
% *                                                                      *
% *   _ _ _ _ _ _ _ _ _ _ _ _ _ _ _ _ _ _ _ _ _ _ _ _ _ _ _ _ _ _ _ _    *
% *  |            A D D I T I O N A L     C O M M A N D S.           |   *
% *   ~ ~ ~ ~ ~ ~ ~ ~ ~ ~ ~ ~ ~ ~ ~ ~ ~ ~ ~ ~ ~ ~ ~ ~ ~ ~ ~ ~ ~ ~ ~ ~    *
% *                                                 (English version)    *
% *                                                                      *
% *  ###  30-NOV-93    MAC.TEX   V2.13  (c) Petrov L.Y.  01-NOV-96  ###  *
% *                                                                      *
% ************************************************************************
\newcommand{\getlength}[1]{\ifx#1\end \let\next=\relax
            \else\advance\count255 by1 \let\next=\getlength\fi \next}
%
% ----- \length  --  LEARN THE LENGTH OF THE ARGUMENT
%
\newcommand{\length}[1]{ \count255=0 \getlength#1\end }
%
% ----- \ifnularg}{} -- VERIFY: IS THE ARGIMENT EMPTY
%
\newcommand{\ifnularg}[1]{ \count255=0 \getlength#1\end \ifnum\count255=0 }
%
%
% ----- \ifm  --  VERIFY: IS IT MATH-MODE NOW?
%
\newcommand{\ifm}{\makebox{}\ifmmode}
\newcount\switch
%
% ----- \Begmat, \Endmat  --  BEGIN OF MATHEMATIC -- END OF MATHEMATIC.
%       IF MATH-MODE IS NOW THEN NOTHING TO DO. iF TEXT-MODE IS NOW THEN
%       SIGN $ WILL BE ADDED.
%
\newcommand{\Begmat}{\ifm\switch=1\else\switch=0$\fi}
\newcommand{\Endmat}{\ifnum\switch=0$\fi}
%
% ----- \flo  --  MAKING THE NUMBER IN EXPONENTAL FORM
%
\newcommand{\Flo}[2]{{\Begmat
                         \ifnularg{#1} 10^{#2}  \else #1\!\cdot\!10^{#2} \fi
                      \Endmat}}
\newcommand{\flo} [2] { {\Begmat
            \ifnularg{#1} 10^{#2}  \else \mbox{#1} \cdot 10^{#2} \fi
                      \Endmat} }
%
\newcommand{\mb}{\makebox}
\newcommand{\mint}{\makebox{--}}
\newcommand{\drt}[1]{\Begmat \dot{#1} \Endmat}
%
% ----- \beq  --  BEGINNING OF FORMULA ARRAY
%
\newcommand{\beq}   { \begin{eqnarray} }
%
% ----- \eeq  --  END OF FORMULA ARRAY. THERE IS OBLIGATORY ARGUMENT: LABEL
%
\newcommand{\eeq}[1]{ \ifnularg{#1} end{eanarray} \else
                      \label{#1}\end{eqnarray}    \fi }
\newcommand{\eeql}   { \end{eqnarray} }
\newcommand{\eeqn}   { \nonumber \end{eqnarray} }
\newcommand{\nn} {\nonumber\\ }
\newcommand{\eref}[1]{(\ref{#1})}
%%
\newcommand{\dss}{\displaystyle}
\newcommand{\tss}{\textstyle}
\newcommand{\tsc}{\scriptstyle}
\newcommand{\ssc}{\scriptscriptstyle}
\newcommand{\st}{\strut}
\newcommand{\Frac}[2]{\frac{\displaystyle\strut #1}{\displaystyle\strut #2} }
%%
\newcommand{\mst}{\mathstrut}
\newcommand{\mat}[1]{\Begmat {\hat {\mathstrut \cal #1}} \Endmat }
\newcommand{\matt}[1]{\Begmat {\widehat {\mathstrut \widetilde{\cal #1}}}
                      \Endmat }
\newcommand{\der}[2] {\Begmat \frac{ \partial #1 }{ \partial #2 } \Endmat }
\newcommand{\Der }[2]{\Begmat \Frac{ \partial #1 }{ \partial #2 } \Endmat }
\newcommand{\ddt}{\der{}{t} }
\newcommand{\Ddt}{\Der{}{t} }
\newcommand{\dmat}[1]{\ddt \mat{#1} }
\newcommand{\vc }[1]{{\Begmat \bf \vec{ #1 } \Endmat }}
\newcommand{\vce }[1]{{\Begmat \bf \vec{ \underline{#1} } \Endmat }}
\newcommand{\Vc}[1]{\Begmat\overrightarrow{\mathstrut{\bf#1}}\Endmat}
\newcommand{\vm }[1]{ {\bf #1}}
\newcommand{\dvc }[1] {\Begmat {\bf \dot  {\vec { #1}}} \Endmat }
\newcommand{\ddvc }[1]{\Begmat {\bf \ddot {\vec { #1}}} \Endmat }
\renewcommand{\Re} { {\rm Re}\hspace{0.2em} }
\renewcommand{\Im} { {\rm Im}\hspace{0.2em} }
%%
\newcommand{\Half}{\Frac{1}{2} }
\newcommand{\half}{\frac{1}{2} }
\newcommand{\ArroW}
    {\Begmat \stackrel{\textstyle\longrightarrow}{\longleftarrow} \Endmat }
\newcommand{\upp}[1]{ {}^#1{\scriptstyle\kern-0.3em . \kern0.15em } }
\newcommand{\ups}{''\!\!.\,}
\renewcommand{\sec}{ \upp{s} }
\newcommand{\Day}{ \upp{d} }
\newcommand{\grad}[2]{\Begmat{ #1^\circ#2' }\Endmat }          % grad, min, sec
\newcommand{\ang}[4]{\Begmat{ #1^\circ#2'#3\ups#4   }\Endmat }
\newcommand{\Ang}[3]{\Begmat{ #1^\circ#2'#3''       }\Endmat }
\newcommand{\tim}[4]{\Begmat{ #1^h#2^m#3\sec#4      }\Endmat }
\newcommand{\Tim}[3]{\Begmat{ #1^h#2^m#3^s          }\Endmat }
\newcommand{\lp}{ \left(  }
\newcommand{\rp}{ \right) }
\renewcommand{\deg}[1]{\Begmat #1^\circ \Endmat }
\newcommand{\iint}{ \int\limits_{\:-\infty}^{\:+\infty}\!\!\!\!\!\int }
%%
\newcommand{\spd}{\,}
\newcommand{\vpd}{\times}
\newcommand{\sun}  {\Begmat _\odot \Endmat }
\newcommand{\earth}{\Begmat _{\oplus} \Endmat }
\newcommand{\VEA}{\vc{V}\earth}
\newcommand{\AEA}{\vc{A}\earth}
\newcommand{\VEAS}{ { \bf \Begmat V^2\earth \Endmat } }
%%
\newcommand{\CCF}[1]{\Begmat \vc{#1}_{\mbox{\sc\tiny CCF  } } \Endmat }
\newcommand{\CEP}[1]{\Begmat \vc{#1}_{\mbox{\sc\tiny CEP  } } \Endmat }
\newcommand{\MED}[1]{\Begmat \vc{#1}_{\mbox{\sc\tiny MED  } } \Endmat }
\newcommand{\CFS}[1]{\Begmat \vc{#1}_{\mbox{\sc\tiny CFS  } } \Endmat }
\newcommand{\VEN}[1]{\Begmat \vc{#1}_{\mbox{\sc\tiny VEN  } } \Endmat }
%%
\newcommand{\tg}{ \mathop{ \rm tg }\nolimits }
\newcommand{\arctg}{ \mathop{ \rm arctg }\nolimits }
\newcommand{\Cov}{ \mathop{ \rm Cov }\nolimits }
\newcommand{\Sp}{ \mathop{ \rm Sp }\nolimits }
\newcommand{\rank}{ \mathop{ \rm rank }\nolimits }
\newcommand{\const}{ \mathop{ \rm const }\nolimits }
%%
\newcommand{\un}[1]{\underline{#1}}
\newcommand{\tr}{ {\bf ^{^{ \top}}}\! }
\newcommand{\tra}[1]{ {#1}^{^{\bf \top}}\! }
\newcommand{\ttra}[1]{ {#1}^{\bf \top}\! }
\newcommand{\as}[1]{ #1^{\bf \ast} }
\newcommand{\Exp}[1]{ {\cal{E}}(#1) }
\newcommand{\tabstrut}{ \rule[-1ex]{0ex}{3.5ex} }
%%
%
% ----- CONVENIENT MACROSES FOR ENVIRONMENT supertabular
%
\newcommand{\superheader}{}
\newcommand{\supertabularbegin}[1]
{
      \tablefirsthead{\hline \superheader \hline}
      \tablehead{\multicolumn{#1} {r} {\small\it Continuation \dots}\\
                 \hline \superheader \hline }
      \tabletail{\hline \multicolumn{#1}{r}
                {\small \it Continuation on the next page \dots} \\ }
      \tablelasttail{\hline}
}
\newcommand{\tableheader}[1]
{ \renewcommand{\superheader}{#1 \\} }
%%%
\newcommand{\ntab}[2]{ \multicolumn{1}{#1}{#2} }
\newcommand{\prog }[1]{{\mbox{\sc\small\bf#1}}}
\newcommand{\arr}{\ifm   \longleftrightarrow    \else
                       $ \longleftrightarrow $  \fi   }
\newcommand{\twosl}{\Begmat \lefteqn{/}\,\, / \,\, \Endmat}
