%\documentclass[11pt]{article}
\documentstyle[big,11pt,epsf]{article}
%
%  First draft:  2006.08.15_13:55:25
%  Last updated: 2014.08.21_13:17:31
%
\input mace.tex
\newcommand{\pid}   {\par\hangindent=2em\hangafter=-9\noindent}
\newcommand{\ppid}  {\par\hangindent=4em\hangafter=-9\noindent}
\newcommand{\pppid} {\par\hangindent=6em\hangafter=-9\noindent}
\newcommand{\ppppid}{\par\hangindent=8em\hangafter=-9\noindent}
%\newcommand{\nthr}[1]{ \multicolumn{3}{ @{} l }{#1} }
\renewcommand{\epsilon}{\varepsilon}
\renewcommand{\phi}{\varphi}
\newcommand{\dint}{\int\hspace{-0.25em}\int}
\newcommand{\hh}{\hphantom}
\newcommand{\hpp}{\hphantom+}
\newcommand{\hbb}{\hspace{-0.5em}}
\newcommand{\hpm}{\hphantom-}
\newcommand{\vhhhex}{ \vspace{0.125ex} }
\newcommand{\vhhex}{ \vspace{0.25ex} }
\newcommand{\vhex}{ \vspace{0.5ex} }
\newcommand{\vnhex}{ \vspace{-0.5ex} }
\newcommand{\vnex}{ \vspace{-1.0ex} }
\newcommand{\vex}{\vspace{1ex}}
\newcommand{\veX} { \vspace{1.0ex} \vphantom{\Frac{m}{s}} }
\newcommand{\den}{\Frac{1}{c}\Frac{1}{1 + \Frac{1}{c}\lp \vec{V_{\earth}}(t_1)\, + \dmat{E}(e(t_1))\,\vec{r}_2\rp} }
\newcommand{\matd}[1]{\Begmat {\widehat {\mathstrut \dot{\cal #1} }} \Endmat }
\newcommand{\matdd}[1]{\Begmat {\widehat {\mathstrut \ddot{\cal #1} }} \Endmat }
\newcommand{\Sum}{\displaystyle\sum}
% \newcommand{\ntab}[2]{ \multicolumn{1}{#1}{#2} }
\newcommand{\nntab}[2]{ \multicolumn{2}{#1}{#2} }
\newcommand{\nnntab}[2]{ \multicolumn{3}{#1}{#2} }
\newcommand{\nnnntab}[2]{ \multicolumn{4}{#1}{#2} }
\newcommand{\Deg}[1]{#1^\circ}
\newcommand{\mss}[1]{\mbox{\scriptsize #1}}
\renewcommand{\epsilon}{\varepsilon}
\renewcommand{\phi}{\varphi}
\newcommand{\twofbox}[2]{ \fbox{\rule{0mm}{4ex}%
\Begmat\stackrel{\mbox{#1\hfill}}{\mbox{#2\hfill}}\Endmat } }
\newcommand{\dintinf}{\int\limits_{\enskip-\infty}^{\enskip+\infty}\hspace{-0.75em}\int}

\voffset    = -25mm
\hoffset    = -15mm
\textwidth  = 160mm
\textheight = 230mm
\tolerance  = 600
\hfuzz      = 0.3mm

\title {  \LARGE\bf Memo: Library VTD for computation of VLBI time delay.
                          Description of the algorithm. }
\author{ {\large\sc L. Petrov} \\
         {\small Leonid.Petrov@lpetrov.net }
       }
\date{Revision of 2014.08.21}

\begin{document}


\maketitle

\hspace{0.05\textwidth}
\begin{minipage}{0.82\textwidth}
\small\it
   An algorithm for computing VLBI time delay partial derivatives with
respect to parameters of the model, the algorithm of parameter estimation
is described.
\end{minipage}\par\bigskip\bigskip\par

\section{Introduction}

  The method of VLBI first proposed by \cite{r:mat65}  allows to measure
precisely the time delay and its rate of change. The time delay is defined
as {\it a difference between two intervals: a) the interval of proper time
measured by clocks of the first station between events: arrival the wavefront
to the reference point of the first antenna and clock synchronization and
b) the interval of proper time measured by clocks of the second station
between events: arrival the wavefront to the reference point of the second
antenna and clock synchronization}. The post-correlator software evaluates
the time delay and its time derivative to a certain moment of time at time
scale TAI called {\it fringe reference time} within the interval of
observation, which is typically 20--700 second long.

  In this paper, the algorithm for computing the time delay and delay rate
is presented in chapter \ref{s:algorithm}. The order of narration in this
chapter follows the order of derivation of this quantity: first the
expression for the time delay will be derived, then the algorithm for computing
of position of the emitter and receivers in the inertial coordinate system,
is described. The motion of receivers is decomposed into rotation and
deformation. Then effects of propagation in refractive medium are taken into
account. Finally, the coupling effects are taken into account.

  The chapter \ref{s:implementation} follows the order of computing the
intermediate quantities which will be finally substituted in the expression
for time delay and delay rate. Description of formats of input files needed
for implementation of calculation is presented in document 
{\tt vtd\_format.txt} .

  The truncation level for computation is $10^{-13}$  s for delay and $10^{-16}$
s/s for delay rate. If the apriori were perfect the accuracy for time delay
computation would be at that level. In practice, the theoretical path
delay can be predicted with the error of \flo{1--10}{-9} s.

\section{Algorithm for computation of the theoretical path delay, delay rate
and partial derivatives with respect to parameters of model}
\label{s:algorithm}

\subsection{Expression for geometric path delay}

  Let us have two stations \#1 and \#2. Reference station \#1 receives 
a radio wave form source $a$ in time $t_1$ according to its local clock 
which are assumed synchronized with TAI via GPS. Remote station \#2 receives 
a radio wave from the emitter $e$ in time $t_2$ according to its clock 
which are also assumed synchronized with TAI via GPS. The problem is, 
knowing position and velocity of stations \#1, \#2, and object $a$,
what will be the time $t_2$?

  We have three events: emission of the radio wave with coordinates
$ \vc{R}_e, T_e $ in a barycentric coordinate system and two events of
receiving that radio wave at stations \#1 and \#2 with coordinates
$\vc{r}_1, t_1$, $\vc{r}_2, t_2$ in a geocentric coordinate system.

  The post Newtonian metric in these coordinates can be written this way:

\beq
\begin{array}{ll}
   G_{00} = \hphantom{-\delta_{mn}A} 1 - 2 \Frac{U}{c^2} + 2\L_b &
            \; + \; O(1/c^4) \\
   G_{0k} =                                                      &
            \hpp \; \; O(1/c^3) \\
   G_{mn} = -\delta_{mn} \lp 1 + 2 \Frac{U}{c^2} - 2\L_b\rp      &
            \; + \; O(1/c^4) \\
\end{array}
\eeq{e:e1}

\beq
\begin{array}{ll}
   g_{00} = \hphantom{-\delta_{mn}A} 1 - 2 \Frac{U_\oplus}{c^2} + 2\L_g &
            \; + \; O(1/c^4) \\
   g_{0k} =                                                      &
            \hpp \; \; O(1/c^3) \\
   g_{mn} = -\delta_{mn} \lp 1 + 2 \Frac{U_\oplus}{c^2} - 2\L_g\rp      &
            \; + \; O(1/c^4) \\
\end{array}
\eeq{e:e2}
%
  for the barycentric and geocentric coordinate systems respectively. Here
$U$ is the sum of the gravitational potential of all external bodies
in the geocenter, $U_\oplus$ --- geopotential, $L_b$ and $L_g$ are some
arbitrary scaling constants. Their numerical values depends on a convention.
Parameters $L_b$ and $L_g$ were introduced into the expression for the metric
artificially. Equations of general relativity allows scaling transformation.
When the coordinate system is defined, 7 parameters should be specified:
three parameters of the origin, three parameters of orientation, and
{\it the scaling parameter}. In physics usually $L_b \: =\: L_g = \: 0$.
That implies that in the infinity the metric tensor becomes the Minkowsky
tensor. In geodesy various values for
$L_b$ and $L_g$ were used:
%
  \begin{itemize}
       \item TDB baricenteric coordinate system,
             $L_b = \Frac{f M\sun}{\bar{R}\sun \, c^2} =
              \flo{1.48082686741}{-8}$, $L_g = 0$;

       \item ITRF2000 (or IERS1992) geocentric coordinate system,
             $L_b = 0$; $L_g = \flo{6.969290134}{-10} \approx 
             \Frac{f M\earth}{\bar{R}\earth c^2} + 
             \Frac{2}{3c^2} \Omega\earth^2 R\earth$;

       \item IAU2000 baricenteric coordinate system, $L_b \: =\: L_g = \: 0$.

       \item IERS1996 geocentric coordinate system,
             $L_b = 0$; $L_g = -\Frac{f M\earth}{\bar{R}\earth \, c^2} =
                    -\flo{6.969290134}{-10} \approx
             -\Frac{f M\earth}{\bar{R}\earth c^2} - 
              \Frac{2}{3c^2} \Omega\earth^2 R\earth$;
\end{itemize}
%
  where $\Omega\earth$ is the nominal Earth's angular velocity.

  Since coordinates of the emitter are represented in the barycentric
coordinate system, let us first compute time delay as the difference of
barycentric time coordinates and then transform it to the difference of
intervals of proper time which are related to quantities derived from
analysis of VLBI fringe phases.

  First, transform coordinates $\vc{r}_1, t_1$, $\vc{r}_2, t_2$ from the
geocentric to the barycentric coordinate system. Using expression for
metric \eref{e:e1}, \eref{e:e2} we find

\beq
 \vc{R}_1(T_1) = \biggl(1 - \Bigl(\Frac{U\sun}{c^2} - L_b\Bigr) \biggr) \, \vc{r}_1 -
           \Frac{1}{2 c^2} \: ( \vc{V}\earth \, \vc{R}\earth ) \, \vc{V}\earth
    \nonumber \\
%
 \vc{R}_2(T_1) = \biggl(1 - \Bigl(\Frac{U\sun}{c^2} - L_b\Bigr) \biggr) \, \vc{r}_2 -
           \Frac{1}{2 c^2} \: ( \vc{V}\earth \, \vc{R}\earth ) \, \vc{V}\earth
    \nonumber \\
%
    T_1 = t_1 + \Frac{1}{c^2} \: \vc{r}_1 \, \vc{V}_\oplus
              + \Frac{1}{c^2} \dss\int\limits_{t_0}^t
                \biggl( \Frac{1}{2} v^2 + U \biggr) \; dt  + 32.184
\eeq{e:e3}
%
  Here we retain only contribution of the gravitational potential of the Sun.

The barycentric travel time of the signal from the emitter $a$ to station \#1
and to station \#2 are
%
\beq
%
\begin{array}{ll}
T_1 - T_e = & \Frac{1}{c} \biggl| \vc{R}_e(T_e) - \vc{R}_1(T_1) \biggr| +
              T_{1,grav}  \\

T_2 - T_e = & \Frac{1}{c} \biggl| \vc{R}_e(T_e) - \vc{R}_2(T_2) \biggr| +
              T_{2,grav}  \\
%% %
%% T_{grav_1}  = &  \dss \sum_k \Frac{2  f  M_k }{c^3} \log \Biggl|
%%     \Frac{ | \vc{R}_k(T'_k) - \vc{R}_e(T_e)| \; + \;
%%            | \vc{R}_k(T'_k) - \vc{R}_1(T_1)| \; + \;
%%            | \vc{R}_e(T_e)  - \vc{R}_1(T_1)| }
%% %
%%          { | \vc{R}_k(T'_k) - \vc{R}_e(T_e)| \; + \;
%%            | \vc{R}_k(T'_k) - \vc{R}_1(T_1)| \; - \;
%%            | \vc{R}_e(T_e)  - \vc{R}_1(T_1)| \:    } \Biggr| \vspace{1ex} \\
%% %
%% T_{grav_2}  = &  \dss \sum_k \Frac{2  f  M_k }{c^3} \log \Biggl|
%%     \Frac{ | \vc{R}_k(T'_k) - \vc{R}_e(T_e)| \; + \;
%%            | \vc{R}_k(T'_k) - \vc{R}_2(T_2)| \; + \;
%%            | \vc{R}_e(T_e)  - \vc{R}_2(T_2)| }
%% %
%%          { | \vc{R}_k(T'_k) - \vc{R}_e(T_e)| \; + \;
%%            | \vc{R}_k(T'_k) - \vc{R}_2(T_2)| \; - \;
%%            | \vc{R}_e(T_e)  - \vc{R}_2(T_2)| \:    } \Biggr|
%%                                                                 \vspace{1ex} \\
T_{i,grav}  = &  \dss \sum_k \Frac{2  f  M_k }{c^3}                  \\ &
                 \biggl(
                    1 + \Frac{1}{c} \dvc{R}_k(T'_e) \vce{S}(T_i,T'_e)
                 \biggr)
                      \ln \lp |\vc{R}_e(T_e) - \vc{R}_k(T'_e)| +
                              (\vc{R}_e(T_e) - \vc{R}_k(T'_e))
                               \vce{S}(T_i,T'_e)
                          \rp -                                      \\ &
%
                 \biggl(
                    1 + \Frac{1}{c} \dvc{R}_k(T'_k) \vce{S}(T_i,T'_e)
                 \biggr)
                     \ln \lp |\vc{R}_i(T_i) - \vc{R}_k(T'_e)| +
                             (\vc{R}_i(T_i) - \vc{R}_k(T'_e))
                              \vce{S}(T_i,T'_e)
                         \rp
\end{array} \hspace{-2em}               
\eeq{e:e4}


  where $f$ is the universal gravitational constant, $M_k$ is the mass
of the gravitating body, summation is done over all big planets of the
Solar system, excluding the Pluto, but including the Moon. Position of
the gravitating body in \eref{e:e4} is taken in the retarded moment
of barycentric time $T'_k$ which is a solution of the gravitation
null-cone equation:
%
\beq
           T'_k = T_1 - \Frac{1}{c}
                  \biggl| \vc{R}_1(T_1) - \vc{R}_k(T'_k) \biggr|
\eeq{e:e6}

  Retarded moment of emission $T'_e$ is a solution of a similar
gravitation null-cone equation:
%
\beq
           T'_e = T_1 - \Frac{1}{c}
                  \biggl| \vc{R}_1(T_1) - \vc{R}_k(T'_e) \biggr|
\eeq{e:e7}

The vector towards the source $ \vce{S}(T_i,T_e) $ is

\beq
   \vce{S}(T_i,T'_e)  = \Frac{\vc{R}_i(T_i) - \vc{R}_e(T'_e)}
                            {|\vc{R}_i(T_i) - \vc{R}_e(T'_e)|}
\eeq{e:e8}

  Derivation of expression \eref{e:e4} was made by many authors. At present,
one of the most comprehensive papers is the article of \cite{r:ks1999}.
For solving equations \eref{e:e4} two cases should be considered
separately, when the object is in the far zone, i.e. we can consider the
wavefront flat, and the case one cannot neglect wavelength curvature.
This happens when the diurnal parallax of the object exceeds the error of
delay computation, $ \sigma\tau/|\tau|, 10^{-11}$ in our case.  This is true
for the objects within the Solar System.

\subsubsection{VLBI time delay for a far zone object}

  In the case, when $ \Frac{|R_i|}{|R_e|} < \sigma_\tau/|\tau| $ expression
for gravitational excess delay \eref{e:e4}
$ \tau_{grav} = T_{1,grav} - T_{2,grav} $ is reduced to
%
\beq
  \tau_{grav} =
   2 \dss\sum_{k=1} \Frac{f \, M_k}{c^3} \lp
                    1 + \Frac{1}{c} \dvc{R}_k(T'_{1k}) \cdot \vce{S} \rp \;
     \ln  \Frac{|\vc{R}_k(T'_k)   - \vc{R}_1| + \vce{S} \cdot
                (\vc{R}_k(T'_{k}) - \vc{R}_1 ) }
                {|\vc{R}_k(T'_k)   - \vc{R}_2| + \vce{S} \cdot
                (\vc{R}_k(T'_{k}) - \vc{R}_2 ) }
\eeq{e:e9}

  Implicit equation for time delay \eref{e:e4} can be simplified using by
expanding it in series. For any vector $\vc{A}$ and $\vc{b}$ such that
$ |\vc{b}| \ll |\vc{A}| $ the module of their difference
$ |\vc{A} - \vc{b}| $ can be expressed as
%
\beq
     | \vc{A} - \vc{b} | = |A| + \vc{a} \cdot \vc{b} +
                     O\biggl( \frac{b^2}{|A|} \biggr)
\eeq{e:e10}
%
where $ \vc{a} = \Frac{A}{|A|}$. Then the difference of barycentric time
coordinates can be written as
%
\beq
   T_2 - T_1 = \Frac{1}{c} \lp \vc{R}_1(T_1) - \vc{R}_2(T_2) \rp \vce{S} +
               \tau_{grav} + \tau_{p}
\eeq{e:e11}
%
   where $\tau_{p}$ is an additional delay which takes into account extra delay
caused by the propagation media.

  After expansion $\vc{R}_2(T_2)$ near instant $T_2$ as
%
\beq
 \vc{R}_2(T_2) = \vc{R}_2(T_1) + \dvc{R}_2(T_1)(T_2 - T_1) +
                   O( \ddvc{R}_2(T_2-T_1)^2 )
\eeq{e:e12}
%
we have
%
\beq
   T_2 - T_1 = \Frac{1}{c}
               \Frac{ \biggl( R_1(T_1) - R_2(T_2) \biggr) \cdot \vce{S} +
                      \tau_{grav} + \tau_p }
                    {1 + \Frac{1}{c} \dvc{R}_2(T_1) \cdot \vce{S}}
\eeq{e:e13}
%
  Now we should relate baricenteric vectors of site positions with geocentric
vector coordinate and transform the difference of baricenteric time
coordinates $T_2 - T_1$ to the difference of intervals of proper time
between events of coming the wavefront to stations \#1 and \#2 and 
clock syncrhonization.  The differences of barycentric time coordinates
$ T_1  - T_{sync} $ and $ T_2 - T_{sync} $ are first to transformed
to geocentric coordinate system, then these differences of geocentric
time coordinates are transformed to intervals of proper time.

  Term $ \Frac{1}{c^2} \, \dss \int \biggl( \Frac{1}{2} v^2 + U \biggr) \, dt $
in \ref{e:e3} deserves special consideration. This sum of centrifugal 
and gravitational potential describes the difference in clock rate 
at a station with respect to the geocentric time. First, notice, time 
as measured by station clock synchronized in TAI or 
TT\footnote{TT - TAI = 32.184}. TAI cooresponds to proper time of 
a hypotetical clock on the geoid. For a station that has the height 
above the geoid, so-called orthometric height, this term is reduced
$ -\Frac{1}{c^2} \, \dss \int \biggl( \frac{2}{3}\Omega_\oplus^2 R_\oplus \cos^2 \phi 
\, + \, g_{loc} \biggr) h_{ort} \, dt $, where $g_{loc}$ is the local gravity 
acceleration. Neglecting site position variations due to tides and mass
loading, the expression under integral is constant. Integration will give
a linear trend and an arbitrary integration constant. The value of the 
constant corresponds to clock syncrhonization that occurs before the 
experiment start.

Finally, we get the following expression for time delay in the far zone:
%
\beq
   \begin{array}{l}
      \biggl((t_{2p} - t_{sync}) - (t_{1p} - t_{sync})\biggr)(t_1) = \\ \hspace{1em}
%
      \Frac{1}{ 1 + \Frac{1}{c} (\vc{V}_\oplus + \dvc{r}_2 ) \cdot \vce{S} }
              \;
%
      \Biggl(
            \Frac{1}{c} \bigl(\vc{r}_1(t_1) - \vc{r}_2(t_1) \bigr) 
                        \cdot \vce{S} \;
            \biggl[1 -
               \biggl(2 \Frac{f M_\odot} { |\vc{R}_\oplus| c^2 } - L_b \biggr) -
               \biggl( \Frac{U_\oplus}{c^2} - L_g \biggr) -
               \Frac{|\vc{V}_\oplus|^2}{2c^2} -
\\ \hspace{1em}
%
               \Frac{\vc{V}_\oplus \cdot \dvc{r}_2}{c^2}
            \biggr] \: + \:
            \Frac{1}{c^2} \vc{V}_\oplus \cdot ( \vc{r}_1(t_1) - \vc{r}_2(t_1) )
                 \biggl( 1 + \Frac{1}{2c} \vc{V}_\oplus \cdot \vce{S} \biggr)
                 \: - \:
                 \Frac{1}{c} \, \rho \, ( \vc{r}_1(t_1) - \vc{r}_2(t_1) ) \,
                 \Frac{\vc{R}_\oplus}{\bar{D}_\oplus} \: + \:
                 \tau_{grav} \: + \: \tau_p 
            \Biggr) \: - \:
\\ \hspace{1em}
            \Frac{1}{c^2}
            \Biggl( g_{loc,2} \, h_{ort,2} - g_{loc,1} \, h_{ort,1} + 
                    \Frac{2}{3} R_\oplus \Omega_\oplus^2 \
                    \bigl( \cos^2 \phi_{gc,2} \, h_{ort,2} - 
                           \cos^2 \phi_{gc,1} \, h_{ort,1} 
                    \bigr)
            \Biggr) \cdot (t_{1p} - t_{sync})
   \end{array}
\eeq{e:e14}
%
  where $\rho$ is the annual parallax of the observed source, $\phi$ is
geocentric latitude, and $ \bar{D}_\oplus $ is the mean distance between 
the barycenter of the Solar System and the barycenter the 
system Earth--Moon (astronomical unit). We can neglect dependence of 
the gravitational potential on height for ground stations and compute it as 
$U_\oplus = \Frac{f M_\oplus}{ | r_{\oplus}| }$ and omite $v_2^2/c^2$
for ground stations.

  The time delay is itself a function of time. The argument of its time 
dependence is the geocentric time $t_1$ of the event of wavefront coming 
to reference station \#1. Differentiating this expression with respect 
to time and discarding terms which are less than $ 10^{-16}$, we get

\beq
   \begin{array}{l}
%%      \Ddt{(t_2 - t_2)} = \\ \hspace{1em}
      \Ddt{}\biggl((t_{2p} - t_{sync}) - (t_{1p} - t_{sync})\biggr)(t_1) = \\ 
      \hspace{1em}

%
      \Frac{1}{ 1 + \Frac{1}{c} (\vc{V}_\oplus + \dvc{r}_2 ) \cdot \vce{S} }
              \;
%
      \Biggl(
            \Frac{1}{c} ( \dvc{r}_1 - \dvc{r}_2 ) \cdot \vce{S} \;
            \biggl[1 -
               \biggl(2 \Frac{f M_\odot} { |\vc{R}_\oplus| c^2 } - L_b \biggr) -
               \biggl( \Frac{U_\oplus}{c^2}- L_g \biggr) -
               \Frac{|\vc{V}_\oplus|^2}{2c^2} -
               \Frac{\vc{V}_\oplus \cdot \dvc{r}_2}{c^2}
            \biggr] + \\ \hspace{1em}
%
            \Frac{1}{c^2} \lp \dvc{V}_\oplus \cdot ( \vc{r}_1 - \vc{r}_2 ) +
                              \vc{V}_\oplus  \cdot ( \dvc{r}_1 - \dvc{r}_2 ) \rp
                 \biggl( 1 + \Frac{1}{2c} \vc{V}_\oplus \cdot \vce{S} \biggr) -
                 \Frac{1}{c} \, \rho \, ( \dvc{r}_1 - \dvc{r}_2 ) \,
                 \Frac{\vc{R}_\oplus}{\bar{D}_\oplus} +
                 \ddt{\tau_{grav}} + \ddt{\tau_p}
      \Biggr) - \\ \hspace{1em}
      \Frac{1}{c}\Frac{ ( \dvc{V}_\oplus  + \ddvc{r}_2 ) \cdot \vce{S} }
                      {\Biggl( 1 + \Frac{1}{c} ( \vc{V}_\oplus + \dvc{r}_2 )
                              \cdot \vce{S} \Biggr)^2 }
      \Biggl(
            \Frac{1}{c} ( \vc{r}_1 - \vc{r}_2 ) \cdot \vce{S} +
            \Frac{1}{c^2} \vc{V}_\oplus \cdot ( \vc{r}_1 - \vc{r}_2 ) +
            \Frac{1}{c} \, \rho \, ( \vc{r}_1 - \vc{r}_2 ) \,
            \Frac{\vc{R}_\oplus}{\bar{D}_\oplus} +
            \tau_{grav} + \tau_p
      \Biggr) - \\ \hspace{1em}
      \Frac{1}{c^2}
      \Biggl( g_{loc,2} \, h_{ort,2} - g_{loc,1} \, h_{ort,1} + 
              \Frac{2}{3} R_\oplus \Omega_\oplus^2 \
              \bigl( \cos^2 \phi_2 \, h_{ort,2} - \cos^2 \phi_1 \, h_{ort,1} 
              \bigr)
       \Biggr) 
   \end{array}
\eeq{e:e15}
%
and
%
\beq
   \begin{array}{lcl}
    \Ddt{\tau_{grav}} & = &
        2 \dss\sum_{k=1} \Frac{f \, M_k}{c^3} \lp
                                  1 + \vce{S}\cdot \ddvc{R}_k(T'_{1k}) \rp \;
          \ln  \Frac{|\vc{R}_k(T'_k)   - \vc{R}_1| + \vce{S} \cdot
                     (\vc{R}_k(T'_{k}) - \vc{R}_1 ) }
                    {|\vc{R}_k(T'_k)   - \vc{R}_2| + \vce{S} \cdot
                      (\vc{R}_k(T'_{k}) - \vc{R}_2 ) } \: +
     \\ & &
         2 \dss\sum_{k=1} \Frac{f \, M_k}{c^3} \lp
                                  1 + \vce{S}\cdot \dvc{R}_k(T'_{1k}) \rp \cdot
     \\ & &
           \Bigggl( \Frac{ - \Frac{ \dvc{R}_k(T'_k) \cdot \vc{R}_1 +
                                    \vc{R}_k(T'_k)  \cdot \dvc{R}_1 }
                          {|\vc{R}_k(T'_k) - \vc{R}_1|} +
                          (\dvc{R}_k(T'_k)   - \dvc{R}_1) \cdot \vec{S}
                    }
                    {|\vc{R}_k(T'_k) - \vc{R}_1| +
                     (\vc{R}_k(T'_k)   - \vc{R}_1) \cdot \vec{S}
                    } \: - \:
     \\ & &
           \hphantom{\Bigggl(}
                     \Frac{ - \Frac{  \dvc{R}_k(T'_k) \cdot \vc{R}_2 +
                                       \vc{R}_k(T'_k)  \cdot \dvc{R}_2 }
                          {|\vc{R}_k(T'_k) - \vc{R}_2|} +
                          (\dvc{R}_k(T'_k)   - \dvc{R}_2) \cdot \vec{S}
                    }
                    {|\vc{R}_k(T'_k) - \vc{R}_2| +
                     (\vc{R}_k(T'_k)   - \vc{R}_2) \cdot \vec{S}
                    }
           \Bigggr)
   \end{array}
\eeq{e:e16}

  In practice, the last term, linear in time for time delay and constant
for delay rate is usually ignored since it is not distingioshable from
the Hydrogen clock frequency offset. This term is solved for during 
post-processing dat aanalysis.

\subsubsection{VLBI time delay when one of antenna is at the Earth's orbit}
\label{s:orbit}

  One of the elements of the radiointerferometer may be at the Earth's orbit.
Equations for time delay and its time derivatives \ref{e:e14}--\ref{e:e16}
remain valid for this case, except the last term that accounts for difference
in clock rate. The satellite orbit should be transformed to the geocentric 
coordinate system with the same expression for metric as for coordinates 
of ground stations before computations.

  In the case if the orbiting station has the clock that preserves its count
through the entire experiment, the path delay is computed the same way
as for the baseline between ground stations, except the fact that the ground 
and the orbiting station have different model of their motion.

  The orbiting station may or may not have a continuous time counter. 
For instance, Radioastron has its on-board Hydrogen maser clock that feeds 
both the receiver and the sampler, but the sample counter is implicitly
reset at the beginning of each scan (Y.~Kovalev (2012), private communication).
The reading of the clock of the ground downlink station is written
in the time tag field in the scan header of data record of the orbiting 
telescope. It should be stressed that the delay is still determined as 
a difference of intervals proper time between events of coming the wavefront 
to the station and clock synchronization measured with station clocks for 
both ground station and orbiting station. It is the on-board clock that 
generates the rail of samples and send the digitized samples down to
the Earth. The interval of time of each sample relative to the first sample
of a scan is the proper time of the on-board clock.

  Clock synchronization usually done only once per experiment for ground 
stations. For the orbiting station without a continuous time counter, clock 
synchronization is done at the beginning of each scan. The difference 
between the time coordinate at the orbiting station $t_s$ and the time 
coordinate at the downlink station $t_d$ for the observer at the downlink
station can be found by solving the light-cone equation
%
\beq
   t_s - t_d = \Frac{1}{c} | \vc{r}_s(t_d + (t_s - t_d)) - \vc{r}_d |
\eeq{e:e16a}
%
  where $\vc{r}_d$ is the position of the downlink station in the geocentric
coordinate system and $\vc{r}_s$ is the position of the satellite
in the geocentric coordinate system according to its ephemeride.

  This equation is solved by iterations. Then the time tag of the first
sample of a scan from the orbiting telescope is $t_d - (t_s - t_d)$. 
For ground station the a~priori clock model is computed from results 
of clock synchronization before and after experiment. The difference between
the sampler clock minus TAI is approximated with a liner function and 
subtracted from the geometric path delay. For the orbiting station without 
a continuous time counter, clock of the downlink station is
synchronized against TAI using the same way as for other ground stations.
Then the delay between the orbiting and the downlink station,
$t_s - t_d$, considered constant for each scan, is added to the 
geometric delay. This difference represents the clock synchronization 
errors and is constant for every sample of a scan. For the next scan
a new difference $t_s - t_d$ is computed for the time of recording
the first sample.

  In the case if the second station of a baseline is on the orbit,
$\tau \, \vec{v}^2_2/c^2$ is not negligible and $U\earth$ cannot be 
computed as $\Frac{fM\earth}{R\earth}$ in expression 
\ref{e:e14}--\ref{e:e15}. We need to add the following terms
$\tau_a$ and $\Ddt{}\tau_a$ to expressions \ref{e:e14}--\ref{e:e15}:

\beq
  \begin{array}{lcr}
     \tau_a & = & - \Frac{\vec{v}^2_2}{2 c^2} ( \vc{r}_1 - \vc{r}_2 ) \cdot \vce{S} 
                \: + \:
                \Frac{fM\earth}{c^2}\lp \frac{1}{|\vc{r}_2|} - 
                     \frac{1}{R\earth} \rp \, 
                \vc{r}_2 \cdot \vce{S}      \\
%
   \Ddt{} \tau_a & = & - \Frac{\vec{v}^2_2}{2 c^2} 
                            ( \dvc{r}_1 - \dvc{r}_2 ) \cdot \vce{S} 
                \: + \:
                \Frac{fM\earth}{c^2}\lp \frac{1}{|\vc{r}_2|} - 
                     \frac{1}{R\earth} \rp \, 
                \dvc{r}_2 \cdot \vce{S}
  \end{array}
\eeq{e:e16b}


  In the case if the first station of a baseline is on the orbit,
corrections $\tau_b$ and $\Ddt{}\tau_b$ should be added to 
expressions \ref{e:e14}--\ref{e:e15}:

\beq
  \begin{array}{lcr}
     \tau_b & = & 
                - \Frac{fM\earth}{c^2}\lp \frac{1}{|\vc{r}_1|} - 
                        \frac{1}{R\earth} \rp \, 
                  \vc{r}_1 \cdot \vce{S}      \\
%
   \Ddt{} \tau_b & = & 
                 - \Frac{fM\earth}{c^2}\lp \frac{1}{|\vc{r}_1|} - 
                         \frac{1}{R\earth} \rp \, 
                   \dvc{r}_1 \cdot \vce{S}
  \end{array}
\eeq{e:e16c}

  In a case of a ground station term 
$ \Frac{1}{c^2} \, \dss \int \biggl( \Frac{1}{2} v^2 + U \biggr) \, dt $
is reduced to a small linear function with rate approximately 
$10^{-16} \cdot h_{ort}$. Since velocity and distance to the spacecraft 
is changing during experiment, this term cannot be reduced. Instead,
we have to integrate this expression
%
\beq
  \Delta t = \dss \Frac{1}{c^2} \int\limits_{t_{s,sync}}^{t_{s,obs}} 
       \biggl(  \Frac{1}{2} \dot{r}_s^2(t) + \Frac{fM_\oplus}{|r_s(t)|}
              - \Frac{f M_\oplus}{R_\oplus} - \Frac{2}{3}\Omega_\oplus^2 R_\oplus^2
              + fM_{moon}\biggl(\Frac{1}{|r_s(t)|} - \Frac{1}{|R_\oplus|} \biggr)
       \biggr) \, dt
\eeq{e:e16d}
%
  using satellite ephemerides. I emphasize here in the lower and upper 
limits that the time coordinate at the orbiting satellite should be used 
that is related to downlink station time $t_d$ via expression \ref{e:e16a}.

  For observations with the orbiting station that does not have a 
sample counter, clock synchronization occurs at the nominal start time. 
Term $\Delta t$ is either added if the orbiting station is 
a reference station \#1 or subtracted if the orbiting station is a remote
station \#2.

\subsubsection{VLBI time delay for a near zone object}

  If the object is in the near zone, then the most straightforward way
of solving equations \eref{e:e4} is the method of consecutive
iterations. At the first iteration we set the the right-hand side
$T_{ap} := T_1$, \quad $T'^{p}_k := T_1$ and $ T_{2p} := T_2$. Then

\beq
\begin{array}{ll}
  T_e & := T_{ap} \; - \; \Frac{1}{c}
                  \biggl| \vc{R}_e(T_{ap}) - \vc{R}_1(T_1) \biggr| \; - \;
           T^p_{grav_1}  \\
%
  T'_k & := T_1 - \Frac{1}{c} \biggl| \vc{R}_1(T_1) - \vc{R}_k(T'^{p}_k) \biggr| \\
%
  T^{p}_{grav_1} & =  \dss \sum_k \Frac{2  f  M_k }{c^3} \log \Biggl|
    \Frac{ | \vc{R}_k(T'^{p}_k) - \vc{R}_e(T_{ap})| \; + \;
           | \vc{R}_k(T'^{p}_k) - \vc{R}_1(T_1)| \; + \;
           | \vc{R}_e(T_e)  - \vc{R}_1(T_1)| }
%
         { | \vc{R}_k(T'^{p}_k) - \vc{R}_e(T_{ap})| \; + \;
           | \vc{R}_k(T'^{p}_k) - \vc{R}_1(T_1)| \; - \;
           | \vc{R}_e(T_e)      - \vc{R}_1(T_1)| \:    } \Biggr| \vspace{1ex} \\
%
  T^{p}_{grav_2} & =  \dss \sum_k \Frac{2  f  M_k }{c^3} \log \Biggl|
    \Frac{ | \vc{R}_k(T'^{p}_k) - \vc{R}_e(T_{ap})| \; + \;
           | \vc{R}_k(T'^{p}_k) - \vc{R}_2(T_2)| \; + \;
           | \vc{R}_e(T_e)  - \vc{R}_2(T_2)| }
%
         { | \vc{R}_k(T'^{p}_k) - \vc{R}_e(T_{ap})| \; + \;
           | \vc{R}_k(T'^{p}_k) - \vc{R}_2(T_2)| \; - \;
           | \vc{R}_e(T_e)      - \vc{R}_2(T_2)| \:    } \Biggr| \\
%
  T_2 & := T_e + \Frac{1}{c} \biggl| \vc{R}_e(T_e) - \vc{R}_2(T_{2p}) \biggr| +
           T_{grav_2}
\end{array}
\eeq{e:e17}
%

  Without a non-negligible loss of accuracy we can compute
$ \vc{R}_k(T'^{p}_k) $ and $ \vc{R}_2(T_{2p}) $ as
%
\beq
  \vc{R}_k(T'^{p}_k) = \vc{R}_k(T_1) \; + \;
                        \dot{\vc{R}}(T_1) \, (T'^{p}_k - T_1)
%
   \nonumber \\
%
  \vc{R}_2(T_{2p})   = \vc{R}_2(T_2) \; + \;
                        \dot{\vc{R}}(T_2)  \, (T_{2p} - T_2)
\eeq{e:e18}
%
  where $\vc{R}_k(T_1)$ and $\vc{R}_2(T_2)$ are computed from geocentric
station coordinates using equations \ref{e:e3}.

  In general, we have to use the ephemerides of the emitter at each step of
iteration in order to get its precise position at a new moment of time.

  There is a special case which should be handled separately: when of the
receivers is located in the geocenter, the expression for metric
\eref{e:e1}--\eref{e:e2} is not valid and if used formally, a singularity
occurs. It should be noted that modeling of the receiver in the geocenter
does not a physical meaning, and this delay occurs only in intermediate
computations. Therefore, it can be set to an arbitrary value. It is set
to zero in the present algorithm.

\bigskip

%%  The baricenteric time delay is $ (T_2 - T_e) - (T_1 - T_e) $. Care should be
%%taken in computing this difference, since a catastrophic loss of precision
%%may occur in subtracting two big quantities even in computation with double
%%precision. One way is to use the quad precision in computing the difference
%%
%%\beq
%%  \biggl| \vc{R}_e(T_e) - \vc{R}_1(T_1) \biggr| -
%%  \biggl| \vc{R}_e(T_e) - \vc{R}_2(T_2) \biggr|
%%\eeqn
%%
%%  Another way is to expand this difference in the series.
%%If $ |\vc{A}| \gg |\vc{b}|$, and $ |\vc{A}| \gg |\vc{c}|$ the difference
%%$ |\vc{A} - \vc{b} | - |\vc{A} - \vc{c} | $ can be expanded this way:
%%%
%%\beq
%%  \begin{array}{ll}
%%  |\vc{A} - \vc{b} | - |\vc{A} - \vc{c} | \approx &
%%    \Frac{|\vc{A}|}{2}(\beta - \gamma) \biggl[
%%    1 \: - \: c_1\, (\beta + \gamma) \: + \:
%%    c_2\, (\beta^2 + \beta\, \gamma + \gamma^2) +
%%%
%%  \\ &
%%%
%%    c_3\, (\beta^3 + \beta^2\, \gamma + \beta \, \gamma^2 + \gamma^3) \: - \:
%%    c_4\, (\beta^4 + \beta^3\, \gamma + \beta^2 \, \gamma^2 + \beta \, \gamma^3
%%           + \gamma^4 ) + \ldots \biggr]
%%  \end{array}
%%\eeq{e:e8}
%% where $ \beta =    \Frac{b^2}{A^2} \; - \;
%%                  2 \Frac{\vc{A}}{|A|} \, \Frac{\vc{b}}{|A|}$, \quad
%%       $ \gamma =   \Frac{c^2}{A^2} \; - \;
%%                  2 \Frac{\vc{A}}{|A|} \, \Frac{\vc{c}}{|A|}$,
%%and
%%
%%\noindent
%%$
%%  c_1 = \Frac{1}{4},         \quad
%%  c_2 = \Frac{1}{2} \, c_1,  \quad
%%  c_3 = \Frac{5}{4} \; \Frac{1}{2} \, c_2,  \quad
%%  c_4 = \Frac{7}{5} \; \Frac{1}{2} \, c_3,  \quad
%%  c_5 = \Frac{9}{6} \; \Frac{1}{2} \, c_4,
%%$ \enskip etc.

  The barycentric delay is formulated as a difference of barycentric time
coordinates of events of emittance the radio wave and its receiving.
However, we should remember that coordinates are not measurable quantities.
They should be transformed to intervals of proper time. The VLBI time delay
which emerges in analysis of fringe phases is defined as the difference
of two intervals of proper time: 1) the interval of proper time of station
\#2 between events: coming the wave front to the reference point on the
moving axis and clock synchronization; 2) the interval of proper time
of station \#1 between events: coming the wave front to the reference point
on the moving axis and clock synchronization. The time delay is referred
to the moment of coming the wave front to the reference point on the moving
axis of the first antenna at time measured by the time-scale TAI.
The reference point of the station for which modeling is done is defined
as the point on the moving axis which has the minimal distance to the
fixed axis. In the  case if axes intersect, this is the point of their
intersection. The differences of barycentric time coordinates
$ T_1  - T_{sync} $ and $ T_2 - T_{sync} $ are first to transformed
to geocentric coordinate system, then these differences of geocentric
time coordinates are transformed to intervals of proper time using this
expression which follows from \eref{e:e1}--\eref{e:e2}:
%
\beq
   (t_{2p} - t_{sync}) - (t_{1p} - t_{sync}) = &
   \biggl( 1 - \Bigl( \Frac{U\sun}{c^2}   - L_b + \Frac{V\earth^2}{2\,c^2} \Bigr)
             - \Bigl( \Frac{U\earth}{c^2} - L_g + \Frac{v_1\earth^2}{2\,c^2}\Bigr) \biggr) \;
   ( T_2 - T_1 ) + \nonumber \\ &
   \Frac{ (\vc{r}_1(t_1) - \vc{r}_2(t_2)) \, \vc{V}_{\oplus} }{c^2} \; + \;
   a \; + \; b \, (t_1 - t_{sync})
\eeq{e:e19}
%
  where $a$ and $b$ are some quantities. Their value is irrelevant, since
they cannot be distinguished from errors of clock model.

\subsection{Computation of station position vector in geocentric inertial
coordinate system}

  Time delay depends on the baricenteric position vector of the emitter and
the geocentric position vector in the inertial coordinate system. The
geocentric system is considered kinematicly non-rotating with respect
to the barycentric coordinate system and therefore, can be labeled as
a ``celestial coordinate system''. When we describe site position, it is
convenient to relate it to the Earth crust fixed coordinate system, because
in that system the motion of the antenna reference point is slow. This
coordinate system is labeled as a ``terrestrial coordinate system''. The
transformation of the position vector from the terrestrial coordinate
$ \vc{r}_{\tiny\sc t} $ system to the celestial coordinate system
$ \vc{r}_{\tiny\sc c} $ can be represented as
%
\beq
   \vc{r}_{\tiny\sc c} = \mat{M}(t) \, \vc{r}_{\tiny\sc t}    \: + \:
                         \vc{q}(t) \times \vc{r}_{\tiny\sc t} \: + \: 
                         \vc{d}_{\tiny\sc t}(t) 
\eeq{e:e20}
%
  where $ \mat{M} $ is the apriori rotation matrix, $ \vc{q} $ --- the small
vector of perturbing rotation, $ \vc{d}_{\tiny\sc t} $ --- the vector of site
motion the terrestrial coordinate system. The site motion can be decomposed
on 1) secular motion due to plate tectonic; 2) harmonic site position
variations caused by a) solid Earth tides; b) ocean loading; c) atmospheric
pressure loading; d) hydrology loading; e) empirical deformations; 3) motion
of the antenna reference due to antenna slewing.

\subsubsection{Reduction for the Earth's rotation}

  Decomposition of the Earth rotation into the vector of small perturbing
rotation $ \vc{q}(t) $ and the matrix of finite rotation $ \mat{M} $ is not
unique, and there are different ways to perform it. The rotation vector
$ \vc{q}(t) $ is small in the sense that one can neglect squares of its
components.

  Three combinations of $ \vc{q}(t) $ and $ \mat{M} $ are considered:
\begin{itemize}
   \item  Newcomb-Andoyer formalism
      \beq
         \begin{array}{l@{}l}
            \vc{q}  & = 0 \\
            \mat{M} & =
%
            \widehat{\mathstrut\cal R}_3(\zeta_0)  \cdot
            \widehat{\mathstrut\cal R}_2(\theta_0)  \cdot
            \widehat{\mathstrut\cal R}_3(z)  \cdot
            \widehat{\mathstrut\cal R}_1(-\epsilon_0)  \cdot
            \widehat{\mathstrut\cal R}_3(\Delta\psi)  \cdot
            \widehat{\mathstrut\cal R}_1(\epsilon_0 + \Delta\epsilon) \cdot
            \widehat{\mathstrut\cal R}_3(-S) \cdot
            \widehat{\mathstrut\cal R}_2(X_p)
            \widehat{\mathstrut\cal R}_1(Y_p)
         \end{array}
       \eeq{e:e21}
%
   \item  Ginot-Capitaine formalism:
      \beq
         \begin{array}{l@{}l}
            \vc{q}  & = 0 \\
            \mat{M} & =
%
	    \mat{R}_3(-E) \cdot
	    \mat{R}_2(-d) \cdot
	    \mat{R}_3( E) \cdot
	    \mat{R}_3( s) \cdot
	    \mat{R}_3(-\theta) \cdot
	    \mat{R}_3(-s') \cdot
	    \mat{R}_1(Y_p) \cdot
	    \mat{R}_2(X_p)
         \end{array}
      \eeq{e:e22}
%
   \item  Empirical Earth Rotation Model (EERM):
%
     \beq
       \vc{q}(t) = & \left(
       \begin{array}{ll}
          \displaystyle\sum_{k=1-m}^{n-1} f_{1k} \, B_k^m(t) \: + &
          \displaystyle\sum_{j}^{N} \left( P^c_{j} \cos \omega_m \, t \: + \:
                                        P^s_{j} \sin \omega_j \, t \right) \:
          \vspace{0.5ex} \\
           & + \, t \, \left( S^c_{j} \cos -\Omega_n \, t \: + \:
                        S^s_{j} \sin -\Omega_n \, t \right)
          \vspace{0.5ex} \vspace{2ex} \\
%
          \displaystyle\sum_{k=1-m}^{n-1} f_{2k} \, B_k^m(t) \: + &
          \displaystyle\sum_{j}^{N} \left( P^c_{j} \sin \omega_j \, t \: - \:
                             P^s_{j} \cos \omega_j \, t \right) \:
          \vspace{0.5ex} \\
          & + \, t \, \left( S^c_{j} \sin -\Omega_n \, t \: - \:
                     S^s_{j} \cos -\Omega_n \, t \right)
          \vspace{0.5ex} \vspace{2ex} \\
%
          \displaystyle\sum_{k=1-m}^{n-1} f_{3k} \, B_k^m(t) \: + &
          \displaystyle\sum_{j}^{N} \left( E^c_{j} \cos \omega_j \, t +
                                           E^s_{j} \sin \omega_j \, t \right)
          \\
       \end{array}
       \right) \\
            \mat{M} = &
            \widehat{\mathstrut\cal R}_3(\zeta_0)  \cdot
            \widehat{\mathstrut\cal R}_2(\theta_0)  \cdot
            \widehat{\mathstrut\cal R}_3(z)  \cdot
            \widehat{\mathstrut\cal R}_1(-\epsilon_0)  \cdot
            \widehat{\mathstrut\cal R}_3(\Delta\psi_e)  \cdot
            \widehat{\mathstrut\cal R}_1(\epsilon_0 + \Delta\epsilon_e) \cdot
            \widehat{\mathstrut\cal R}_3(-S)
       \nonumber
     \eeq{e:e23}
\end{itemize}

  All three approaches provide the same transformation with the same accuracy.
Only two formalisms Newcomb-Andoyer and the EERM approaches are considered
here.

  Parameters of the Newcomb-Andoyer formalism:
\beq
   \begin{array}{lcl}
      \zeta_0    & = & \zeta_{01} \, t + \zeta_{02} \, t^2 + \zeta_{03} \, t^3 \\
%
      \theta_0   & = & \theta_{01} \, t + \theta_{02} \, t^2 + \theta_{03} \, t^3 \\
%
       z         & = & z_1 \, t + z_2 \, t^2  + z_3 \, t^3  \\
%
      \epsilon_0 & = & \epsilon_{00} + \epsilon_{01} \, t +
                       \epsilon_{02} \, t^2                         \\
%
      \Delta\psi & = & \displaystyle \sum_i^l
                       \biggl(\: ( \Psi_i^{pro} + {\Psi'}_i^{pro} T_J ) \:
                       \: \sin \: \Bigl( \dss\sum_j^m k_{ji} a_j(T_J) \Bigr) + \\
                       & & \hphantom{\sum \biggl(\:\:\: \biggr.}
                       ( \Psi_i^{ret} + {\Psi'}_i^{ret} T_J )
                         \cos \Bigl( \dss\sum_j^m k_{ji} a_j(T_J) \Bigr)
                         \biggr) + \Psi_0 + \dot{\Psi}_0 T_J +
                         \delta\psi_{geod} \\
%
      \Delta\epsilon
                 & = & \displaystyle \sum_i^l
                       \biggl(\: ( E_{pi}^{pro} + {E'_{pi}}^{pro} T_J ) \:
                       \: \cos \: \Bigl( \dss\sum_j^m k_{ji} a_j(T_J) \Bigr) + \\
                       & & \hphantom{\sum \biggl(\:\:\: \biggr.}
                       ( E_{pi}^{ret} + {E'_{pi}}^{ret} T_J )
                         \sin \Bigl( \dss\sum_j^m k_{ji} a_j(T_J) \Bigr)
                         \biggr) + E_{p0} + \dot{E}_{p0} T_J +
                         \delta\epsilon_{geod} \\
%
      S   & = & S_0 + (\Omega_n + \zeta_{01} + z_1 ) \, t +
                    (\zeta_{02} + z_2 ) \, t^2 +
                    (\zeta_{03} + z_3 - \Frac{1}{6} \theta^2 ) \, t^3  \\
%
          &   & - (\zeta_{01} + z_1 ) \,
                   \dss\int\limits_{t_o}^{t} \Delta\epsilon(t) \, dt -
                   \sin\epsilon_o \dss\int\limits_{t_o}^{t} \Delta\psi'(t) \,
                                  \Delta\epsilon(t) \, dt \\
          &   & +   \Delta\psi \, \cos\epsilon_0 +
                    \dss \kappa \mbox{UT1}(t) \: + A_s \: + B_s \, t
   \end{array}
\eeq{e:e24}
%
  $a_j$ are so-called fundamental arguments represented by lower degree
polynomials,  The last two ad hoc spurious terms may be included
for no particular reason. The geodesic nutation $ \delta\psi_{geod},
\delta\epsilon_{geod}, $
is represented in the form of expansion
%
\beq
  \begin{array}{lcl}
     \delta\psi_{geod}     & = & \dss\sum_{i=1}^{i=6} - A_{ng} \sin
        ( \phi_i + (f_i + \Omega_n + z_1 + \zeta_{01} + \dot{\Psi}_0 ) \, t ) \\
     \delta\epsilon_{geod} & = & \dss\sum_{i=1}^{i=6} \hpm A_{ng} \cos
        ( \phi_i + (f_i + \Omega_n + z_1 + \zeta_{01} + \dot{\Psi}_0 ) \, t )
  \end{array}
\eeq{e:e25}
%
  The integral of cross precession-nutation and nutation-nutation terms
\hspace{20em} \linebreak
$ - \dss\int\limits_{t_o}^{t} \lp (z_{1} + \zeta_{01}) \Delta\epsilon(t) \: + \:
  \Delta\psi'(t) \, \Delta\epsilon(t) \, \sin\epsilon_o \rp dt $ have the
secular and periodic terms. The secular with term truncated to the
$10^{-12}$~rad level has a form
%
\beq
   B_{nn} T_J = \Frac{1}{2} \dss \sum_i^l \, \sum_j^m k_{ji} \,
                \tss\ddt{a}(T_J) \,
                \Psi^{pro}_i \,  E^{pro}_i \, \sin \epsilon_0 \, T_J
\eeq{e:e26}
%
  Other periodic terms can be presented in the form
\beq
      \rho(T_J) = \rho_c \cos ( \phi_{\rho} + f_{\rho} T_J ) +
                  \rho_s \sin ( \phi_{\rho} + f_{\rho} T_J )
\eeq{e:e27}

  To summarize, the matrix of the transformation according to the
Newcomb-Andoyer formalism using the MHB2000 semi-empirical nutation
expansion depends on
%
\begin{itemize}
   \item 9 precession parameters: $ \zeta_{0i}, \theta_{0i}, z_i$;
   \item 3 parameters of expansion of the angle of inclination of the ecliptic
         to the equator $\epsilon_{0i}$;
   \item $ 14\times 3 = 42 $ low degree coefficients of expansion of fundamental
           arguments over time;
   \item $ 14\times 1365 = 19110 $ integer multipliers $k_{ij}$;
   \item $ 1365 \Psi^{pro}  $ coefficients of the prograde nutation in
                               longitude;
   \item $ 1365 \Psi'^{pro} $ coefficients of the rate of change of the
                               prograde nutation in longitude;
   \item $ 1365 \Psi^{ret}  $ coefficients of the retrograde nutation
                               in longitude;
   \item $ 1365 \Psi'^{ret} $ coefficients of the rate of change of
                               the retrograde nutation in longitude;
   \item $ 1365 E^{pro}  $ coefficients of the prograde nutation in obliquity;
   \item $ 1365 E'^{pro} $ coefficients of the rate of change of the prograde
                            nutation in obliquity;
   \item $ 1365 E^{ret}  $ coefficients of the retrograde nutation in
                            obliquity;
   \item $ 1365 E'^{ret} $ coefficients of the rate of change of the
                            retrograde nutation in longitude;
   \item $ S_0 $ --- nominal position angle of the Earth at J2000;
   \item $ \Omega_n $ --- nominal angular velocity of the Earth;
   \item $ B_{nn} $ --- secular term of the nutation-nutation cross-terms;
   \item $ 6\times 3 = 18 $ phases and frequencies, and amplitudes of the
                            geodesic nutation expansion;
   \item $ 15\times 4 = 60 $ phases and frequencies, and amplitudes of periodic
                             nutation-precession and nutation-nutation cross
                             terms.
   \item $ 2 $ spurious ad~hoc terms $A_s$ and $B_s$.
   \item $ \kappa $ --- scaling parameter of the UT1 function
\end{itemize}

  In total, 30168 parameters! In addition to that, three empirical
functions UT1(t), $X_p$(t), and $Y_p$(t) describes non-predictable part of
the transformation. These empirical functions are typically sampled with a step
of 1 day, which requires for describing the Earth's rotation at the period
of 20 years, $365 \times 3 \times 20 = 21900$ terms. The transformation in
this form does not account for the free core nutation, and therefore, has
the accuracy not exceeding $10^{-9}$, even if empirical the functions UT1(t),
$X_p$(t), and $Y_p$(t) would have been known with an infinite accuracy.

  Parameters of the EERM formalism:
%
\begin{eqnarray}
   \begin{array}{lcl}
      \zeta_0    & = & \zeta_{01} \, t + \zeta_{02} \, t^2 \\
%
      \theta_0   & = & \theta_{01} \, t + \theta_{02} \, t^2 \\
%
       z         & = & z_1 \, t + z_2 \, t^2  \\
%
      \epsilon_0 & = & \epsilon_{00} + \epsilon_{01} \, t +
                       \epsilon_{02} \, t^2                         \\
%
      \Delta\psi & = & \displaystyle \sum_j^3
                       -n_j \, \sin ( \phi^{n}_{j} + \omega^{n}_{j} \, t +
                                      \dot{\omega}^{n}_{j} \, t^2/2 )
                       \, / \, \sin\epsilon_0  \\
%
      \Delta\epsilon
                 & = & \displaystyle \sum_j^3 \hphantom{-}
                        n_j \, \cos ( \phi^{n}_{j} + \omega^{n}_{j} \, t +
                                      \dot{\omega}^{n}_{j} \, t^2/2 )  \\
%
      S   & = & S_0 + E_0 + (\Omega_n + \zeta_{01} + z_1 + E_1 ) \, t +
                    (\zeta_{02} + z_2 + E_2 ) \, t^2 + \\
          &   & - (\zeta_{01} + z_1 ) \, \dss
                   \int\limits_{t_o}^{t} \Delta\epsilon(t) \, dt -
                   \sin\epsilon_o \int\limits_{t_o}^{t} \Delta\psi'(t) \,
                                  \Delta\epsilon(t) \, dt \\
          &   & +   \Delta\psi \, \cos\epsilon_0 +
                    \displaystyle
                    \sum_i^2 \left( E^c_i \cos \omega^{e}_{i} \,t +
                                    E^s_i \sin \omega^{e}_{i} \, t \right)
   \end{array}
\eeq{e:e28}
%
\begin{itemize}
   \item 6 precession parameters: $ \zeta_{0i}, \theta_{0i}, z_i$;
   \item 3 parameters of expansion of the angle of inclination of the ecliptic
         to the equator $\epsilon_{0i}$;
   \item $4\times 3 = 12$ parameters of expansion of $\delta\psi$ and
         $\delta\epsilon$;
   \item $ S_0 $ --- nominal position angle of the Earth at J2000;
   \item $ \Omega_n $ --- nominal angular velocity of the Earth;
   \item $3 $ parameters of $E_0, E_1, E_2$ of empirical shift, drift and
              acceleration of the apriori $E_3$ Euler angle;
   \item $ B_{nn} $ --- secular term of the nutation-nutation cross-terms;
   \item $ 5\times 4 = 20 $ phases and frequencies, and amplitudes of periodic
                             nutation-precession and nutation-nutation cross
                             terms.
   \item $ 3\times 2 = 6 $ frequencies and the amplitudes a~priori terms
                           of the harmonic expansion of the $E_3$ Euler angle.
\end{itemize}
%
   In total, 53 parameters. The empirical function $\vc{q}_e(t)$ consists of
approximately 500 terms of expansion into the Fourier basis, i.e.
$500 \times 5=2500$ parameters and 12100 parameters of the expansion into
the B-spline basis over the 20 year period
($20\times 365 \times (1+2/3)=12100$), considering the time span for the
B-spline basis 3 days for the first two components of the probating
rotation vector $\vc{q}_e(t)$ and 1 day for the third component.

  Computation of station velocity and acceleration requires the first and
the second time derivative of vector $\vec{q}_e$ and matrix $\mat{M}$.
The first derivative of the matrix $\mat{M}$ is computed as
%
\beq
   \begin{array}{l}
     \Frac{\partial}{\partial t} \mat{M}(t) = \\ \hspace{2em}
             \matd{R}_3(\zeta_0)  \cdot
             \mat{R}_2(\theta_0)  \cdot
             \mat{R}_3(z)  \cdot
             \mat{R}_1(-\epsilon_0)  \cdot
             \mat{R}_3(\Delta\psi)  \cdot
             \mat{R}_1(\epsilon_0 + \Delta\epsilon) \cdot
             \mat{R}_3(-S) \cdot
             \mat{R}_2(-X_p) \cdot
             \mat{R}_3(Y_p) \: - \\ \hspace{2em}
%
             \mat{R}_3(\zeta_0)  \cdot
             \matd{R}_2(\theta_0)  \cdot
             \mat{R}_3(z)  \cdot
             \mat{R}_1(-\epsilon_0)  \cdot
             \mat{R}_3(\Delta\psi)  \cdot
             \mat{R}_1(\epsilon_0 + \Delta\epsilon) \cdot
             \mat{R}_3(-S) \cdot
             \mat{R}_2(-X_p) \cdot
             \mat{R}_3(Y_p) \: + \\ \hspace{2em}
%
             \mat{R}_3(\zeta_0)  \cdot
             \mat{R}_2(\theta_0)  \cdot
             \matd{R}_3(z)  \cdot
             \mat{R}_1(-\epsilon_0)  \cdot
             \mat{R}_3(\Delta\psi)  \cdot
             \mat{R}_1(\epsilon_0 + \Delta\epsilon) \cdot
             \mat{R}_3(-S) \cdot
             \mat{R}_2(-X_p) \cdot
             \mat{R}_3(Y_p) \: - \\ \hspace{2em}
%
             \mat{R}_3(\zeta_0)  \cdot
             \mat{R}_2(\theta_0)  \cdot
             \mat{R}_3(z)  \cdot
             \matd{R}_1(-\epsilon_0)  \cdot
             \mat{R}_3(\Delta\psi)  \cdot
             \mat{R}_1(\epsilon_0 + \Delta\epsilon) \cdot
             \mat{R}_3(-S) \cdot
             \mat{R}_2(-X_p) \cdot
             \mat{R}_3(Y_p) \: + \\ \hspace{2em}
%
             \mat{R}_3(\zeta_0)  \cdot
             \mat{R}_2(\theta_0)  \cdot
             \mat{R}_3(z)  \cdot
             \mat{R}_1(-\epsilon_0)  \cdot
             \matd{R}_3(\Delta\psi)  \cdot
             \mat{R}_1(\epsilon_0 + \Delta\epsilon) \cdot
             \mat{R}_3(-S) \cdot
             \mat{R}_2(-X_p) \cdot
             \mat{R}_3(Y_p) \: + \\ \hspace{2em}
%
             \mat{R}_3(\zeta_0)  \cdot
             \mat{R}_2(\theta_0)  \cdot
             \mat{R}_3(z)  \cdot
             \mat{R}_1(-\epsilon_0)  \cdot
             \mat{R}_3(\Delta\psi)  \cdot
             \matd{R}_1(\epsilon_0 + \Delta\epsilon) \cdot
             \mat{R}_3(-S) \cdot
             \mat{R}_2(-X_p) \cdot
             \mat{R}_3(Y_p) \: + \\ \hspace{2em}
%
             \mat{R}_3(\zeta_0)  \cdot
             \mat{R}_2(\theta_0)  \cdot
             \mat{R}_3(z)  \cdot
             \mat{R}_1(-\epsilon_0)  \cdot
             \mat{R}_3(\Delta\psi)  \cdot
             \mat{R}_1(\epsilon_0 + \Delta\epsilon) \cdot
             \matd{R}_3(-S) \cdot
             \mat{R}_2(-X_p) \cdot
             \mat{R}_3(Y_p) \: + \\ \hspace{2em}
%
             \mat{R}_3(\zeta_0)  \cdot
             \mat{R}_2(\theta_0)  \cdot
             \mat{R}_3(z)  \cdot
             \mat{R}_1(-\epsilon_0)  \cdot
             \mat{R}_3(\Delta\psi)  \cdot
             \mat{R}_1(\epsilon_0 + \Delta\epsilon) \cdot
             \mat{R}_3(-S) \cdot
             \matd{R}_2(-X_p) \cdot
             \mat{R}_3(Y_p) \: - \\ \hspace{2em}
%
             \mat{R}_3(\zeta_0)  \cdot
             \mat{R}_2(\theta_0)  \cdot
             \mat{R}_3(z)  \cdot
             \mat{R}_1(-\epsilon_0)  \cdot
             \mat{R}_3(\Delta\psi)  \cdot
             \mat{R}_1(\epsilon_0 + \Delta\epsilon) \cdot
             \mat{R}_3(-S) \cdot
             \mat{R}_2(-X_p) \cdot
             \matd{R}_3(Y_p)
  \end{array}
\eeq{e:e29}
%
  For computing the second derivative with relative accuracy $10^{-5}$ it
is sufficient to retain 8 terms:
\beq
   \begin{array}{l}
     \Frac{\partial^2}{\partial t^2} \mat{M}(t) = \\ \hspace{2em}
         \hphantom{2\,}
             \mat{R}_3(\zeta_0)  \cdot
             \mat{R}_2(\theta_0)  \cdot
             \mat{R}_3(z)  \cdot
             \mat{R}_1(-\epsilon_0)  \cdot
             \mat{R}_3(\Delta\psi)  \cdot
             \mat{R}_1(\epsilon_0 + \Delta\epsilon) \cdot
             \mat{R}_3(-S) \cdot
             \mat{R}_2(-X_p) \cdot
             \matdd{R}_3(Y_p) \: - \\ \hspace{2em}
%
         2\, \matd{R}_3(\zeta_0)  \cdot
             \mat{R}_2(\theta_0)  \cdot
             \mat{R}_3(z)  \cdot
             \mat{R}_1(-\epsilon_0)  \cdot
             \mat{R}_3(\Delta\psi)  \cdot
             \mat{R}_1(\epsilon_0 + \Delta\epsilon) \cdot
             \matd{R}_3(-S) \cdot
             \mat{R}_2(-X_p) \cdot
             \mat{R}_3(Y_p) \: + \\ \hspace{2em}
%
         2\, \mat{R}_3(\zeta_0)  \cdot
             \matd{R}_2(\theta_0)  \cdot
             \mat{R}_3(z)  \cdot
             \mat{R}_1(-\epsilon_0)  \cdot
             \mat{R}_3(\Delta\psi)  \cdot
             \mat{R}_1(\epsilon_0 + \Delta\epsilon) \cdot
             \matd{R}_3(-S) \cdot
             \mat{R}_2(-X_p) \cdot
             \mat{R}_3(Y_p) \: - \\ \hspace{2em}
%
         2\, \mat{R}_3(\zeta_0)  \cdot
             \mat{R}_2(\theta_0)  \cdot
             \matd{R}_3(z)  \cdot
             \mat{R}_1(-\epsilon_0)  \cdot
             \mat{R}_3(\Delta\psi)  \cdot
             \mat{R}_1(\epsilon_0 + \Delta\epsilon) \cdot
             \matd{R}_3(-S) \cdot
             \mat{R}_2(-X_p) \cdot
             \mat{R}_3(Y_p) \: - \\ \hspace{2em}
%
         2\, \mat{R}_3(\zeta_0)  \cdot
             \mat{R}_2(\theta_0)  \cdot
             \mat{R}_3(z)  \cdot
             \mat{R}_1(-\epsilon_0)  \cdot
             \matd{R}_3(\Delta\psi)  \cdot
             \mat{R}_1(\epsilon_0 + \Delta\epsilon) \cdot
             \matd{R}_3(-S) \cdot
             \mat{R}_2(-X_p) \cdot
             \mat{R}_3(Y_p) \: - \\ \hspace{2em}
%
         2\, \mat{R}_3(\zeta_0)  \cdot
             \mat{R}_2(\theta_0)  \cdot
             \mat{R}_3(z)  \cdot
             \mat{R}_1(-\epsilon_0)  \cdot
             \mat{R}_3(\Delta\psi)  \cdot
             \matd{R}_1(\epsilon_0 + \Delta\epsilon) \cdot
             \matd{R}_3(-S) \cdot
             \mat{R}_2(-X_p) \cdot
             \mat{R}_3(Y_p) \: - \\ \hspace{2em}
%
         2\, \mat{R}_3(\zeta_0)  \cdot
             \mat{R}_2(\theta_0)  \cdot
             \mat{R}_3(z)  \cdot
             \mat{R}_1(-\epsilon_0)  \cdot
             \mat{R}_3(\Delta\psi)  \cdot
             \mat{R}_1(\epsilon_0 + \Delta\epsilon) \cdot
             \matd{R}_3(-S) \cdot
             \matd{R}_2(-X_p) \cdot
             \mat{R}_3(Y_p) \: - \\ \hspace{2em}
%
         2\, \mat{R}_3(\zeta_0)  \cdot
             \mat{R}_2(\theta_0)  \cdot
             \mat{R}_3(z)  \cdot
             \mat{R}_1(-\epsilon_0)  \cdot
             \mat{R}_3(\Delta\psi)  \cdot
             \mat{R}_1(\epsilon_0 + \Delta\epsilon) \cdot
             \matd{R}_3(-S) \cdot
             \mat{R}_2(-X_p) \cdot
             \matd{R}_3(Y_p)
  \end{array}
\eeq{e:e30}
%
  Expression fir the first two derivatives of $\vec{q}_e$ is
\beq
    \vc{\dot{q}}(t) = & \left(
       \begin{array}{ll}
          m \displaystyle\sum_{k=1-m}^{n-1} f_{1k} \, B_k^{m-1}(t) \: + &
          \displaystyle\sum_{j}^{N} \omega_j
                \left( -P^c_{j} \sin \omega_j \, t \: + \:
                                        P^s_{j} \cos \omega_j \, t \right) \:
          \vspace{0.5ex} \\ & \hphantom{t \, \Omega_n} +
           \left( S^c_{j} \cos -\Omega_n \, t \: + \:
                        S^s_{j} \sin -\Omega_n \, t \right)
          \vspace{0.5ex} \\ &
           - t \: \Omega_n \left( -S^c_{j} \sin -\Omega_n \, t \: + \:
                        S^s_{j} \cos -\Omega_n \, t \right)
          \vspace{0.5ex} \vspace{2ex} \\
%
          m \displaystyle\sum_{k=1-m}^{n-1} f_{2k} \, B_k^{m-1}(t) \: + &
          \displaystyle\sum_{j}^{N} \omega_j
              \left( P^c_{j} \cos \omega_j \, t \: + \:
                             P^s_{j} \sin \omega_j \, t \right) \:
          \vspace{0.5ex} \\ & \hphantom{t \, \Omega_n} +
          \left( S^c_{j} \sin -\Omega_n \, t \: - \:
                     S^s_{j} \cos -\Omega_n \, t \right)
          \vspace{0.5ex} \\
          & - t \: \Omega_n \left( S^c_{j} \cos -\Omega_n \, t \: + \:
                     S^s_{j} \sin -\Omega_n \, t \right)
          \vspace{0.5ex} \vspace{2ex} \\
%
          m \displaystyle\sum_{k=1-m}^{n-1} f_{3k} \, B_k^{m-1}(t) \: + &
          \displaystyle\sum_{j}^{N} \omega_j
                       \left( - E^c_{j} \sin \omega_j \, t +
                                E^s_{j} \cos \omega_j \, t \right)
          \\
       \end{array}
       \right)
\eeq{e:e31}
%
\beq
    \vc{\ddot{q}}(t) = & \left(
       \begin{array}{ll}
          m\,(m-1) \displaystyle\sum_{k=1-m}^{n-1} f_{1k} \, B_k^{m-2}(t) \:\: - &
          \displaystyle\sum_{j}^{N} \omega^2_j
                \left( -P^c_{j} \cos \omega_j \, t \: + \:
                                        P^s_{j} \sin \omega_j \, t \right) \:
          \vspace{2ex} \\
%
          m\,(m-1) \displaystyle\sum_{k=1-m}^{n-1} f_{2k} \, B_k^{m-2}(t) \:\: - &
          \displaystyle\sum_{j}^{N} \omega^2_j
              \left( P^c_{j} \sin \omega_j \, t \: - \:
                     P^s_{j} \cos \omega_j \, t \right) \:
          \vspace{2ex} \\
%
          m\,(m-1) \displaystyle\sum_{k=1-m}^{n-1} f_{3k} \, B_k^{m-1}(t) \:\: - &
          \displaystyle\sum_{j}^{N} \omega^2_j
                 \left( - E^c_{j} \sin \omega_j \, t +
                          E^s_{j} \sin \omega_j \, t \right)
          \\
       \end{array}
       \right)
\eeq{e:e32}
%
   The parameterization of the Earth rotation according to the Newcomb-Andoyer
representation requires knowledge of empirical functions UT1(t), $X_p(t)$,
$Y_p(t)$. They are presented in the form of of time series with equal time
spacing. The empirical EOP function used for data reduction are produced from
analysis of observations with applying smoothing, filtering and re-sampling.
Coefficients of the interpolating spline of the 3rd degree are computed
using several points around the date of interest. A strong periodic signal
caused by zonal tides affects function UT1(t). In order to alleviate effect
of this signal in interpolation, the contribution of zonal tides to UT1
can be subtracted from the initial UT1(t) series at epochs of nodes, and
then computed to the epoch of observation and added back. The contribution
to UT1 caused by zonal tides can be obtained either from analysis of
observations or from a a theory. In both case it is presented in the form
of quasi-harmonic expansion:
%
\beq
    \Delta\mbox{UT1}(t) = \dss\sum_{i}^{n}
        Z_{ci} \cos ( \phi_i + \omega_i\, t + \dot{\omega}_i\, t^2/2 ) +
        Z_{si} \sin ( \phi_i + \omega_i\, t + \dot{\omega}_i\, t^2/2 )
\eeq{e:e33}

\subsection{Reduction for station secular motion}

  In the most general from, the reduction secular motion of an antenna can be
presented in the form
%
\beq
    \vc{r}_{\tiny\sc t}(t) =
                             \vc{r}_{0}(t) \: + \:
                             \dvc{r}(t)    \: + \:
                             \dss \sum_{k=1-m}^{n-1} F_{k} \, B_k^m(t)
\eeq{e:e34}
%
  where $B_k^m$ is the spline of the $m$ degree. The spline may have multiple
nodes and account for the discontinuity in positions and velocities.
The discontinuities may arise from either tectonic events: earthquakes or
volcanic activity or be a result of a human activity, such as rail repairs.
The linear part of the secular motion is caused by the plate tectonic and
the isostatic glacial adjustment. The separation of the motion into linear
part and expansion into the B-spline basis is entirely arbitrary.

\subsection{Reduction for the displacement caused by antenna slewing}

  An antenna has a fixed axis $A$ and a moving axis $B$ (figure~\ref{f:axes}).
In general, these axis do not intersect. Even if the antenna was designed
to have intersecting axes, during the manufacturing process the moving axis
may be shifted at 1--10~mm from the fixed axis. The antenna reference point
$f$ is the point on the fixed axis which is the closest to the moving axis.
The antenna position is referred to that point. The fringe phase is referred
to the phase center of the receiver that is located at the primary or
secondary focus of the antenna $p$. During antenna slewing, position of the
point $b$ on the moving axis that is located at the point the closest of the
fixed axis changes with respect to the reference point, unless the axes are
perfectly intersect. We can notice that the vector $\vc{bp}$ is always
coincides with the source vector $\vce{S}$ during observation, therefore,
the VLBI path delay for wave propagation from the point $p$ to the point
$b$ is constant and cannot be distinguished from the clock offset. Therefore,
we do not need to make reduction from the point $p$ to the point $b$. However,
we need to perform reduction from the point $b$ to the point $f$, i.e.,
to compute delay for wave propagation from the point $b$ to the point $f$.
We will achieve it by computing the vector $\vc{fb}$ in the terrestrial
coordinate system and adding it to the a~priori station position.

\begin{figure}
   \caption{Antenna axes \label{f:axes}}
   \par\vspace{1ex}\par
   \centerline{\mbox{ \epsfclipon
                      \epsfxsize=0.5\textwidth
                      \epsffile{antenna_axis.eps}
                    }
              }
\end{figure}

  Different types of antenna mountings can be classified according to
direction of the unit vector of the fix antenna \vc{A}%
%
\begin{itemize}
     \item {\sf Azimuthal mounting}:
         \beq
             \vce{a}_{\tiny\sc u} = \lp
                                       \begin{array}{c}
			                      1 \\ 0 \\ 0
                                       \end{array}
                                   \rp
         \eeq{e:e35}
%
     \item {\sf Horizontal XY-N mounting}:
         \beq
             \vce{a}_{\tiny\sc u} = \lp
                                       \begin{array}{c}
			                      0 \\ 0 \\ 1
                                       \end{array}
                                   \rp
         \eeq{e:e36}
%
     \item {\sf Horizontal XY-E mounting}:
         \beq
             \vce{a}_{\tiny\sc u} = \lp
                                       \begin{array}{c}
			                      0 \\ 1 \\ 0
                                       \end{array}
                                   \rp
         \eeq{e:e37}
%
     \item {\sf Equatorial mounting}:
         \beq
             \vce{a}_{\tiny\sc c} = \lp
                                       \begin{array}{c}
			                      1 \\ 0 \\ 0
                                       \end{array}
                                   \rp
         \eeq{e:e38}
%
     \item {\sf Special case of RICHMOND}
         \beq
             \vce{a}_{\tiny\sc u} =
               \lp
                 \begin{array}{l}
                     \cos 39^\circ 03' 36'' \cdot \cos -00^\circ 07' 12'' \\
                     \cos 39^\circ 03' 36'' \cdot \sin -00^\circ 07' 12'' \\
                     \sin 39^\circ 03' 36''
                 \end{array}
               \rp
         \eeq{e:e39}
\end{itemize}
%
   Subscript $_{\tiny\sc u}$ means the the vector is in the local topocentric
Up-East-North coordinate system. This can be converted from to the terrestrial
coordinate system through this rotation matrix transformation:
%
\beq
   \vc{r}_{\tiny\sc t} =
   \lp
      \begin{array}{lll}
             \hpm \cos \phi_{geod} \, \cos \lambda & -\sin \lambda &
                                   -\sin \phi_{geod} \, \cos \lambda        \\
             \hpm \cos \phi_{geod} \, \sin \lambda &  \hpm \cos \lambda &
                                   -\sin \phi_{geod} \, \sin \lambda        \\
             \hpm \sin \phi_{geod}                &  \hpm 0            &
                                   \hpm \cos \phi_{geod}
      \end{array}
   \rp \: \vc{r}_{\tiny\sc u}
\eeq{e:e40}

  The geodetic latitude $\phi_{geod}$ is computed according to
\cite{r:bowring85} equations:
\beq
  \phi_{geod} & = & \arctg \Frac{ r_3(1-f\earth) +
            (2f\earth -f\earth^2)\, R\earth \sin^3 \mu }
            { r_p(1-f\earth) ( r_p - (2f\earth -f^2\earth)\, \cos^3 \mu ) }
      \nonumber \\
          r_p & = & \sqrt { r^2_1 + r_2^2 }
      \nonumber \\
          \mu & = & \arctg \left( \Frac{r_3}{r_p} \left[ \left( 1-f \right) +
                    \Frac{ (2f\earth -f\earth^2)R\earth }{ |r| } \right] \right)
      \nonumber \\
      h_{ort} & = & r_p \cos \phi_{geod}  + r_3 \sin \phi_{geod} - R\earth
                    \sqrt { 1 - (2f\earth -f\earth^2)\sin^2 \phi_{geod} }
\eeq{e:e41}
%
   where $f\earth$ is the Earth's figure flattening and $e\earth$ is its
eccentricity.

  The vector $\vc{fb}$ is expressed as
%
\beq
    \vc{fb} \enskip = \enskip
    l \Frac{ \vce{a} \vpd [ \vce{S}_{app} \vpd \vce{a} ]   }
           { \left| \vce{a} \vpd [ \vce{S}_{app} \vpd \vce{a} ] \right| }
      \enskip = \enskip
           l \Frac{   \vce{S}_{app} - \vce{a} ( \vce{S}_{app} \spd \vce{a})   }
                  { | \vce{S}_{app} - \vce{a} ( \vce{S}_{app} \spd \vce{a}) | }
\eeq{e:e42}
%
  where $l$ is a parameter called ``antenna axis offset'', and
$\vce{S}_{app}$ is an apparent source vector. The maximum value of the
antenna axis offset may reach 14 meters, therefore, in order to reach
$10^{-13}$~s accuracy in delay prediction, we need to compute vector
$\vec{fb}$ with accuracy $10^{-5}$. At this level of accuracy we should
take into account annual aberration and refraction.

  The annual aberration shifts the position vector as
%
\beq
     \vce{s}_a = \Frac{1}{c}\vc{V}\earth -
                  \Frac{1}{c} ( \vce{s}\spd\vc{V}\earth ) \spd \vce{s}
\eeq{e:e43}
%
   The refractively angle $\rho$ can be computed using this expression
\cite{r:masterfit}:
%
\beq
     \rho = \Frac{\flo{3.13}{-4}}{\tg E}
\eeq{e:e44}
%
   where $E$ is the source elevation. Reduction for refractively is equivalent
to rotation the source vector at the angle $rho$ in the plane common to the
the direction to the zenith and the source vector. Therefore, the position
vector reduced for refraction is
$ \vce{s}_\rho = \cos \rho \,\, \vce{s} + \sin \rho \,\, \vce{s}_{\,\perp} $,
where \( \vce{s}_{\,\perp} \) --- the vector in the plane common for the zenith
vector an the vector which is perpendicular to the source vector:
$ \vce{s}_{\,\perp} = [\vce{s} \vpd \vce{r}_3 ] \vpd \vce{s}$, where
$ \vce{r} $ is the unit vector $\frac{\vc{r}}{|r|}$.

  Combining expressions \ref{e:e43}--\ref{e:e44}, we get the following
expression for the apparent source vector:
%
\beq
  \begin{array}{rcl}
    \vce{S}_{app} & = & \Frac{1}{c}\vc{V}\earth -
                  \Frac{1}{c} ( \vce{S}\spd\vc{V}\earth ) \spd \vce{S} \, +
    \vspace{1pt} \\
       & & \cos \lp \flo{3.13}{-4} \cdot \Frac{ \sqrt{ 1 - \vce{r}_3
                \spd \vce{S} } } { \vce{r}_3 \spd \vce{S} } \rp \vce{S} \, +
    \vspace{1pt} \\
              & & \sin \lp \flo{3.13}{-4} \cdot \Frac{ \sqrt { 1 - \vce{r}_3
                       \spd \vce{S} } } { \vce{r}_3 \spd \vce{S} } \rp
                       \lp \vce{r}_3 - ( \vce{r}_3 \spd \vce{S} ) \vce{S} \rp
   \end{array}
\eeq{e:45}
%
   Since the source vector in in the inertial coordinate system, other
vectors, $\vc{r}$ and $\vce{a}$ should be transformed into the inertial
coordinate system before computation of $\vce{S}_{app}$.

\subsection{Reduction for Earth solid tides}

\subsubsection{Algorithm for the computation of displacements due to solid
               Earth tide of the second Degree}

  \cite{r:mat95} proposed the following formalism for the representation of
a displacement field caused by the solid Earth tides of the second degree:
%
\beq
   \begin{array}{l}
      \vec{d}_{\sc{ren}} = \\ \hspace{2em}
          \Sum_{m=0}^{m=2} \Frac{\Phi^m_2 a_e^2}{g_e}
          \Biggl[ \Biggr.
                  \Bigl(h^{(0)} + h^{(2)} P^0_2 \Bigr) \, {\bf{R}}^m_2 \: + \:
                        h'                             \, {\bf{R}}^m_0 \: + \:

                  \Bigl(l^{(0)} + l^{(2)} P^0_2 \Bigr) \, {\bf{S}}^m_2 \: + \:
                        l^{(1)}P^0_1                   \, {\bf{T}}^m_2 \: + \:
                        l'                             \, {\bf{T}}^m_1
          \Biggl.\Biggr]
    \end{array}
\eeq{e:a01}
%
where $\Phi^m_2$ is the tidal potential of the second degree and {\bf R, S, T}
denote radial, transverse spheroidal and toroidal vector harmonic fields:
%
\beq
   {\bf{R}}^m_\ell =      \vec{r} \, Y^m_\ell             \qquad\quad
   {\bf{S}}^m_\ell = r \,         \nabla Y^m_\ell         \qquad\quad
   {\bf{T}}^m_\ell = i \, \vec{r} \times \nabla Y^m_\ell  \quad
\eeq{e:a02}
%
here $\vec{r}$ \enskip is a unit station coordinate vector,
$Y_{\ell}^m$ is a spherical harmonic function of degree $\ell$ and order $m$
normalized over the unit sphere, $a_e$ is the Earth's equatorial radius,
$g_e$ is the gravity acceleration at the equator, $P_\ell^m$ is
a Legendre function, and $h$ and $l$ are the generalized Love numbers:

\par\bigskip\par
\begin{flushleft}
\begin{tabular}{l @{\,---\,} l }
  $ h^{(0)}    $ & principal Love number; \\
  $ h^{(i)}    $ & out-of-phase radial Love number; \\
  $ h^{(2)}    $ & latitude Love number ; \\
  $ h^{\prime} $ & zero degree Love number; \\
  $ l^{(0)}    $ & principal Shida number; \\
  $ l^{(i)}    $ & out-of-phase Shida number; \\
  $ l^{(1)}    $ & second degree toroidal Love number; \\
  $ l^{(2)}    $ & latitude Shida number; \\
  $ l^{\prime} $ & first degree toroidal Love number; \\
\end{tabular}
\end{flushleft}
\par\bigskip\par

\newpage

   In order to transform equation \eref{e:a01} to the form suitable for
computations we do the following operations:
1)~substitute direct expressions for vector harmonic fields \eref{e:a02};
2)~add out-of-phase terms;
3)~expand the tidal potential in a Fourier time series;
4)~separate the terms which depend on station latitude and longitude from
the terms which depend on time.
After some algebra we get the following expression for a tidal displacement
vector $\vec{d}_{\sc{ren}}$ with radial, east and north components:
%

\begin{table}[h]
% \par\vspace{-11ex}\par
\beq
   \begin{array}{ll}
   \vec{d}_{\sc{ren}} = &
   \dss \sum_{m=0}^{m=2} \left(
   \begin{array}{l c r}
       \hphantom{-}
       \vec{X}^{rc}_1(m,\phi) \cdot \dss \sum_{k=1}^{n(m)} A_k     \,
                              \vec{L}^r_1(k) \, \cos{ \gamma_{km}} & - &
%
       \vec{X}^{rs}_1(m,\phi) \cdot \dss \sum_{k=1}^{n(m)} A_k     \,
                              \vec{L}^r_1(k) \, \sin{ \gamma_{km}} \vex \\
%
       \hphantom{-}
       \vec{X}^{rc}_2(m,\phi) \cdot \dss \sum_{k=1}^{n(m)} A_k     \,
                              \vec{L}^r_2(k) \, \sin{ \gamma_{km}} & + &
%
       \vec{X}^{rs}_2(m,\phi) \cdot \dss \sum_{k=1}^{n(m)} A_k
                              \vec{L}^r_2(k) \, \cos{ \gamma_{km}} \vex \\
%
       \hphantom{-}
       \vec{X}^{rc}_3(m,\phi) \cdot \dss \sum_{k=1}^{n(m)} A_k     \,
                              \vec{L}^r_3(k) \, \cos{ \gamma_{km}} & - &
%
       \vec{X}^{rs}_3(m,\phi) \cdot \dss \sum_{k=1}^{n(m)} A_k     \,
                              \vec{L}^r_3(k) \, \sin{ \gamma_{km}} \vex \\
%
   \end{array}
   \right) +  \veX \veX \\ &
%
   \dss \sum_{m=0}^{m=2}
   \left(
   \begin{array}{l c r}
      - \vec{X}^{ic}_1(m,\phi) \cdot \dss \sum_{k=1}^{n(m)} A_k     \,
                               \vec{L}^i_1(k) \, \sin{ \gamma_{km}} & - &
%
        \vec{X}^{is}_1(m,\phi) \cdot \dss \sum_{k=1}^{n(m)} A_k     \,
                               \vec{L}^i_1(k) \, \cos{ \gamma_{km}} \vex  \\
%
   \hpm \vec{X}^{ic}_2(m,\phi) \cdot \dss \sum_{k=1}^{n(m)} A_k     \,
                               \vec{L}^i_2(k) \, \cos{ \gamma_{km}} & - &
%
        \vec{X}^{is}_2(m,\phi) \cdot \dss \sum_{k=1}^{n(m)} A_k     \,
                               \vec{L}^i_2(k) \, \sin{ \gamma_{km}} \vex  \\
%
      - \vec{X}^{ic}_3(m,\phi) \cdot \dss \sum_{k=1}^{n(m)} A_k     \,
                              \vec{L}^i_3(k) \, \sin{ \gamma_{km}}  & - &
%
        \vec{X}^{is}_3(m,\phi) \cdot \dss \sum_{k=1}^{n(m)} A_k     \,
                               \vec{L}^i_3(k) \, \cos{ \gamma_{km}} \vex  \\
%
   \end{array}
   \right)
   \end{array}
\eeq{e:a04}
\end{table} % \ns

%
\par\noindent
  where vector $ \vec{X} $ depends only on station coordinates:

\par\medskip\par
\begin{flushleft}
     \begin{tabular}{r @{\,} c @{\,} l }
        $ \vec{X}^{rc}_j(m,\phi) $ & = & $ \vec{Z}^r_j(m,\phi) \cdot \cos
           m \lambda $ \vex\\
        $ \vec{X}^{rs}_j(m,\phi) $ & = & $ \vec{Z}^r_j(m,\phi) \cdot \sin
           m \lambda $ \vex\\
        $ \vec{X}^{ic}_j(m,\phi) $ & = & $ \vec{Z}^i_j(m,\phi) \cdot \cos
           m \lambda $ \vex\\
        $ \vec{X}^{is}_j(m,\phi) $ & = & $ \vec{Z}^i_j(m,\phi) \cdot \sin
           m \lambda $ \vex\\
     \end{tabular}
\end{flushleft}

\par\vspace{-3ex}\par
\beq
\eeq{e:a05}

%
\par\noindent
  here $\phi$ is geocentric latitude and $\lambda$ is positive towards
east longitude. Vector $\vec{Z}$ is
\goodbreak

\beq
   \begin{array}{ l @{\:} c @{\:} c @{\hspace{2.5em}} l @{\:} c @{\:} c }
      \vec{Z}^r_1 & = & \left(
                        \begin{array}{r}
                               \bar{P}^m_2          \Frac{1}{g_e}    \veX \\
                               P^0_2 \, \bar{P}^m_2 \Frac{1}{g_e}    \veX \\
                               0                                     \veX \\
                                              \Frac{1}{g_e}          \veX \\
                        \end{array}
                        \right) &
      \vec{Z}^i_1 & = & \left(
                               \begin{array}{r}
                                     \bar{P}^m_2    \Frac{1}{g_e} \veX \\
                               \end{array}
                        \right)
      \vex \vex \\
%
      \vec{Z}^r_2 & = & \left(
                        \begin{array}{r}
                          - \Frac{m}{\cos{\phi}} \, \bar{P}^m_2
                                                          \Frac{1}{g_e} \veX \\
                          - \Frac{m}{\cos{\phi}} \, P^0_2 \, \bar{P}^m_2
                                                          \Frac{1}{g_e} \veX \\
                            P^0_1 \Der{\bar{P}^m_2}{\phi} \Frac{1}{g_e} \veX \\
                          \Der{\bar{P}^m_1}{\phi}         \Frac{1}{g_e} \veX \\
                        \end{array}
                        \right) &
%
      \vec{Z}^i_2 & = & \left(
                               \begin{array}{r}
                                      - \Frac{m}{\cos{\phi}} \, \bar{P}^m_2
                                                          \Frac{1}{g_e} \veX \\
                                \end{array}
                        \right)
%
      \vex \vex \\
%
      \vec{Z}^r_3 & = & \left(
                        \begin{array}{r}
                          \Der{\bar{P}^m_2}{\phi}         \Frac{1}{g_e} \veX \\
                          P^0_2 \, \Der{\bar{P}^m_2}{\phi}\Frac{1}{g_e} \veX \\
                          - \Frac{m}{\cos{\phi}} \, P^0_1 \,\bar{P}^m_2
                                                          \Frac{1}{g_e} \veX \\
                          - \Frac{m}{\cos{\phi}} \, \bar{P}^m_1
                                                          \Frac{1}{g_e} \veX \\
                        \end{array}
                        \right) &
%
      \vec{Z}^i_3 & = & \left(
                                \begin{array}{r}
                                       \Der{\bar{P}^m_2}{\phi}
                                                          \Frac{1}{g_e} \veX \\
                                \end{array}
                        \right)
  \end{array}
\eeq{e:a06}

$P^0_m$ is a Legendre function normalized to have maximal
value 1:
\beq
  \small
  \begin{array}{ l @{\hspace{1.5em}} l @{\hspace{1.5em}} l }
         P^0_1 = \sin{ \phi } &
         P^1_1 = \cos{ \phi } &
         P^2_1 = 0
%
         \vex \\
%
         P^0_2 = \left( \Frac{3}{2} \sin^2{\phi} - \Frac{1}{2} \right) &
         P^1_2 = 2 \sin{\phi} \, \cos{\phi}                            &
         P^2_2 = \cos^2{\phi}
  \end{array}
\eeq{e:a07}
%
  and $\bar{P}^m_{l}$ are Legendre functions normalized over the surface of the
unit sphere:
%
\beq
  \small
  \begin{array}{ l @{\hspace{2.5em}} l @{\hspace{2.5em}}  l }
     \bar{P}^0_1 = P^0_1                              &
     \bar{P}^1_1 = P^1_1                              &
     \bar{P}^2_1 = P^2_1                              \vex \\
%
     \bar{P}^0_2 = \sqrt{\Frac{5} {4 \pi} } \, P^0_2  &
     \bar{P}^1_2 = \sqrt{\Frac{15}{32 \pi}} \, P^1_2  &
     \bar{P}^2_2 = \sqrt{\Frac{15}{32 \pi}} \, P^2_2
  \end{array}
\eeq{e:a08}
%
  $g_e$ --- the Earth's equatorial gravity acceleration.

%\par\smallskip\par

  The summing in \eref{e:a04} is done over the constituents of the spectral
expansion of the tide-generating potential which is assumed to be in the form
%
\beq
   \Phi^m_2(t,r) = \dss \sum_{m=0}^{m=2}
                   \biggl( \Frac{r}{a_e} \biggr)^2 \bar{P}^m_2(\phi)
                   \cdot \dss \sum_{k=1}^{n(m)} A_k \cdot \cos \gamma_{km}
   \qquad
\eeq{e:a09}
%
 where $r$ is the distance from the geocenter, $a_e$ is the semi-major
axis of the Earth, $A_k$ is the normalized amplitude of the $k$-th tidal
wave and $\gamma_{km}$ is its argument:
%
\beq
   \gamma_{km} = \psi_k + \theta_k + \omega_k \, t_{\sc{tdb}} +
          m \, \Frac{2 \pi({\sc{ut1}}(t) - t_{\sc{tdb}})}{86400}
   \qquad
\eeq{e:a10}
%
  $\psi_k$ is the phase of the k-th wave, $\theta_k$ and $\omega_k$ are the
phase and frequency of the harmonic argument of that wave. $t_{\sc{tdb}}$ is
the time elapsed since the fundamental epoch J2000.0 ($12^h$ 1 January, 2000)
at the TDB scale. The difference $ {\sc ut1} - t_{\sc{tdb}}$ in \eref{e:a10}
takes into account variations in the Earth's rotation which were omitted in
producing the tidal potential series. The variable m=0,1,2 in
\ref{e:a04}--\ref{e:a06} denotes the order of a tidal wave, subscript index
1,2,3 denotes component of the displacement vector: radial, east, north, and
summation is carried out over spectral harmonics of the tidal expansion.
The number of constituents in sum \eref{e:a09}, $ n(m) $, is determined by
a truncation level.

  The vector of generalized Love numbers is presented in the form
\beq
  \begin{array}{lcl @{\hspace{2em}} lcl }
     \vec{L}^r_1 & = & \bigl( \: h^{(0)}, \: h^{(2)}, \: 0,         \:
                 h^{\prime}   \: \bigr)^{\bf \top}                  &
     \vec{L}^i_1 & = & h^{(i)} \vex \\
%
     \vec{L}^r_2 & = & \bigl( \: l^{(0)}, \: l^{(2)}, \: l^{(1)},   \:
                 l^{\prime}   \: \bigr)^{\bf \top}                  &
     \vec{L}^i_2 & = & l^{(i)} \vex \\
%
     \vec{L}^r_3 & = & \bigl( \: l^{(0)}, \: l^{(2)}, \: l^{(1)},   \:
                 l^{\prime}   \bigr)^{\bf \top}                     &
     \vec{L}^i_3 & = & l^{(i)} \vex \\
  \end{array}
  \quad
\eeq{e:a11}

\bigskip

  All generalized Love numbers are considered to be complex and
frequency-dependent. The generalized Love numbers are computed according to
the analytical expressions presented in \cite{r:mat01} with corrections for
some specific tidal waves taken from the tables.

  The advantages of this scheme are that the sums like \quad $ \sum A_k \,
\vec{L}^r_1(k) \, \sin{ \gamma_{km}} $ \quad depend only on time and
do not depend on station coordinates, and therefore, may be used for the
calculation of displacements of many stations at the same epoch. The vectors
$\vec{X}$ do not depend on time and are computed only once.

  The HW95 expansion contains sine and cosine coefficients $C_0$ and $S_0$.
Having these coefficients, we can compute phases and amplitudes for
\eref{e:a04} as \enskip $ \psi_k = -\arctan \Frac{S_{0k}}{C_{0k}} $ \enskip
and \enskip $ A^m_k = \rho(m)\, \sqrt{ C_{0k}^2 + S_{0k}^2 } $\enskip , where
$\rho(m)$ is a re-normalization factor. It is
\enskip $ \sqrt{4 \pi} $ \enskip for tides of the 0-th
order and \enskip $ \sqrt{8 \pi} $ \enskip for other tides.

  Frequencies and phases of tidal constituents are easily computed via
coefficients at fundamental arguments.
%
\beq
  \begin{array}{r@{\:}c@{\:}l}
     \theta_i   &=& \hphantom{6} \dss \sum_{j=1}^{j=11} k_{ij} \, F_{jo} +
                    \theta_{ai}                                \vhex \veX  \\
     \omega_i   &=& \hphantom{6} \dss \sum_{j=1}^{j=11} k_{ij} \, F_{j1} +
                    \omega_{ai}                                \vhex \veX  \\
  \end{array}
\eeq{e:a14}
%
  where $ F_{jq} $ are fundamental coefficients from the theory of planetary
motion \cite{r:simon}. We neglected terms of the 2-nd degree and higher.

\subsubsection{Algorithm for the computation of displacements due to solid
               Earth tides of the 3-rd degree}

   For computation of displacements due to solid Earth tides of the 3-rd degree
with a precision of 0.1mm we can neglect frequency dependence of Love numbers
and an admixture of terms in tide-generating potential other than 3-rd degree
as well as out-of-phase Love numbers. Therefore, vector of displacements
can be written in the form

\beq
   \small
   \vec{d}_{\sc{ren}} = \dss \sum_{m=0}^{m=3} \left(
   \begin{array}{l}
       X^{3c}_1(m,\phi) \, h_3 \cdot  \Sum_{k=1}^{n(m)} A_k  \,
                               \cos{   \gamma_{km}}          \quad - \quad
       X^{3s}_1(m,\phi) \, h_3 \cdot   \Sum_{k=1}^{n(m)} A_k \,
                               \sin{   \gamma_{km}}                    \vex \\
%
       X^{3s}_2(m,\phi) \, l_3 \cdot  \Sum_{k=1}^{n(m)} A_k  \,
                               \sin{   \gamma_{km}}          \quad + \quad
       X^{3c}_2(m,\phi) \, l_3 \cdot   \Sum_{k=1}^{n(m)} A_k \,
                               \cos{   \gamma_{km}}                    \vex \\
%
       X^{3c}_3(m,\phi) \, l_3 \cdot  \Sum_{k=1}^{n(m)} A_k  \,
                               \cos{   \gamma_{km}}          \quad - \quad
       X^{3s}_3(m,\phi) \, l_3 \cdot   \Sum_{k=1}^{n(m)} A_k \,
                               \sin{   \gamma_{km}}
   \end{array}
   \right)
\eeq{e:a15}

\par\noindent
  where $X$ depends only on station coordinates in the following way:

\par\bigskip\par
\begin{flushleft}
     \begin{tabular}{r @{\,} c @{\,} l }
        $ X^{3c}_j(m,\phi) $ & = & $ Z^3_j(m,\phi) \cdot \cos
           m \lambda $ \vex \\
        $ X^{3s}_j(m,\phi) $ & = & $ Z^3_j(m,\phi) \cdot \sin
           m \lambda $ \vex\\
     \end{tabular}
\end{flushleft}
\label{e:a16}
%
\par\medskip\noindent
%
 here $Z^3_j$ is
%
\beq
   \begin{array}{ l }
      Z^3_1(m,\phi) = \bar{P}^m_3                       \Frac{1}{g_e} \veX  \\
      Z^3_2(m,\phi) = -\Frac{m}{\cos \phi} \bar{P}^m_3  \Frac{1}{g_e} \veX  \\
      Z^3_3(m,\phi) = \Der{\bar{P}^m_3}{\phi}           \Frac{1}{g_e}
   \end{array} \quad
\eeq{e:a17}

  Legendre functions of third order are
\beq
   \begin{array}{ l }
   P^0_3 = \lp \frac{5}{2} \sin^3 {\phi}- \frac{3}{2} \sin{\phi} \rp
           \vex \\
   P^1_3 = \lp \frac{5}{2} \sin^2 {\phi}- \frac{1}{2} \sin{\phi} \rp \cos{\phi}
           \vex \\
   P^2_3 = \sin{\phi} \cos^2{\phi}
           \vex \\
   P^3_3 = \cos^3{\phi}
   \end{array}
\eeq{e:a18}
%
\beq
   \begin{array}{ l @{\qquad} l }
          \bar{P}^0_3 = \sqrt{\Frac{7}{4    \, \pi}} P^0_3 &
          \bar{P}^1_3 = \sqrt{\Frac{21}{16  \, \pi}} P^1_3   \vex \\
          \bar{P}^2_3 = \sqrt{\Frac{105}{32 \, \pi}} P^2_3 &
          \bar{P}^3_3 = \sqrt{\Frac{35}{64  \, \pi}} P^3_3
   \end{array}
\eeq{e:a19}

\par\smallskip\par

  Analogously to the tides of the second degree, the amplitudes of the
tide-generating potential produced from the HW95 expansion should be
multiplied by the same re-normalization factors.

\subsection{Computing displacements caused by pole tide}

  Earth's rotation causes the deformation of the Earth's figure. The variations
in the Earth's rotation causes time variable deformations.
The centrifugal potential at the point in the Earth with coordinate vector
$\vc{r}$ is
%
\beq
       V_c = \Frac{1}{2} ( \vc{r} \times \vec\Omega )
\eeq{e:p1}
%
   where $\vec{\Omega}$ is the vector of the Earth's angular velocity. This
potential has a permanent component and the variable one. The permanent
component causes a permanent displacement which cannot be observed. According
to an agreement, the variable part of the centrifugal potential is determined
as $ V_{vc} = \Frac{1}{2} ( \vc{r} \times \vec\Omega_{vc} ) $ where
\beq
   \vec\Omega_{vc} = \Omega_n \lp
      \begin{array}{c}
         E_1 - E_{10} - E_{11} \, t \\
         E_2 - E_{20} - E_{21} \, t \\
         1
      \end{array}
    \rp
\eeq{e:p2}
%
  where $\Omega_n$ is the nominal Earth's angular velocity and $E_1, E_2,
E_{11}, E_{21}$ are reference position of the vector of the angular velocity
and its reference rate of change. Variations of third component of the vector
of the Earth's angular is two order of magnitude less than the variations of
the first and second component, and can be neglected. Since the dependence
of the centrifugal potential with radius is the same as for the
tide-generating potential, the theory of the solid Earth tides can be
applied for computing displacements caused by variations of the vector
of the Earth's angular velocity. Using expression \ref{e:a01} for tidal
displacements, after simple algebra we get the following expression for
displacement caused by pole tide:
%
\beq
   \vec{d}_{\sc{ren}} = \lp
   \begin{array}{l}
      h_2   (\omega_{SA}) \: \Frac{\Omega_n^2 \; r^2}{g_{loc}} \,
            \hphantom{\sqrt{1 - \underline{r}_3^2}}
            \Bigl(
                  -\underline{r}_1 \, ( E_2 + E_{10} + E_{21} \, t ) \; + \;
                   \underline{r}_2 \, ( E_1 - E_{10} - E_{11} \, t )
            \Bigr)  \vex \\
      \ell_2(\omega_{SA}) \: \Frac{\Omega_n^2 \; r^2}{g_{loc}} \,
            \Frac{\underline{r}_3}{\sqrt{1 - \underline{r}_3^2}} \,
            \Bigl( \hpm
                   \underline{r}_1 \, ( E_1 - E_{10} - E_{11} \, t ) \; + \;
                   \underline{r}_2 \, ( E_2 - E_{20} - E_{21} \, t )
            \Bigr) \vex \\
      \ell_2(\omega_{SA}) \: \Frac{\Omega_n^2 \; r^2}{g_{loc}} \,
            \Frac{1 - 2\, \underline{r}_3^2}{\sqrt{1 - \underline{r}_3^2}} \,
            \Bigl(
                  -\underline{r}_1 \, ( E_2 + E_{10} + E_{21} \, t ) \; + \;
                   \underline{r}_2 \, ( E_1 - E_{10} - E_{11} \, t )
            \Bigr)  \vex
   \end{array}
   \rp
\eeq{e:p3}
%
  Love numbers are taken for the annual frequency
$ \omega_{SA} = \flo{1.991}{-7}$ rad s${}^{-1}$. It should be noted, there
is no unanimous agreement which parameters $E_1, E_2, E_{11}, E_{21}$ to use
in the expression \ref{e:p3}. These parameters can be determined from fitting
empirical series of $\vc{q}(t)$. But since the polar motion is a stochastic
process, regression parameters depend on the time period of estimation.
Change in $E_1, E_2, E_{11}, E_{21}$ will result in change of estimates of
site position and velocity.

\subsection{Computing displacements caused by mass loading}

  Displacement caused by various mass loading, ocean, atmospheric pressure,
hydrology, are computed by evaluating the convolution integral over the
land and over the ocean:
%
\beq
        u(\vec{r},t) = u_{\sc l}(\vec{r},t) + \Delta \bar{P}_o(t) \, u_o
\eeq{e:b1}
%
where $ \Delta \bar{P}_o(t) $ is the uniform sea-floor pressure and
$u_{\sc l}(\vec{r},t)$, $u_o(\vec{r}\,)$ are
%
\beq
  \begin{array}{rcl}
      u_{\sc l}(\vec{r},t) &=& \dss\sum_{i=1}^{n} \sum_{j=1}^{m} \,
                           \Delta P(\vec{r}{\,'\!}_{ij},t) \,
                           q(\vec{r},\vec{r}{\,'\!}_{ij}) \, \cos \phi_i \,
                           \dint\limits_{\hbb\! cell_{ij}}
                           G(\psi(\vec{r},\vec{r}{\,'}_{ij})) \, ds
                           \vspace{0.5ex} \\
      u_o(\vec{r}\,) &=& \dss\sum_{i=1}^{n} \sum_{j=1}^{m} \,
                           q(\vec{r},\vec{r}{\,'\!}_{ij}) \, \cos \phi_i \,
                           \dint\limits_{\hbb\! cell_{ij}}
                           G(\psi(\vec{r},\vec{r}{\,'\!}_{ij})) \, ds
   \end{array}
\eeq{e:b2}
%
 and index $i$ runs over latitude and index $j$ runs over longitude. Here
the integration over the sphere is replaced with a sum of integrals over small
cells. $q=1$ for the vertical component.

  Green's functions have a singularity in 0, so care must be taken in using
numerical schemes for computing the convolution integral. Although the Green's
function cannot be represented analytically over the whole range of its
argument, we can always find a good approximation over a small range.
We approximate the function $ G(\psi) \cdot \psi $ by a polynomial of the
third degree
\mbox{$ \alpha + \beta \, \psi + \gamma \, \psi^2 + \delta \, \psi^3 $}.
In order to compute the integral \ref{e:b2} over the cell, we introduce
a two-dimensional Cartesian coordinate system with the origin in the center
of the cell and the axis $x$ towards east, the axis $y$ towards north.
We neglect the Earth's curvature and consider the cell as a rectangle with
borders [-a, a], [-b, b] on the $x$ and $y$ axes respectively. Then the
integral of the Green's function over the cell with respect to a site with
coordinates ($x_s$, $y_s$) is evaluated analytically:

\beq
   \begin{array}{ll}
         \dss \dint\limits_{\hbb\!\! cell} G(\psi(x_s,y_s)) \, ds = &
          \dss \int\limits_{\hbb -b}^{b} \int\limits_{\hbb -a}^{a} \:
                      \Biggl(
                             \Frac{\alpha}{\sqrt{x^2 + y^2}}  \: + \:
                                   \beta                      \: + \:
                                   \gamma \, \sqrt{x^2 + y^2} \: + \:
                                   \delta (x^2 + y^2)
                      \Biggr) \, dx \, dy =            \vex \vex \vex \vex \\ &
          \biggl( \alpha \, y_2 + \Frac{\gamma}{6} \, y^3_2 \biggr)
                  \ln \Frac{x_2 + z_{22}}{x_1 + z_{12}}       \: - \:
          \biggl( \alpha \, y_1 + \Frac{\gamma}{6} \, y^3_1 \biggr)
                  \ln \Frac{x_2 + z_{21}}{x_1 + z_{11}}       \: + \: \vex \\ &
          \biggl( \alpha \, x_2 + \Frac{\gamma}{6} \, x^3_2 \biggr)
                  \ln \Frac{y_2 + z_{22}}{y_1 + z_{21}}       \: - \:
          \biggl( \alpha \, x_1 + \Frac{\gamma}{6} \, x^3_1 \biggr)
                  \ln \Frac{y_2 + z_{12}}{y_1 + z_{11}}       \:\: +  \vex \\ &
          (y_2 - y_1) \, (x_2 - x_1)
                \Biggr[ \beta + \Frac{\delta}{3}
                        \biggl(
                                 z^2_{11} + z^2_{22} + x_1 \, x_2 + y_1 \, y_2
                        \biggr)
                \Biggr]                                       \:\: +  \vex \\ &
          \Frac{\gamma}{3}
                \Biggr[
                    x_2 \, \biggl( y_2 \, z_{22} - y_1 \, z_{21} \biggr) \: - \:
                    x_1 \, \biggl( y_2 \, z_{12} - y_1 \, z_{11} \biggr)
                \Biggr]
%
      \vspace{3ex} \\ & \hspace{2em}
%
        \begin{array}{l @{\hspace{4em}} l}
           x_1 =  -a - x_s   &  x_2 = \hpm a - x_s  \vex \\
           y_1 =  -b - y_s   &  y_2 = \hpm b - y_s  \vex \\
           z_{11} = \sqrt{x_1^2 + y_1^2}  &  z_{12} = \sqrt{x_1^2 + y_2^2}
                                                    \vex \\
           z_{21} = \sqrt{x_2^2 + y_1^2}  &  z_{22} = \sqrt{x_2^2 + y_2^2}
        \end{array}
   \end{array}
\eeq{e:b3}
\par\medskip\par

  Coordinates $x_s, y_s$ are  computed as
%
\beq
        x_s = \vec{E}(\vec{r}{\,'\!}_{ij}) \cdot
              \vec{T}(\vec{r}{\,'\!}_{ij},\vec{r}\,) \qquad\qquad
        y_s = \vec{N}(\vec{r}{\,'\!}_{ij}) \cdot
              \vec{T}(\vec{r}{\,'\!}_{ij},\vec{r}\,)
\eeq{e:b4}
%
where $\vec{T}(\vec{r}{\,'\!}_{ij},\vec{r}\,)$ is
%
\beq
     \vec{T}(\vec{r}{\,'\!}_{ij},\vec{r}\,) = \Frac
       {   \vec{r}{\,'\!}_{ij} \times [ \vec{r} \times \vec{r}{\,'\!}_{ij} ]   }
       { | \vec{r}{\,'\!}_{ij} \times [ \vec{r} \times \vec{r}{\,'\!}_{ij} ] | }
\eeq{e:b5}
%
and $\vec{E}(\vec{r}{\,'\!}_{ij}), \vec{N}(\vec{r}{\,'\!}_{ij})$ are unit
vectors in east and north direction with respect to the center of the cell:
%
\beq
     \vec{E}(\vec{r}{\,'\!}_{ij}) = \left(
        \begin{array}{r}
           \sin \lambda'  \\
           \cos \lambda'  \\
	   0              \\
        \end{array}
        \right)
        \qquad \qquad
%
     \vec{N}(\vec{r}{\,'\!}_{ij}) = \left(
        \begin{array}{r}
          \hpm \sin \phi' \cos \lambda'  \\
             - \sin \phi' \sin \lambda'  \\
	  \hpm \cos \phi'
        \end{array}
        \right)
\eeq{e:b6}

  It was found that when the coefficients $ \alpha(\psi) $, $ \beta(\psi) $,
$ \gamma(\psi) $ and $ \delta(\psi) $ are computed with the step 0.002 rad
over the range [0, 0.16] rad, and with the step 0.02 rad over the range
[0.16, $\pi$], the error of the approximation of the integral~\ref{e:b3}
for a cell of size 0.044 rad ($\Deg{2}\!.5$) does not exceed 1\%.
At large angular distances we can consider the Green's function to be constant
over the cell. For an angular distance more than 0.16 rad, taking the Green's
function out of the integral \ref{e:b2} and replacing it with the value
at the angular distance between the site and the center of the cell causes
an error of less than 1\%.

  Two land-sea masks are used for practical computation: coarse with the
resolution of the surface pressure grid, and fine. If the cell of the coarse
land-sea mask is completely land or completely sea, this cell is used for
computing the integral \ref{e:b3}. Otherwise, the coarse resolution cell is
subdivided in smaller cells of the fine resolution grid, and the integral over
each fine resolution cell is computed independently. The surface pressure is
considered as defined at the corners of the coarse resolution cell.
The pressure at the center of the cell is obtained by bi-linear interpolation.
When $u_{\sc l}(\vec{r}\,)$ is computed, the cells which are over ocean are
bypassed. Alternatively, the cells which are over land are bypassed when
$u_o(\vec{r}\,)$ is computed.

  The computation of horizontal vectors is done separately for north and east
components. The north and east components of the vector
$\vec{q}(\vec{r},\vec{r}\,')$ are
%
\beq
    \vec{q}_{\sc n}(\vec{r},\vec{r}\,') = - \vec{T}(\vec{r},\vec{r}\,')
             \cdot \vec{N}(\vec{r}\,) \qquad\qquad
    \vec{q}_{\sc e}(\vec{r},\vec{r}\,') = - \vec{T}(\vec{r},\vec{r}\,')
             \cdot \vec{E}(\vec{r}\,)
\eeq{e:a7}
%
  where $\vec{T}(\vec{r},\vec{r}\,')$ is defined in a way similar
to \ref{e:b5}, but with the reverse order of arguments,
$\vec{E}(\vec{r}\,), \vec{N}(\vec{r}\,)$ are defined according to \ref{e:b6},
but are the unit north and east vectors for the site under consideration.

  Displacement caused by ocean tidal mass loading, non-tidal ocean mass
loading, atmosphere pressure loading and hydrology loading are computed the
same ways. Only the the pressure fields $ \Delta P(\vec{r}{\,'\!}_{ij},t) $,
$ \Delta \bar{P}_oP(\vec{r}{\,'\!}_{ij},t) $ differ. These are empirical
functions which are derived from analysis of observations.

\subsection{Tropospheric path delay}

   Propagation medium causes an additional delay $\tau_m$, which can be
written in the form of the integral along the path $l$, which in general
is bended:
%
\beq
     \tau_m = \int \bigl( n(l) - 1 \bigr) dl
\eeq{p8:e1}

   Traditionally path delay $\tau_m$ in the ionosphere, the neutral equilibrium
atmosphere, and in the non-equilibrium constituent of the atmosphere is
considered separately. The ionosphere is a dispersive medium, so there exists
a linear combination of observables at two or more frequencies which reduces
the ionospheric contribution to that combination to zero. The propagation of
the signal in the neutral atmosphere depends on the dependence of
the refractivity with height and possibly with spatial coordinates.
This dependence can be computed on the basis of the gas law for the
equilibrium component, and the integral \ref{p8:e1} can be computed
analytically. It turns out, in that case the integral \ref{p8:e1} depends only
on the surface pressure and the local gravity acceleration. The integral
\ref{p8:e1} should be computed numerically using empirical data about the
global partial refractivity distribution due to the non-equilibrium constituent.
Being viewed in the local topocentric coordinate system, the tropospheric
path delay depends on elevation and azimuth. This can be parameterized
in the form
%
\beq
   \tau_{tr} = \rho_{zd}R_{zd}(E_v,A) + \rho_{zw}R_{zw}(E_v,A)
\eeq{p8:e2}
%
   where $A$ is the azimuth and $E_v$ is the source elevation {\it ignoring}
bending in the atmosphere, but taking into account annual aberration,
$E_v = \mbox{arcsin} ( \vec{r_3}_{\tiny\sc c} \vc{S}_a)$. The tropospheric
path delay
is split into two parts, the equilibrium (dry) and the non-equilibrium part
\cite{r:davis85}. Both the equilibrium and the non-equilibrium parts are
presented as a product of the part which depends on ground meteorological
parameters, and the dimensionless part called mapping function, which
depends only on elevation and azimuth and normalized to unity in the
zenith direction. Thus, the first part of the product has a meaning of
zenith path delay.

  In the past various expressions were proposed for the zenith path delay
and the mapping function. Currently, the expression of Saastamoinen for the
zenith path delay and the Niell mapping function are used. There is no
evidence that alternative expressions which use ground meteorological
information produce better results.

  Saastamoinen expression \cite{r:saa72a,r:saa72b} for the non-equilibrium 
(dry) zenith path delay:
\beq
       \begin{array}{rcl}
              \rho_{zd} & = & \Frac{K_d \,R}{M_d\, c} \cdot
                        \Frac{P}{ g_{loc} - \der{g}{h}(0.9h_{ort} + 7300) }
        \vex \\
              \rho_{zw} & = & \Frac{K_d \,R}{M_d\, c} \cdot
                    \Frac{ 1255 \bigl( ( T\deg\,C + 273.15) + 0.05 \bigr)
                         E_w } { g_{loc} - \der{g}{h}(0.9h_{ort} + 7300) }
       \end{array}
\eeq{p8:e3}
%
   where
    \begin{itemize}
       \item [] \( K_d =\flo{7.7604}{-4} \) --- dry air refractivity;
       \item [] \( R = 8.314742 \) (J $\cdot$ K${}^{-1} \cdot$ mole${}^{-1}$)  ---
                                        the universal gas constant;
       \item [] \( M_d =28.9644 \) --- the mole mass of dry air;
       \item [] \( c \)   --- velocity of light;
       \item [] \( P \)   --- the surface total pressure (Pascal);
       \item [] \( E \)   --- the surface partial pressure of water
                              vapor (Pa);
       \item [] \( T\deg\,C \) --- the surface temperature (in Celsius)
       \item [] \( g_{loc} \)  --- the local gravity acceleration;
       \item []  \( h_{ort} \) --- ortometric height of the antenna
                                   reference point (ref \ref{e:e41}).
    \end{itemize}

   The following expression for the local gravity acceleration can be used:
%
\beq
      g_{loc}= g_e \Frac{ 1 + \der{g}{\phi} \sin^2 \phi_{ast} }
                        { \sqrt { 1 - (2f\earth -f\earth^2) \, \sin^2
                          \phi_{geod} } } +
                        \der{g}{h} h_{ort}
\eeq{p5:e3}
%
  where \( g_e \)   --- the equatorial gravity acceleration.

  The absolute humidity $E_w$ is not measured directly but deduced from
observations of the of the relative humidity, i.e. the ration of the partial
water vapor pressure to the pressure of the saturated water vapor at a given
temperature. The following expression of \cite{r:goff} for the pressure of the
saturated water vapor $E_s$ accepted by the International Meteorological
Association can be used which is accurate to a 0.02\% level:
%
\beq
   \begin{array}{rcl}
   \lg E_s & = & 10.795\,74 \lp 1 - \Frac{T\deg\,K_i}{T\deg\,K}  \rp -   \\&&
                 5.028\,00 \lg \lp  \Frac{T\deg\,K}{T\deg\,K_i}  \rp +   \\&&
                \flo{1.504\,75}{-4} \cdot \lp 1 - 10^{\tss -8.2969
                \lp \frac{T\deg\,K}{T\deg\,K_i} - 1 \rp} \rp -           \\&&
                \flo{4.287\,3}{-4} \cdot \lp 1 - 10^{\tss -4.769\,55 \lp
                \frac{T\deg\,K}{T\deg\,K_i} - 1 \rp} \rp                 \\&&
               + 2.786\,14
   \end{array}
\eeq{p8:e5}
%
   here $T\deg\,K_i$ is the triple point of water
$(T\deg\,K_i = 273\deg.16 K$).

  In the case, if no pressure measurement were done at the antenna, so-called
standard atmosphere of the International Meteorological Association can
be used. The dependence of the atmospheric pressure on height for the
standard atmosphere (table 3.9-2 in \cite{r:hrgian} can be approximated
with the accuracy better 7~Pa at the height range [-700, 5500] meters using
the following regression:
%
\beq
  \begin{array}{rcl}
      P & = & 101324.2 \cdot \exp \bigl( -\flo{1.1859}{-4} H_g -
              \flo{1.1343}{-9} H_g^2 -  \\
        &   & \flo{2.5644}{-14} H_g^3 \bigr)
  \end{array}
\eeq{p8:e7}
%
 where $H_g$ is the geopotential height, which on the Earth surface can
be computed as
%
\beq
      H_g = \Frac{ 9.806\,65 }{ g_{loc} } h_{ort}
\eeq{p8:e8}

  In general the barometer is located at a different height than the antenna
reference point. The air pressure at the antenna reference point differs
from the air pressure of the barometer at
$ \Delta P = -101324.2 \cdot \flo{1.1859}{-4} \Delta h$.

  The global Niell mapping functions \cite{r:nmf} $R_d(E)$ and
$R_e(E)$ are used:
%
\beq
  R(E) =
            \Frac{ 1 + \Frac{ a }{ 1 + \Frac{ b }{ 1 + c } } }
                 { \sin E + \Frac{ a }{ \sin E + \Frac{b}{ \sin E + c}}}+
        \lp \Frac{1}{\sin E} -
            \Frac{ 1 + \Frac{ a_t }{ 1 + \Frac{ b_t }{ 1 + c_t } } }
                 { \sin E + \Frac{ a_t }{ \sin E + \Frac{b_t}{ \sin E + c_t}}}
        \rp \cdot \flo{}{-3} h_{ort}
\eeq{p8:e9}

  Coefficients $a, b, c$ depends on latitude and time for the the hydrostatic
mapping function, and only on latitude for the non-hydrostatic mapping
function. The coefficients $a_t, b_t, c_t$ are constants for the hydrostatic
mapping function and zero for the non-hydrostatic mapping function:
%
\beq
   \begin{array}{l @{\qquad} l @{\qquad} l}
      a_h    = \dss\sum_k^7 a_{1k} \, B^1_k(|\phi|) -
               \dss\sum_k^7 a_{2k} \, B^1_k(|\phi|) \cos ( \psi_0 + \omega_{sa} t ) &
      a_w    = \dss\sum_k^7 a_{3k} \, B^1_k(|\phi|) &
      a_{ht} = \mbox{const} \\
%%
      b_h    = \dss\sum_k^7 b_{1k} \, B^1_k(|\phi|) -
               \dss\sum_k^7 b_{2k} \, B^1_k(|\phi|) \cos ( \psi_0 + \omega_{sa} t ) &
      b_w    = \dss\sum_k^7 b_{3k} \, B^1_k(|\phi|) &
      b_{ht} = \mbox{const} \\
%%
      c_h    = \dss\sum_k^7 c_{1k} \, B^1_k(|\phi|) -
               \dss\sum_k^7 c_{2k} \, B^1_k(|\phi|) \cos ( \psi_0 + \omega_{sa} t ) &
      c_w    = \dss\sum_k^7 c_{3k} \, B^1_k(|\phi|) &
      c_{ht} = \mbox{const} \\
%%
   \end{array}
\eeq{p8:e10}
%
   and $a_{wt} = b_{wt} = c_{wt} = 0$. Here $\omega_{sa}$ --- the angular
frequency, which corresponds to the period of one year, $B^1_k(|\phi|)$
is the B-spline of the 1st degree which depends on the absolute value of
the geocentric latitude.

  In a case if the distribution of residual refractivity is known, mapping
function can be computed by integration. Let us consider a case when 
a) the {\it residual} atmosphere is considered uniform, i.e. its density
does not depends on longitude and latitude; b) dependence of the 
{\it residual} refractivity on height $h$ is described by a Gaussian layer 
model: $r(h) = r_o \exp{-4\ln(2)\, (h-h_l)^2/w^2}$, where $h_l$ is the 
height of the layer with maximum residual refractivity and $w$ is the 
full width half maximum of the distribution; c) bending in the atmosphere
is neglected.

\begin{figure}
   \caption{Geometry of modeling residual refractivity}
   \par\vspace{1ex}\par
   \centerline{\mbox{ \epsfclipon
                      \epsfxsize=0.5\textwidth
                      \epsffile{mf_drawing.eps}
                    }
              }
   \label{f:mf}
\end{figure}

   According to that model, photon propagates along slanted direction $s$.
Within interval $\Delta d$ the path delay is proportional to $r(s) ds$.
The photon at at distance $s$ has height $h$ which by solving the triangle
with sides $R, s, R+h$ in figure~\ref{f:mf} is 
$ h = R (\sqrt{1 + 2 \sin e \, s/R + (s/R)^2} -1 )$. For the vertical
direction, the integral of refractivity over the path is proportional
to integral $ \dss\int \exp\{ -4\ln2 \: (h-h_l)^2/w^2 \} \, dh$. Thus, 
the mapping  function for the residual atmosphere is

\beq
   m(e) = \Frac{ \dss\int\limits_{0}^{\infty} \exp\{ - 4 \ln 2 \; R^2/w^2 \;
                 \Bigl(\sqrt{1 + 2\sin e \, y + y^2} - (1 + h_l/R)\Bigr)^2 \; dy \}
               }
               { \dss\int\limits_{0}^{\infty} \exp\{ 4 \ln 2 \; (h-h_l)^2/w^2 \} 
               \, dh
               },
\eeq{e:a8}
%
  where $y=s/R$. In practice, the limits of the integral in the denominator
falls below some value $\epsilon$. For instance, the expression under integral
is below $\flo{1.5}{-5}$ outside these limits: 
$h_{\min} = \min(0,h_l - 2w), \enskip h_{\max} = h_l + 2w$. Analogously, 
using the relationship between $y$ and $h$ by solving triangle with sides 
$R, s, R+h$ we find limits for the nominator: 

\beq
  y_{\min} & = & \sqrt { \sin^2 e + 2h_{\min}/R + (h_{min}/R)^2 } - \sin e,  \hfill \\
  y_{\max} & = & \sqrt { \sin^2 e + 2h_{\max}/R + (h_{max}/R)^2 } - \sin e . \hfill
\eeq{e:a9}

\subsection{Ionospheric path delay}

  Electromagnetic wave propagates in plasma with phase velocity
\beq
    v_p = c \sqrt{ 1 - \Frac{N_v \, e^2}{m_e \, \epsilon_o \, \omega^2} }
\eeq{e:i1}
where $N_v$ --- electron density, $e$ --- charge of an electron, $m_e$ ---
mass of an electron, $\epsilon_o$ --- permittivity of free space, $\omega$ ---
angular frequency of the wave and c --- velocity of light in vacuum. Phase
velocity in ionosphere is faster than velocity of light in vacuum.

  After integration along ray path, expanding expression \ref{e:i1} with
holding only the term of the first order, we get the following expression for
additional phase rotation caused by ionosphere:
\beq
    \Delta\phi = - \Frac{\alpha}{\omega}
\eeq{e:i2}
where $ \alpha $ is
\beq
     \alpha = \Frac{e^2}{ 2 \, c \, m_e \,  \epsilon_o }
              \lp \int N_v \, d s_1 - \int N_v \, d s_2 \rp
\eeq{e:i3}
  $s_1$ and $s_2$ are paths of wave propagation from a source to the
first and second station of the radio interferometer.

  If to express $\int N_v \, d s $ in units \flo{1}{16} electrons/$m^2$
(so-called TEC units) then after having substituted values of constants we get
\beq
   \alpha = \flo{5.308018}{10} \:\: \mbox{sec}^{-1}
\eeqn

  A fringe phase in channel $i$ is expressed as
\beq
     \phi_i = \tau_{ph} \, \omega_o + \tau_{gr} \, (\omega_i - \omega_o) -
              \Frac{\alpha}{\omega_i}
\eeq{e:i4}
  where $\omega_o$ --- is a reference sky frequency.

  Unknown quantities phase and group delays: $\tau_{ph}$ and $\tau_{gr}$ can
be determined from equations \eref{e:i4} by using weighted LSQ. Equations
of conditions for such a problem can be written as
\beq
     \tau_{ph} \Der{\phi_i}{\tau_{ph}} + \tau_{gr} \Der{\phi_i}{\tau_{gr}} =
     \phi_i + \Frac{\alpha}{\omega_i}
\eeq{e:i5}

  Actually, group and phase delays are obtained in fringing software by
minimizing delay resolution function. However, it can be demonstrated that
this method gives the same results as solving equations \eref{e:i5} by LSQ
provided
\begin{itemize}
   \item weights $ \Frac{(U_i + L_i) \, A_i}{\nu} $ are ascribed equations
         of conditions. $U_i$ and $L_i$ is the number of processed samples
         in upper and lower sideband of the $i$-th channel, $A_i$ amplitude
         in this channel, $\nu$ --- sampling rate.

   \item residual phases are not large. In practice the difference in estimates
         of group delay obtained by minimizing delay resolution function and
         by solving equations of conditions does not exceed 0.1~ps if
         residual phases are less than 0.4 rad (\deg{23}).
\end{itemize}

  Now let's obtain explicit expression for delays by solving \eref{e:i4}
by LSQ:
\beq
   \hat{x} = \lp( r\tra{A}) \, r A \rp^{-1} \, (r\tra{A}) \, x
\eeq{e:i6}
where $A$ --- matrix of equations of conditions, $r$ --- vector of weights,
$x$ --- vector of right parts of equations of conditions.

  Partial derivatives are
\beq
     \Der{\phi_i}{\tau_{ph}} = \omega_o
     \hspace{5em}
     \Der{\phi_i}{\tau_{gr}} = \omega_i - \omega_o
\eeq{e:i7}

  We can use the explicit expression for an invert of a symmetrical 2x2 matrix:
\beq
    \left(
    \begin{array}{r@{\qquad}r}
           A_{11} & A_{12}  \vspace{3.5ex} \\
           A_{12} & A_{22}  \\
    \end{array}
    \right)^{-1}
    =
    \left(
    \begin{array}{rr}
           \Frac{A_{22}}{\Delta} & -\Frac{A_{12}}{\Delta}  \vhex  \\
          -\Frac{A_{12}}{\Delta} &  \Frac{A_{11}}{\Delta}         \\
    \end{array}
    \right)
%
    \hspace{3em} \Delta = A_{11}A_{22} - A_{12}^2
\eeq{e:i8}

  We can easily find blocks of normal matrix and normal vector:
\beq
  \begin{array}{rcl}
    r A_{11} & = & \omega_o^2 \Sum_i^n r_i                           \vhex  \\
    r A_{12} & = & \omega_o   \Sum_i^n r_i (\omega_i - \omega_o )    \vhex  \\
    r A_{22} & = &            \Sum_i^n r_i (\omega_i - \omega_o )^2  \vhex  \\
    \Delta   & = & \omega_o^2 \lp \Sum_i^n r_i \cdot
                   \Sum_i^n r_i ( \omega_i - \omega_o)^2  -
                   \lp \Sum_i^n r_i ( \omega_i - \omega_o ) \rp^2 \, \rp
  \end{array}
\eeq{e:i9}

  Then we can write solution of system of normal equations \eref{e:i7}:
\beq
    \lp
    \begin{array}{l}
           \tau_{ph}   \vspace{3.5ex} \\
           \tau_{gr}   \\
    \end{array}
    \rp
%
    =
%
    \lp
    \begin{array}{rr}
           \Frac{A_{22}}{\Delta} & -\Frac{A_{12}}{\Delta}  \vhex \\
          -\Frac{A_{12}}{\Delta} &  \Frac{A_{11}}{\Delta}       \\
    \end{array}
    \rp
%
    \:
%
    \lp
    \begin{array}{l@{\:}c@{\:}l}
       \omega_o \Sum_i^n r_i \phi_i &+&
                \alpha \, \omega_o \Sum_i^n \frac{r_i}{\omega_i}       \vhex \\
       \omega_o \Sum_i^n r_i ( \omega_i - \omega_o ) \phi_i &+&
                \alpha \, \Sum_i^n {r_i}\frac{\omega_i -\omega_o}{\omega_i} \\
    \end{array}
    \rp
\eeq{e:i10}
  or after some algebra
\beq
   \begin{array}{r@{\,}c@{\,}l}
   \tau_{ph} &=& \Frac{ \Sum_i^n r_i (\omega_i - \omega_o)^2 \cdot
                      \Sum_i^n r_i \phi_i -
                      \Sum_i^n r_i ( \omega_i - \omega_o )
                               \Sum_i^n r_i ( \omega_i - \omega_o ) \phi_i }
                    {\omega_o \lp \Sum_i^n r_i \cdot
                       \Sum_i^n r_i ( \omega_i - \omega_o)^2  -
                       \lp \Sum_i^n r_i ( \omega_i - \omega_o ) \rp^2 \, \rp}
             - \Frac{\alpha}{\omega^2_{ph}}
%
   \vhex\vhex \\
%
   \tau_{gr} &=& \Frac{ \Sum_i^n r_i \cdot
                      \Sum_i^n r_i (\omega_i - \omega_o) \phi_i -
                      \Sum_i^n r_i ( \omega_i - \omega_o )^2
                               \Sum_i^n r_i \phi_i }
                    { \lp \Sum_i^n r_i \cdot
                          \Sum_i^n r_i ( \omega_i - \omega_o)^2  -
                      \lp \Sum_i^n r_i ( \omega_i - \omega_o ) \rp^2 \, \rp}
             + \Frac{\alpha}{\omega^2_{gr}}
   \end{array}
\eeq{e:i11}
  where $\omega_{ph}$ and $\omega_{gr}$ are
\beq
   \begin{array}{r@{\,}c@{\,}l}
   \omega_{ph} &=& \sqrt { \omega_o \, \Frac
                 { \Sum_i^n r_i \cdot
                   \Sum_i^n r_i (\omega_i - \omega_o)^2  -
                   \lp \Sum_i^n r_i ( \omega_i - \omega_o ) \rp^2 \,
                 }
%
                 {
                   \Sum_i^n r_i (\omega_i - \omega_o)
                   \Sum_i^n r_i \frac{(\omega_i - \omega_o)}{\omega_i} -
                   \Sum_i^n r_i (\omega_i - \omega_o)^2 \cdot
                   \Sum_i^n \frac{r_i}{\omega_i}
                 } }
%
   \vhex\vhex\vhex \\
%
   \omega_{gr} &=& \sqrt \Frac
                 { \Sum_i^n r_i \cdot
                          \Sum_i^n r_i ( \omega_i - \omega_o)^2  -
                      \lp \Sum_i^n r_i ( \omega_i - \omega_o ) \rp^2 \,
                 }
%
                 {
                   \Sum_i^n r_i ( \omega_i - \omega_o )
                   \Sum_i^n \frac{r_i}{\omega_i} -
                   \Sum_i^n r_i \cdot
                   \Sum_i^n r_i \frac{(\omega_i - \omega_o)}{\omega_i}
                 }
   \end{array}
\eeq{e:i12}

$\omega_{ph}$ and $\omega_{gr}$ are called effective ionosphere frequencies
for ionosphere contribution. They have clear physical meaning: if the
wide-band signal be replaced by a quasi-monochromatic signal with a group
or phase effective ionosphere frequency then contribution to group or phase
delay would be the same.

\subsubsection{Ionosphere-free linear combinations}

   Using notion of ionosphere effective frequencies we can express observed
group and phase delays at X and S bands through ionosphere free delay
$\tau_{if}$ and parameter $\alpha$:
%
\beq
    \begin{array}{rcl}
        \tau_{gx} & = & \tau_{if} + \Frac{\alpha}{\omega_{gx}^2}   \vhex \\
        \tau_{gs} & = & \tau_{if} + \Frac{\alpha}{\omega_{gs}^2}   \vhex \\
        \tau_{px} & = & \tau_{if} - \Frac{\alpha}{\omega_{px}^2}   \vhex \\
        \tau_{ps} & = & \tau_{if} - \Frac{\alpha}{\omega_{ps}^2}   \vhex \\
    \end{array}
\eeq{e:i13}
%
  Here the first letter in indexes stands for group or phase delay and the
second letter stands for X or S band. Using these equations we can eliminate
unknown parameter $\alpha$ and express ionosphere free delay through a linear
combination of two or three observables. The most important ionosphere-free
linear combination of observables are given below:
%
\beq
  \begin{array}{rcl}
   \mbox{G\_Gxs}
          & = & \Frac{\omega_{gx}^2}{\omega_{gx}^2 - \omega_{gs}^2} \tau_{gx} -
                \Frac{\omega_{gs}^2}{\omega_{gx}^2 - \omega_{gs}^2} \tau_{gs}
                \vhex \\
   \mbox{PxGs}
          & = & \Frac{\omega_{px}^2}{\omega_{px}^2 + \omega_{gs}^2} \tau_{px} +
                \Frac{\omega_{gs}^2}{\omega_{px}^2 + \omega_{gs}^2} \tau_{gs}
                \vhex \\
   \mbox{PxGx}
          & = & \Frac{\omega_{px}^2}{\omega_{px}^2 + \omega_{gx}^2} \tau_{px} +
                \Frac{\omega_{gx}^2}{\omega_{px}^2 + \omega_{gx}^2} \tau_{gx}
  \end{array}
\eeq{e:i14}

  Alternative way is to use expression for ionosphere contribution to
group delay at X band:
\beq
    \tau_{igx} = \Frac{\omega_{gs}^2}{\omega_{gx}^2 - \omega_{gs}^2} \,
                 (\tau_{gx} - \tau_{gs})
\eeq{e:i15}
%
  One can say $\tau_{igx}$ is to be added to  theoretical delay and thus 
it will ``correct'' or ``calibrate'' group delay at X band for the 
ionosphere contribution. This approach is rather ugly since ``calibration'' 
or ``correction'' to group delay already contains this quantity (we correct 
observable X using the measurement of this observable itself). In order 
to avoid this logical pitfall is its preferable that the concept of 
ionosphere-free linear combinations of observables should be used.

  Ionosphere frequencies vary from an experiment to experiment and they
even varies during the same experiment. The table below shows typical
{\it cyclic} ionosphere frequencies and ionosphere delays for experiment
c1014 (01JUL09XA):

\beq
  \begin{array}{rcl@{\quad\quad}rcl}
     f_{gx}     & = & \flo{8.557}{9} \:\:  \mbox{Hz}                  &
     f_{gs}     & = & \flo{2.293}{9} \:\:  \mbox{Hz}                  \\
     f_{px}     & = & \flo{8.215}{9} \:\:  \mbox{Hz}                  &
     f_{ps}     & = & \flo{2.233}{9} \:\:  \mbox{Hz}                  \\
     \tau_{igx} & = & 18.3  \:\, \Delta \mbox{TEC} \:\:  \mbox{ps}  &
     \tau_{igs} & = & 255.6 \:\, \Delta \mbox{TEC} \:\:  \mbox{ps}  \\
     \tau_{ipx} & = & 19.9  \:\, \Delta \mbox{TEC} \:\:  \mbox{ps}  &
     \tau_{ips} & = & 269.5 \:\, \Delta \mbox{TEC} \:\:  \mbox{ps}  \\
  \end{array}
\eeq{e:i16}

\subsubsection{Ionosphere calibration using ionosphere electron contents
from GPS}

  GPS observations are made at two frequencies, 1.2276 and 1.57542~GHz.
A linear combination of two observables provides an estimate of the 
instantaneous TEC. Analysis of continuous GPS observations from a global 
network  comprising 100--300 stations makes it feasible to derive an
empirical model of the total electron contents over the span of 
observations using data assimilation technique. Such a model is routinely
delivered by GPS data analysis centers since 1998. The model provides 
values of the TEC on a regular 3D grid. The axes of the grid are longitude,
latitude, and time. The accuracy and resolution of GPS TEC model is 
constantly improving and is expected to improve in the future. In 2010,
several analysis centers produced TEC model outputs with spatial 
resolution $5^\circ \times 5^\circ$ and time resolution 2~hours.

\begin{figure}
   \caption{Ray passing through the ionosphere \label{f:iono}}
   \par\vspace{1ex}\par
   \centerline{\mbox{ \epsfclipon
                      \epsfxsize=0.5\textwidth
                      \epsffile{iono_300.eps}
                    }
              }
\end{figure}

  For the purpose of modeling, the ionosphere is considered as a thin
spherical layer at the constant height $H$. Typical value of $H$ is 450~km.
In order to compute the TEC from GPS maps we need to know the coordinates
of the point at which the ray pierce the ionosphere --- point J in 
figure~\ref{f:iono}. First we find the distance from the station to the 
ionosphere piercing point $D = |SJ|$ by solving triangle $OSP$. Noticing
that $|OS| = R\earth$ and $|OJ| = R\earth +H$, we immediately get
%
\beq
   \begin{array}{rcl}
     \beta & = & \arcsin \Frac{ \cos E }{1 + \frac{H}{R\earth}} \vex \\
      D    & = & R\earth \sqrt{ 
                  2 \frac{H}{R\earth} \lp 1 - \sin\lp E + \beta \rp\rp +
                    \lp \frac{H}{R\earth} \rp^2 }
   \end{array}
\eeq{e:i17}

  Then Cartesian coordinates of point $J$ are $\vec{r} + D\vec{s}$. 
Transforming them into polar coordinates geocentric latitude and longitude,
we get arguments for interpolation in the 3D grid. Since the accuracy of
TEC models relatively low, the choice of interpolation is irrelevant.
The VTD uses 3-dimensional B-spline interpolation by expanding the TEC field
into the tensor products of basic splines of the 3rd degree, although linear 
interpolation between the closest nodes of the grid would be sufficient.
Interpolating the TEC model output, we get the TEC through the vertical
path $|JI_o|$. The slanted path $|JI_1|$ is $|JI_o|/\cos{\beta}$. Therefore, we 
need to multiply the vertical TEC by $1/\cos{\beta(E)}$, which maps the 
vertical path delay through the ionosphere into the slanted path delay. Here we
neglect the ray path bending in the ionosphere. We also neglect Earth's
ellipticity, since the Earth was considered spherical in the data assimilation
procedure of the TEC model.

  Combining equations, we get the final expression for the contribution
of the ionosphere to path delay:
%
\beq
  \tau_{iono} = \pm \Frac{\alpha}{4\, \pi^2 \, f^2_{\mbox{eff}} } \: 
                \mbox{TEC} \: \Frac{1}{\cos{\beta(E)}} \quad,
\eeq{e:i18}
  where $f_{\mbox{eff}}$ is the effective {\it cyclic} frequency, and sign
plus is for contribution to group delay, and sign minus is for contribution
to phase delay.

\subsection{Delay caused by source structure}

  In derivation of the expression for VLBI delay we assumed that the source
is point-like. In general, the complex coherence function $\Gamma_{12}$
according to the Van~Zitter--Zernike theorem is
%
\beq
   \Gamma_{12}(b_x,b_y,\omega) = e^{\tss i\omega\tau_0} \dintinf B(x,y,\omega)
         e^{\tss -2\pi i (x b_x + y b_y )/ \lambda} dx\,dy
\eeq{p9:e1}
  where $\omega$ is the angular reference frequency of received signal,
$\lambda$ is the wavelength wavelength, $\tau_0$ --- the geometric delay to
the reference point on the source, $B$ --- the two-dimensional function
of the brightness distribution, which depends on local Cartesian spatial
coordinates $x$, $y$. These coordinates are zero in the point to
which $\tau_0$ corresponds. $b_x$, $b_y$ are projection of the baseline
vector $\vc{b} = \vc{r}_1 - \vc{r}_2$ to the plane that is perpendicular
to the center of the map (x=0, y=0). Integral \eref{p9:e1} is called
visibility function.

  Ort \vce{y} of the local Cartesian spatial coordinates related to the source
position is defined as a unit vector which lies in the plane of vectors
$\vce{S}$ and $\vce{z}$ and is perpendicular to \vce{s} (where \vce{z} is
the unit vector in the direction of the pole, i.e $\vce{z}_{\sc\tiny t} =
(0, 0, 1){}^{{}^{\bf \top}}$). Ort $x$ is defined as the vector which is
perpendicular to both \vce{s} and \vce{y}, in such a manner that three vectors
\( (\vc{x}, \vc{y}, \vce{s}) \) would form the right triplet:
%
\beq
  \begin{array}{rcl}
      \vc{x} & = & \Frac{ \vc{z} \vpd \vce{s} }{ |\vc{z} \vpd \vce{s}| }   \\
      \vc{y} & = & \Frac{ \vce{s} \vpd \vc{x} }{ |\vce{s} \vpd \vc{x}| } =
                   \Frac{ \vc{z} - \vce{s} \spd ( \vce{s} \spd \vc{z} )  }
                        {|\vc{z} - \vce{s} \spd ( \vce{s} \spd \vc{z} )| }
  \end{array}
\eeq{p9:e2}

   The phase of the coherence function is expressed as
%
\beq
     \Phi = \tau_0 \omega + \arctg
            \Frac{\mbox{Im}\, V(b_x,b_y,\omega)}{\mbox{Re}\, V(b_x,b_y,\omega)}
\eeq{p9:e3}
%
  Here we denoted the visibility function (integral in \ref{p9:e1}) with
letter $V$. The contribution of the source structure to phase delay is the
second term in \ref{p9:e3} divided by the speed of light:
%
\beq
    \tau_{ps} = \Frac{1}{c} \arctg
            \Frac{\mbox{Im}\, V(b_x,b_y,\omega)}{\mbox{Re}\, V(b_x,b_y,\omega)}
\eeq{p9:e4}

  The group delay is determined as
%
\beq
    \tau_{gs} = \Der{}{\omega} V(b_x,b_y,\omega)
\eeq{p9:e5}

  We can transform expression \ref{p9:e5} to
\beq
  \begin{array}{r@{}l}
   \tau_{str}(b_x,b_y) = \Frac{2\pi}{c|V|^2} \Biggl[ &
                       \Re V(b_x,b_y)\cdot\lp \vc{b}_s \spd
                       \Vc{\rm \Im \lp \nabla V(b_x,b_y) \rp } \rp -
                                                  \\ &
                       \Im V(b_x,b_y)\cdot\lp \vc{b}_s \spd
                       \Vc{\rm \Re \lp \nabla V(b_x,b_y) \rp } \rp
                                              \Biggl]
  \end{array}
\eeq{p9:e6}

   where
\beq
  \mbox{ \( \vc{b}_s = \lp \begin{array}{l@{}}
                                     \vc{b} \spd \vc{x}  \\
                                     \vc{b} \spd \vc{y}
                              \end{array}
                          \rp \)  }
  \qquad
  \mbox{ \( \Vc{\nabla V} = \lp \begin{array}{l@{}}
                                          \Der{}{b_x} V \\
                                          \Der{}{b_y} V
                                   \end{array}
                               \rp \) }
\eeq{p9:e7}

   This gives us an expression for the contribution of the source structure
to group delay in the general form.

\subsection{Coupling effects}

  There is a term $\tau_p$, the propagation delay, in the expression for
time delay \ref{e:e13}. It is convenient to take it out the expression
\ref{e:e13}, since the propagation delay depends on meteorological
parameters which cannot be predicted in advance. If we take the term
$\tau_p$ out of \ref{e:e13}, than the total delay is
%
\beq
    t_2 - t_1 = (t_2 - t_1){}_{geom} + \tau_p -
           \Frac{1}{c} \vc{V}\earth \cdot \vce{S} \, \tau_p
\eeq{e:c1}
%
   where $(t_2 - t_1){}_{geom}$ is the contribution to delay without
delay in the medium of propagation. The last term which give rises from
the denominator in \ref{e:e13} is called coupling between the geometric
delay and the propagation delay.

  The second coupling effects give rise due to change of the height of the
phase center of the receiver due to slewing as
$\Delta h = \vc{fb} \cdot \vc{r}_{3\sc\tiny c}$. The path
length through the atmosphere is changing and therefore, the tropospheric
path delay. Since the expression for the zenith was referred to the reference
point of the antenna, a small correction is needed. Since the tropospheric
path delay is proportional to the surface pressure,
$\Delta\tau_p = \tau_p \Frac{\Delta P}{P}$. Using expression \ref{p8:e7},
we get
%
\beq
    \Delta \tau_{ca} = \flo{-1.1859}{-4}
       \lp
           \tau_{t1} \, \vc{fb}_1 \cdot \vc{r}_{1,3\sc\tiny c} -
           \tau_{t2} \, \vc{fb}_2 \cdot \vc{r}_{2,3\sc\tiny c}
       \rp
\eeq{e:c2}
  where $\vc{r}_{1,3\sc\tiny c}$ denotes the 3rd component of the position
vector in the geocentric inertial coordinate system of the 1st antenna,
and $\vc{r}_{1,3\sc\tiny c}$ denotes the 3rd component of the position
vector of the 2nd antenna. $\tau_{ti}$ is the tropospheric path delay at
the $i$th antenna. The additional delay \ref{e:c2} is called ``coupling
between the antenna axis and the tropospheric path delay''.

\section{Implementation of computation of the theoretical path delay, delay rate
and partial derivatives with respect to parameters of model}
\label{s:implementation}

  The model described above is implemented in library VTD. Computation
procedures is specified in the control file. The control file has the
syntax of a pair {\sf keyword}---{\tt value}. The value may be either
the name of the file with a~priori values, or the option code, or the name
of another control file. The computation procedure for delay, delay rate
and partial derivatives of delays over parameters of the model has many
options. Care should be taken for using options. Some options are designed
for comparison tests only, some options produce correct results only if
they correspond to specific a~priori files.

\subsection{Computation before processing the first observation}

  Some reduction quantities can be computed beforehand when a range of
instants of observations, typically 24 hour or less, is known.

  Before processing the first observation station coordinates from the input
catalogue specified in the keyword {\sf STATION\_COORDINATES} are read.
The following quantities are computed for each station: longitude, geocentric
latitude, geodetic latitude, orthometric height (the hight with respect to the
reference ellipsoid), local gravity acceleration, transformation matrix from
the local topocentric coordinate system Up--East--North to the Cartesian
terrestrial coordinate system as well as the transformation matrix from
the local topocentric coordinate system Radial--East--North to the Cartesian
terrestrial coordinate system. In the first case the geodetic latitude is
used in expression \ref{e:e40} and the second case the geocentric latitude.
Time independent vectors $ \vec{X}^{rc}_j(m,\phi), \vec{X}^{rs}_j(m,\phi),
\vec{X}^{ic}_j(m,\phi), \vec{X}^{is}_j(m,\phi), X^{3c}_j(m,\phi),
X^{3s}_j(m,\phi) $ which are used for computation of displacement caused
by the solid Earth tides, expressions \ref{e:a05}, \ref{e:a16}, are computed
and stored at this step.

  The station motion is determined with several models. Contribution from each
model is summed up.

  Station linear motion is determined by velocities in the apriori file
specified in the the keyword {\sf STATION\_VELOCITIES}. These a~priori velocities
are either adjusted in the VLBI solution or computed on the basis of
apriori models, f.e. NUVEL if no observation at that station was made.

  For each station the B-spline model of 0-th degree is defined in the file
specified by the keyword {\sf STATION\_ECCENTRICITIES}. These are either
motions of the antenna reference point with respect to the ground marker or
displacements due to human activity, for instance, rail repairing, measured
with a high accuracy local survey. For many stations this model is zero.

  The type of antenna mounting, the length of the antenna axis offset
$|\vc{fb}|$ and the code of the tectonic plate where the station resides
are defined in the file specified by the keyword {\sf STATION\_DESCRIPTION}.

  Traditionally, time tags of VLBI formatters are shifted to show
pseudo--UTC. The UTC is a non-differentiable function of time. It can be
represented in a form of expansion over the basis of B-splines of zeroth
degree. The table with epochs and amounts of jumps is specified by the
keyword {\sf LEAP\_SECOND}. {\bf NB:} formatter tag keeps the so-called
pseudo-UTC. If the jump of UTC(t) function took place in the middle of
an experiment, this jump is not applied till the end of an experiment.
Therefore, in order to get a TAI instance of time that corresponds to UTC(t),
one should substitute as an argument of [UTC-TAI](UTC) not the time tag at
the moment of an observation, but the time tag at the moment of a nominal
session start.

  Position of big planets, the Sun, the Earth and the Moon are computed
in accordance with numerical ephemerides using Chebyshev polynomials.
Ephemerides DE403 and DE405 are supported. The file name is specified by
the keyword {\sf DE\_EPHEMERIDES}. The argument for the numerical ephemerides
is TDB. The difference $TDB - TAI(t)$ is
\beq
    TDB(t) = TAI + \dss\int\limits_{t_0}^t \biggl( \Frac{1}{2c^2} v^2 +
                 \Frac{U}{c^2} - L_B \biggr)  \; dt  + 32.184
\eeq{e:im1}
  This differs from \ref{e:e3} only by term $L_B$ under the integral. With
accuracy $10^{-5}$ s we can get the simplified expression for TDB:
%
\beq
     TDB(t) = t \: + \: 32.184 \: + \:
              A_1 \, \sin ( \phi_{SA} + \omega_{SA} T_J ) +
              A_2 \, \sin ( 2(\phi_{SA} + \omega_{SA} T_J) )
e\eeq{e:im2}
%
   where $\omega_{SA} = \flo{1.990968752920}{-7}$  rad s${}^{-1}$ is the
annual frequency, $\phi_{SA} = 6.240076$ rad, $A_1 =  0.001658$ s, and
$A_2 =  \flo{1.4}{-5}$ s

\subsection{Computation of the rotation matrix from the terrestrial coordinate
            system to the celestial coordinate system}

  There are several options to compute the rotation matrix which accounts
for the Earth's rotation. All these options involves a sum of the secular
model and coefficients of the empirical expansion. The secular model is
either analytical, or semi-analytical or empirical. It is valid for
a long period of time (more than 10 years). The empirical expansion is valid
for a short period of time (0.5--5.0 days). The accuracy of the resulting
rotation matrix depends on accuracy of the a~priori empirical expansion.
There is no ``good'' or ``bad'' a~priori model to the rotation matrix,
but the model can be consistent or inconsistent,
However, the empirical expansion was made on the basis of a certain secular
model. Therefore, they both should be applied as a consistent pair.
An inconsistent pair of model may potentially have very large errors.
Library VTD similar to other modern astronomical reduction programs supports
a wide range of options. Care should be taken to set options which are
consistent.

\subsubsection{IERS time series approach}

   In accordance to that approach parameters $\zeta_0, \theta_0, z_0,
\epsilon_0$ are computed using an expansion over low degree polynomials,
$\delta\psi, \delta\epsilon$ are computed using expansion over the
quasi-harmonic basis, UT1, $X_p$, $Y_p$ are computed by interpolating time
series. One of the modifications of the IERS time series approach suggests
using time series of expansion of empirical corrections to $\delta\psi,
\delta\epsilon$. Library VTD does not implement such a modification.

  The file with time series for UT1, $X_p$, $Y_p$ is specified by keyword
{\sf EOP\_SERIES}. {\tt EOP-MOD Ver 2.0} format of input EOP files is
supported. The keyword {\sf EOP\_TIME\_SCALE} specified by name if the argument
used in that file: {\tt TAI}, or some function of time: {\tt TDB}, {\tt TDT},
{\tt UTC}, {\tt UT1}. Coefficients of interpolating spline of the 3rd degree
are computed for UT1, $X_p$, $Y_p$. An option to subtract harmonic model
of variations in UT1 caused by zonal model is supported. The name of the
model is specified by keyword {\sf UZT\_MODEL}. The following values are
supported: {\tt DICKMAN1993}, {\tt DICKMAN1993\_SHORT}, and 
{\tt DICKMAN1993\_PRINCIPLE} for \cite{r:dickman1993} model. 
Value {\tt DICKMAN1993} means that all term of that model are used. 
Value {\tt DICKMAN1993\_SHORT} means that terms with periods less than 
60 days are used. Value {\tt DICKMAN1993\_PRINCIPLE} means that 14 terms 
of the expansion are used, the contribution of omitted terms to UT1 rate being
less than $10^{-14}$ rad~s${^{-1}}$. The value of the keyword {\sf UZT\_USE}
specifies how to apply contribution to UT1 caused by zonal tides. Value
{\tt ADD} means that the contribution to UT1 will be added to function UT1
computed on the moment of observation; value {\tt SUBTRACT} means that the
contribution to UT1 will be subtracted from function UT1 computed on the
moment of observation; value {\tt INTERPOLATE} means that the contribution
to UT1 will be subtracted from tabulated values of UT1 before computing
coefficients of the interpolating and added back to function UT1 computed
on the moment of observation.

  Precession expansion is defined by keyword {\sf PRECESSION\_EXPRESSION}.
It can have values either {\tt LIESKE\_1977} for \cite{r:lieske1977}
semi-empirical low degree polynomial expansion or {\tt CAPITAINE\_2003}
for low degree polynomial expansion \cite{r:Cap03a}.

  Nutation expansion is defined by keyword {\sf NUTATION\_EXPANSION}.
It can have the following values:
\begin{itemize}
   \item {\tt WAHR1980}        --- for Wahr 1980 expansion;
   \item {\tt IERS1996}        --- for IERS1996 expansion;
   \item {\tt REN2000}         --- for REN2000 expansion;
   \item {\tt MHB2000}         --- for MHB2000 expansion;
   \item {\tt MHB2000\_TRANSF} --- for the transfer function from the MHB2000
                                   expansion applied to the REN2000 expansion
                                   without so-called add-on ad hoc terms;
   \item {\tt PETA}            --- for the shortened version of the REN2000
                                   expansion which comprises only of three
                                   largest terms.
\end{itemize}

  Including geodesic nutation in $\Delta\psi, \Delta\epsilon$ computation
is optional and is controlled by the keyword {\sf GEODESIC\_NUTATION}.
Value {\tt YES} forces to add contribution of the geodesic nutation
to nutation angles.

  Computation of the rotation matrix maybe slightly altered of keyword
{\sf EROT\_COMPAT} has value {\tt CALC10}. In that case
a)~additional spurious terms are added to parameter $S$:
1)~\mbox{\flo{7.07827}{-8}} --- residual in the LLR solution of Chapront;
2)~\mbox{\flo{-4.557249}{-10}} --- accumulated difference of using UT1
instead of TAI in expression for $S$ in the past
3)~\mbox{\flo{4.462899}{-20}} \, $T_J$ \mbox{rad s${}^{-1}$} unknown term;
\mbox{\flo{-7.220525}{-20}} \, $T_j$ \mbox{rad s${}^{-1}$} --- spurious term
introduced by N.~Capitaine in order to have a symmetry in her formulae;
b)~terms $\delta\Psi_0$ and $E_{p0}$ are subtracted from $\delta\Psi$ and
$\delta\epsilon$, and the rotation transformation matrix is multiplied from
the left by
$ \mat{R}_1(-\Delta\Psi_0\sin\epsilon_0) \cdot \mat{R}_2(E_{p0}) \cdot
  \mat{R}_3(\flo{7.07827}{-8})$. This change is done for comparison
tests with Calc10.

  The keyword {\sf HARMONIC\_EOP\_FILE} specified the file with empirical
harmonic variations in the Earth rotation with respect to some apriori
model. It compensates errors in the nutation model and omitted terms which
are not forced nutations, for example, terms excited by the oceanic
response and the free core nutation.

\subsubsection{Earth rotation model approach}

  In that case instead of specifying the file with time series of $X_p$,
$Y_p$, UT1, code of precession, nutation, the file with 30 coefficients 
of the a~priori Earth rotation model is specified. The apriori model 
of expansion of the vector of of small perturbational rotation $\vc{q}$(t) 
over the B-spline basis and Fourier basis can optionally be specified in
keywords {\tt ERM\_FILE} and {\tt HARMONIC\_EOP\_FILE}.

  Typical setup when the IERS time series approach is selected

\begin{verbatim}
AEM_FILE:                NONE
ERM_FILE:                NONE
HARMONIC_EOP_FILE:       NONE
#
EOP_SERIES:              {file_name}
EOP_TIME_SCALE:          TDB
UZT_MODEL:               NONE
UZT_USE:                 NONE
PRECESSION_EXPRESSION:   CAPITAINE_2003
NUTATION_EXPANSION:      MHB2000
GEODESIC_NUTATION:       NONE
EROT_COMPAT:             NONE
\end{verbatim}

  Typical setup when the ERM approach is selected

\begin{verbatim}
AEM_FILE:                {file_name}
ERM_FILE:                {file_name}
HARMONIC_EOP_FILE:       {file_name}
#
EOP_SERIES:              NONE
EOP_TIME_SCALE:          NONE
UZT_MODEL:               NONE
UZT_USE:                 NONE
PRECESSION_EXPRESSION:   NONE
NUTATION_EXPANSION:      NONE
GEODESIC_NUTATION:       NONE
EROT_COMPAT:             NONE
\end{verbatim}

\subsection{Site displacement models}

   Computation of displacements caused by solid Earth tides is controlled
by several keywords. The keyword {\sf SOLID\_EARTH\_TIDES\_2ND\_DEGREE}
specifies the name of the model for the dependence of the generalized Love
numbers on frequency for the solid Earth tides of the 2nd degree, except
the constituent with zero frequency. The models differ in parameters of
the resonance close to the nearly diurnal free wobble and effect
of anelasticity on Love numbers at low frequencies.

\begin{itemize}
   \item  {\tt MDG97EL} --- The elasticity variant of \cite{r:mdg97} model;
   \item  {\tt MDG97AN} --- The anelasticity variant of \cite{r:mdg97} model;
   \item  {\tt DDW99EH} --- The equilibrium variant of \cite{r:ddw98} model;
   \item  {\tt DDW99IN} --- The non-equilibrium variant of \cite{r:ddw98} model;
   \item  {\tt LOVE}    --- Love numbers are considered to be
                            frequency independent: $h = 0.609, l = 0.0852$;
   \item  {\tt MATHEWS\_2000} --- The variant of Mathews model as it was presented
                                  in IERS Conventions 2003;
   \item  {\tt MATHEWS\_2001} --- The variant of \cite{r:mat01} model'
   \item  {\tt NONE}          --- Displacements caused by solid tides are
                                  considered to be zero.
\end{itemize}

  The expansion of the tide-generating potential contains terms of zero
frequency. This constituents in the expansion induce the permanent displacement
which is included in estimates of site position. The keyword
{\sf SOLID\_EARTH\_TIDES\_ZERO\_FREQ} specifies the Love number which is
to be used for computing displacement due to zero-th frequency in the
tide-generating potential, i.e. the permanent tide.
%
\begin{itemize}
   \item  {\tt MDG97AN} --- The anelastisity variant of \cite{r:mdg97} model;
%
   \item  {\tt FLUID}   --- The Love numbers of the fluid limit:
                            $ h_2(\omega\!\!=\!\!0) = 0.94, \quad
                              \ell_2(\omega\!\!=\!\!0) = 0$;
%
   \item  {\tt ZERO}    --- $h_2(\omega\!\!=\!\!0) = 0.0, \quad
                             \ell_2(\omega\!\!=\!\!0) = 0.0$,
                            i.e. the displacement vector will have only
                            periodic terms and the zero mean.
\end{itemize}

  Displacements caused by pole tide are computed using the Love numbers
model specified by the keyword {\sf MPL\_FILE}. The supported values of
this keyword are the same as for the keyword
{\sf SOLID\_EARTH\_TIDES\_2ND\_DEGREE}. Parameters for the linear
model of components 1,2 of the perturbing vector of the Earth rotation
$\vc{q}$ are defined in an external file. The name of this external file
is specified in the keyword {\sf MPL\_FILE}. Value {\tt NONE} means that
parameters $E_1, E_2, E_{12}, E_{22}$  are zero.

   Computation of displacements caused by solid Earth tides of the 3rd
degree is control be keyword {\sf SOLID\_EARTH\_TIDES\_3RD\_DEGREE}.
Two values are supported: {\tt NONE} and {\tt MDG97}. Value {\tt NONE}
means that no displacements caused by tides of the 3rd degree should be
computed. Value {\tt MDG97} means that the Love numbers of the 3rd
degree according to \cite{r:mdg97} should be used.

  The keyword {\sf AXIS\_OFFSET\_MODEL} specifies whether the contribution
to delay caused by motion of the receiver's phase center with respect to
the antenna's reference point should be taken into account (value {\tt YES}),
or not (value {\tt NONE}).

  Displacements caused by various loading are computed by stand-alone programs.
Displacements at a specific station caused by ocean loading are computed by
evaluating the convolution integral \ref{e:b2} of the ocean tides model with
appropriate Green's functions. The ocean tides model presents the complex
amplitude of sea level change at a latitude/longitude grid for each harmonic
constituent. The ocean tides model typically consists of 8--18 constituents.
As a results of evaluating the convolution integral, complex amplitude
for three components of the position vector in local topocentric coordinates
system Up--East--North are computed.

  Displacements at a specific station caused by the atmospheric pressure
loading are computed by evaluating the convolution integral \ref{e:b2} and
the surface atmospheric pressure field from global numerical weather models.
First, the diurnal and semi-diurnal variations in surface pressure of the
global atmospheric field are removed, since the atmospheric pressure variations
at these frequencies in numerical weather models are corrupted, because their
frequencies are close to the Nyquist frequency of the sampling input
meteorological data. The semi-diurnal pressure variations are transformed to
a standing wave, and the diurnal variations are folded with the ter-diurnal.
Amplitudes of the diurnal and semi-diurnal signal present in the data are
computed by LSQ fitting over a long time interval and subtracted from the
initial pressure field fro the numerical weather mode. The convolution integral
is computed separately over the land and over the sea. The actual surface
pressure is used for computing the contribution to the convolution integral
over the land. However under the ocean surface is deformed due to atmospheric
pressure changes which should be taken into account. Currently, no reliable
methods for evaluation of the oceanic response were proposed. Two extreme
cases are modeled: so-called inverted barometer hypothesis which assumes
that local atmospheric pressure variations are fully compensated by sea
height variations, and pressure variation at the sea floor are zero,
and the non-inverted barometer hypothesis which neglects sea height
variations. However the inverted barometer hypothesis violates sea mass
conservation. Therefore, it is modified by adding a term
%
\beq
   \Delta \bar{P}_o =
          \Frac{\dint\limits_{\hbb\hbb\: ocean} \Delta P(\vec{r}\,',t)
               \cos \phi' d \lambda' d \phi'}
               {\dint\limits_{\hbb\hbb\: ocean} \cos \phi' d \lambda' d \phi'}
\eeq{e:mass}
%
   which is applied uniformly at the sea floor. This pressure term, which
depends only on time but does not depend on spatial coordinates, is used
for computation of the contribution to the convolution integral over the
ocean.

  Since the global pressure field used for computing convolution integrals
is taken from the time series of numerical weather models, the resulting
displacements for stations of our interest are computed in the form of time
series with the time interval of numerical weather models.

  The contribution to the atmospheric pressure loading displacement
caused by atmospheric tides is computed on the basis of the global model of
atmospheric tides at the diurnal and semi-diurnal frequency. The complex
amplitude is convolved with Green's function invoking the non-inverted
barometer hypothesis in a way similar to computing displacement caused by
ocean tides.

  The hydrology loading is computed in a similar way as the ocean loading:
the global model of the pressure caused by stored water in land is convolved
with Green's function. The resulting displacements for stations of our
interest are computed in the form of time series with the time interval of
hydrology models.

  In addition to various mass loading, stations may have irregular
displacements. These irregular displacements are estimated from analysis
of VLBI observations. They are presented in the most general form: in the
form of expansion over B-spline basis of an arbitrary order with
non-equidistant nodes which may be multiple.

  Various models of site displacements computed externally can be applied
for computation of time delay. More than one external model can be used.
Including these contributions to the model is controlled by the following
keywords:

\begin{itemize}
    \item {\sf POSVAR\_FIL}. This keyword specifies the file name with
               coefficients of external displacement supplied in the second
               value. The first value specifies the model index. The model
               index should be 1 for the first model, 2 for the second model,
               etc.
    \item {\sf POSVAR\_MOD}. This keyword specifies the type of displacement
               model in the second value. The first value specifies the model
               index. The following second values are allowed:
%
               \begin{itemize}
                  \item  {\tt TIME\_SERIES} --- the displacement model is in
                         the form of time series which are stored in files
                         in BINDISP format;
                  \item  {\tt HARMONIC\_MODEL} --- the displacement model is in
                         the form of coefficients of the harmonic expansion.
                         The model is stored in a file in HARPOS format.
                  \item  {\tt B\_SPLINE} --- the displacement model is in
                         the form of coefficients of B-spline.
                         The model is stored in a file in BSP format.
               \end{itemize}
%
    \item {\sf POSVAR\_INT} --- This keyword specifies the type of
               interpolation between nodes of displacement time series.
               The first value specifies the model index. The following second
               values are recognized:
%
               \begin{itemize}
		   \item {\tt CLOSE\_POINT}
		   \item {\tt LINEAR}
		   \item {\tt SPLINE}
               \end{itemize}
%
    \item {\sf POSVAR\_USE} --- This keyword specifies the action which is
               be performed in the case if no coefficients were found for
               a station of a baseline in the process of applying the
               displacement model. The first value specifies the model
	       If the second valued is {\tt REQUIRED}, then this situation
               is considered as a fatal error, and the process of computation
               will be terminated. If the second valued is {\tt USE},
               then a warning will be printed, but the process of computation
               will continue.
\end{itemize}
%POSVAR_FIL:  3  /vlbi/solve/apriori_files/aplo_bds/
%POSVAR_MOD:  3  TIME_SERIES
%POSVAR_INT:  3  LINEAR
%POSVAR_USE:  3  IF_AVAILABLE

\subsection{Computation of delay caused by propagation media. 
            Rigorous approach.}

  Rigorous computation of the path delay in neutral atmosphere requires 
knowledge of distribution of aid temperature, atmospheric pressure and air 
humidity. It was shown by L. Petrov (2010, manuscript in preparation)
that the atmosphere path delay can be computed with accuracies of 1--3~cm 
at elevation angles of 20--{$90^\circ$} from the output of modern numerical
weather models. Rigorous computation of the atmosphere path delay requires 
voluminous input data (1Tb per year or more) and requires significant 
computing resources. It is not practical to include these computations in VTD. 

  Slanted path delay is computed for each station in a form of a uniform
3D series. The last dimension of the grid is time. The time step 
corresponds to the time step of numerical weather model (3 or 6 hours). 
The second dimension is azimuth and the first dimension is a function of
elevation $m_i(e)$. function $m_i(e)$ is the so-called ``mapping function''
for the ISO atmosphere: $\frac{t_{na}(e)}{t_{na}(\pi/2)}$ --- the ratio of 
the atmosphere path delay at a given direction to the atmosphere path delay
in zenith direction computed for the reference model of the atmosphere ISO
ISA at geoid at the latitude $45^\circ$. Function $m_i(e)$ is computed by
numerical integration of equations of wave propagation, and is represented 
in the form of $m_i(e) = \sum C_k T_k((\Frac{1}{\sin{e+\alpha}}-a)/b)$ 
where $T_k(x)$ is the Chebyshev polynomial of the $k$-th degree, $a$, $b$,
$\alpha$, $C_k$ are some coefficients. The function reciprocal to $m(e)$,
$m^{-1}(r)$, is represented as 
$m^{-1}(r) = -\alpha + \arcsin(1/\sum D_k T((r-c)/d) )$.

  Interpolation of the slated path delay $\tau_{na}(e,A,t)$ is performed 
by the following way. First, for the range of epochs within observing 
session (3 epochs before the experiment start and 3 epochs after the experiment, 
each station the 3D array of slanted path delay for dimensions 
$r,A,t ( r=m_i(e) $ is extracted. Then the 3D array of B-spline coefficients 
that represents the slanted path delay is computed on place and replaces 
the array of 3D path delay.

  Finally, the slanted path delay at elevation $e$, azimuth $A$ at time 
epoch $t$ is computed using these interpolation B-spline coefficients as
a function $(r,A,t)$. Coordinate $r = m_i(e)$. The partial derivative of
the slanted path delay with respect to the path delay in zenith direction
$\Frac{\tau_{na}(e)}{\tau_{na}(\pi/2)}$ is computed as 
$\Frac{\tau_{na,nh}(e)}{\tau_{na,nh}(\pi/2)}$, where $\tau_{na,nh}(e)$ is the
contribution of the non-hydrostatic constituent of the atmosphere on path delay.

\subsubsection{Computation of path delay through the neutral atmosphere using
               the output of numerical weather models.}

  Numerical whether models produces the 4D field of the atmospheric pressure,
air temperature, specific humidity, and other parameters on a non-regular, global
4D grid. For computing path delay, the non-regular grid is replaced with the
regular grid through re-gridding and the use of curvilinear coordinate.
The curvilinear coordinates $H,L,P,T$ are related to the Cartesian, crust-fixed
coordinates $x,y,z,t$ through the matrix of transformation $\mat{H}$, such
that $\vec{R(H,L,P,T)} = \mat{H}\vec{r(x,y,z,t)}$. The coefficients of the 
transformation $\mat{H}$ are global and do not depend on coordinates.

  For a given station, local Cartesian coordinates $\xi,\eta,\zeta$ with the 
original ant the station reference point are introduced, such that the direction 
of axis $\xi$ is along the direction to the emitter, as it were in the absence
of the atmosphere, $\eta$ is the perpendicular to $\xi$ an lies in the plane 
of $\xi$ and the Earth pole, and $\zeta = \xi \times \zeta$. It was shown by
L. Petrov 2010, paper in preparation, that exploiting the Fermat principle and 
solving the variational problem for finding the trajectory through the 
heterogeneous atmosphere, the differential equations of wave propagation 
can be written in this form:

\beq
    \begin{array}{lcl}
       \Frac{d^2\eta}{d\xi^2} \: + \: \Frac{n_\xi}{n} \Frac{d\eta}{d\xi}   &  = &  
                     \Frac{n_\eta}{n} 
                     + U(n,n_\xi, n_\eta, n_\zeta, \Frac{d\eta}{d\xi},
                        \Frac{d\zeta}{d\xi}, \Frac{d^2\zeta}{d\xi^2} )
       \vex \\
       \Frac{d^2\zeta}{d\xi^2} \: + \: \Frac{n_\xi}{n} \Frac{d\zeta}{d\xi} &  = &  
                      \Frac{n_\zeta}{n}    
                      + V(n,n_\xi, n_\eta, n_\zeta, \Frac{d\eta}{d\xi},
                         \Frac{d\zeta}{d\xi}, \Frac{d^2\zeta}{d\xi^2} )
    \end{array}
\eeq{e:n5}
%
   where $U$ and $V$ gather non-linear terms that can be omitted if elevations 
are greater than $3^\circ$ and the accuracy of computation 1~ps is considered
sufficient. Here $n$ is the refractivity coefficient that is computed on the
basis of the atmospheric pressure, air temperature and specific humidity as

\beq
   n = 1 + k_{1d} \Frac{P_d}{T} Z_d^{-1} \: + \:
           \lp k_{1w} \Frac{P_w}{T} + k_{2w} \Frac{P_w}{T^2} \rp \, Z_w^{-1} ,
\eeq{e:r1}
%

After numerical solving differential equations \eref{e:n5}, functions $\eta(\xi), 
\zeta(\xi), \eta'(\xi), \zeta'(\xi)$ become known. Then the slanted path delay 
is found by integration of the refractivity index along the curvilinear 
trajectory $\eta(\xi),\zeta(\xi)$:

\beq
     \tau_{na} = \strut 
                 \Frac{1}{c} \int_0^\infty \lp n \lp \xi,\eta,\zeta \rp 
                  \sqrt{1 + \lp \Frac{d\eta}{d\xi}  \rp^2 +
                            \lp \Frac{d\zeta}{d\xi} \rp^2}  - 1 \rp d\xi .
\eeq{e:r2}


  Computation of the path delay through the neutral atmosphere is a 
computationally intensive process are requires voluminous dataset 
(1Tb per year). Therefore, for practical reasons, the slanted path delay
is computed outside of VTD. The slanted path delay is computed for each 
site, each time epoch of the numerical weather model by first solving 
differential equations \ref{e:r1} and then integrating the refractivity
index along the trajectory at a 2D grid: azimuth and elevation. In order
to exploit efficient interpolation, the grid should be equidistant.
Since the dependence of path delay on elevation is strongly non-linear,
equidistant grid over azimuth and elevation is the not the optimal: in order
to get interpolation errors below 1~ps, too many nodes over elevation are
needed. The efficiency of interpolation is significantly improved if to
perform a non-linear transformation of the grid, i.e. to present the 
slanted path delay on a regular grid of arguments others than azimuth 
and elevation. The optimal choice is a function that represents the 
average path delay. Numerical experiments show that a satisfactory 
approximation is achieved when when we transform
arguments $(A,E) \longrightarrow (A,M_{\mbox{iso}}(P(E)))$, where $A$ stands
for azimuth, $E$ for elevation, and $M_{\mbox{st}}(P(E))$ is the mapping 
function computed for the refractivity index distribution determined by the 
ISO International Standard Atmosphere \cite{r:iso_isa}. This function
itself is represented not as a function of elevation but as a function of
another, more simple, approximation of the mapping function, namely
%
\beq
    P(E) = \Frac{1}{E \, (1 + \frac{2}{\pi}\, E_0) - E_0} \quad, 
\eeq{e:r3}
%
where $E_0$ is $-0.052$ radians is selected in such a way that to avoid 
singularity even for the ray grazing the horizon and remain normalized to
1 in zenith direction. Series of expansion $M_{\mbox{st}}(P(E))$ are 
converging much faster than series $M_{\mbox{st}}(P(E))$. Function 
$M_{\mbox{st}}(P(E))$ was first computed by numerical integration 
at the range of elevations $[-0.052, \frac{\pi}{2}]$ at a regular elevation
grid of size 1024. Then the series of $M_{\mbox{st}}(P(E))$ were expanded
over Chebyshev polynomial of the 12th degree:
%
\beq
  \begin{array}{lcl}
     M_{\mbox{st}}(P(E)) & = & \dss\sum_{n=0}^{n=12}
        c_n \; T_n\lp \Frac{P(E) - P_b}{P_e - P_b} \rp
     \\
     M^{-1}_{\mbox{st}}(M) & = &
         \Frac{1}{1 + \frac{2}{\pi}\, E_0} \,
         \arcsin \Frac{1}{\lp \dss\sum_{n=0}^{n=12}
                d_n \; T_n\lp \Frac{M - M_b}{M_e - M_b} \rp \rp } , 
   \end{array}
\eeq{e:r4}
%
  where $P_b = -0.052$, $P_e = \Frac{\pi}{2}$ and $M_b = 1.0$, 
$M_e = 46.815214\;$. The inverse $M^{-1}_{\mbox{st}}(M)$ returns
elevation angle as a function of mapping function.

  Numerical values of Chebyshev polynomial coefficients of 
$M_{\mbox{st}}(P(E))$ and its inverse $M^{-1}_{\mbox{st}}(M)$ are presented 
in table~\ref{t:misa}.


\begin{table}[h]
   \caption{The second column presents the coefficients of expansion 
   of mapping function for the International Standard Atmosphere 
   as a function of P(E) over Chebyshev polynomials at the range of
   elevations $[-0.052, \frac{\pi}{2}]$. The third column presents
   the coefficients of expansion of its inverse $M^{-1}_{\mbox{st}}(M)$
   over Chebyshev polynomials at the range $[1.0, 46.815214]$. 
   The argument of the inverse is the mapping function, and the value
   is elevation. The maximal error of interpolation is $8 \cdot 10^{-6}$. }
   \label{t:misa}
\beq
   \begin{array}{r@{\qquad} r @{\qquad}r}
       \hline
       \mbox{Deg} & M_{\mbox{st}}(P(E)) & M^{-1}_{\mbox{st}}(M) 
            \vphantom{\biggr[\biggr]} \\
       \hline \vnex \\
        0 &   \flo{2.3496276}{+01} &   \flo{1.433399}{+01} \\
        1 &   \flo{2.3756641}{+01} &   \flo{1.255039}{+01} \\
        2 &   \flo{2.8960727}{-01} &  -\flo{1.159927}{-01} \\
        3 &  -\flo{8.8489567}{-01} &   \flo{4.812282}{-01} \\
        4 &   \flo{1.4212949}{-01} &  -\flo{1.142513}{-01} \\
        5 &   \flo{3.4080806}{-02} &   \flo{4.717299}{-02} \\
        6 &  -\flo{2.1776292}{-02} &  -\flo{1.584444}{-02} \\
        7 &   \flo{2.6966697}{-03} &   \flo{5.871918}{-03} \\
        8 &   \flo{1.3182015}{-03} &  -\flo{1.949435}{-03} \\
        9 &  -\flo{9.2308499}{-04} &   \flo{7.900266}{-04} \\
       10 &   \flo{6.8354478}{-05} &  -\flo{2.467537}{-04} \\
       11 &   \flo{1.3502649}{-05} &   \flo{9.892615}{-05} \\
       12 &  -\flo{1.7105554}{-05} &  -\flo{1.280538}{-04} \\
       \hline
   \end{array}
\eeqn
\end{table}

  Numerical experiments showed that the interpolation errors are below
1~ps at elevations $[3^\circ, 90^\circ]$ when the slanted path delay 
at a given station is expanded over B-spline basis at a regular grid 
$A, M_{\mbox{st}}(P(E))$ with 12 steps over azimuth in the range 
of $[0, 2\pi]$ and 16 steps over $M_{\mbox{st}}(P(E))$ in the range of 
[1.0,  14.65859].

  In order to interpolate slanted path delay at $A,E,T$ ($T$ is time),
the triplet of arguments should be transformed to $A,M,T$ using
expression \ref{e:r4}. The 3D field of slanted path delay at a given 
station is expanded over B-spline series of the $m$-th degree:
%
\beq
  \begin{array}{l}
      \tau_{\mss{na}}(A,M,T) = 
       \mst \dss \sum_{l=1\!-m}^{l=d_T} \sum_{k=1\!-m}^{k=d_M} 
                 \sum_{j=1\!-m}^{j=d_A} \!\! f_{ijk} \, 
                   B_i^m(A) \, B_j^m(M) \, B_k^m(T) \quad .
  \end{array}
\eeq{e:r5}

   Interpolation of slanted path delay and applying it to the total
path delay in VTD is controlled by keywords 
{\sf SLANTED\_PATH\_DELAY\_MODEL}, 
{\sf SLANTED\_PATH\_DELAY\_BIAS\_FILE}, {\sf EXTERNAL\_DELAY\_DIR},
{\sf EXTERNAL\_DELAY\_DIR\_2ND}, {\sf EXTERNAL\_DELAY\_DIR\_3RD},
and {\sf EXTERNAL\_DELAY\_DIR\_4TH}.

  The first keyword is either {\sf NONE} or {SPD\_3D}. The latter
values indicates that the 3D slanted path delay should be read 
from external files and interpolated. The files with path delay
are located up to 4 directories, one file per station. VTD will
look first directory {\sf EXTERNAL\_DELAY\_DIR}, then
{\sf EXTERNAL\_DELAY\_DIR\_2ND}, then {\sf EXTERNAL\_DELAY\_DIR\_3RD},
and at last{\sf EXTERNAL\_DELAY\_DIR\_4TH}. If more than one file for 
a given station is provided, the file from the first directory will
be picked up.

  Slanted path delay can be corrected for station-dependent empirical 
additive offset $a$ and multiplicative bias $b$:
%
\beq
   \tau_{\mss{na}} = \tau_{\mss{na}-\mss{orig}} + a_{\mss{offset}}
               + b\, \tau_{\mss{nhy}-\mss{orig}} \quad,
\eeq{e:r6}
%
  where $\tau_{\mss{nhy}-\mss{orig}}$ is original non-hydrostatic
component of the path delay through the neutral atmosphere. Keyword
{\sf SLANTED\_PATH\_DELAY\_BIAS\_FILE} specifies the name of that
file. If it {\sf NONE}, no empirical correction is made.


\subsubsection{Computation of path delay through the neutral atmosphere.
               Regression approach.}


  In the absence of rigorous computations, the path delay can be evaluated as
%
\beq
  \tau_{atm} = \tau_{hz} \, m_h(e + \eta \cos(A) + \epsilon \sin(A)) + 
               \tau_{nz} \, m_n(e + \eta \cos(A) + \epsilon \sin(A))
\eeq{e:trop}
%
  where $\tau_{hz}$ --- is the hydrostatic constituent of the path delay
in the neutral atmosphere in zenith direction, $\tau_{nz}$ --- is the
the non-hydrostatic constituent of the path delay the neutral atmosphere 
in zenith direction, $m_h(e)$ and $m_n(e)$ are the so-called mapping 
functions that describe the dependence of the atmosphere path delay on
the angle between the source and the symmetry axis of the atmosphere.
$e$ are $A$ the elevation angle and the azimuth in vacuum, $\eta$ and
$\epsilon$ are inclination angles of the symmetry axis in the north and
east directions.

  Hydrostatic constituent of the path delay in zenith direction can be 
computed using the atmospheric pressure at the level of the reference
point with accuracy 2--5~ps. Other parameters of \ref{e:trop} cannot be
predicted without detail knowledge of the atmospheric parameters profile.
Therefore, a regression expression is used. The most precise regression
expression was computed using the output of the numerical weather model
GEOS--5. Regression coefficients were computed for all 205 VLBI stations.

  Computation of the hydrostatic path delay in the source direction is
controlled by two options: {\sf HYDROSTATIC\_ZENITH\_DELAY} and
{\sf HYDROSTATIC\_MAPPING\_FUNCTION}. The
keyword {\sf HYDROSTATIC\_ZENITH\_DELAY} support values {\tt NONE}, 
{\tt SAASTAMOINEN} for the model of \cite{r:saa72a,r:saa72b}, or MMF for the 
Mean Mapping Function Model (L. Petrov (2009), manuscript in preparation).
Saastamoinien model is the best model. No hydrostatic path delay is computed
when keyword {\sf HYDROSTATIC\_ZENITH\_DELAY} has value {\tt NONE}.
The zenith path delay is to be multiplied by the mapping function which 
is defined in the keyword {\st HYDROSTATIC\_MAPPING\_FUNCTION}.  
The alternatives are {\tt NMFH} for the Niell hydrostatic mapping function 
and MMF for the Mean Mapping Function. 

  Computation of the non-hydrostatic, wet path delay in the source direction
is controlled by two options: {\sf WET\_ZENITH\_DELAY} and
{\sf WE\_MAPPING\_FUNCTION}. The keyword {\sf WET\_ZENITH\_DELAY} support 
values MMF for the Mean Mapping Function Model, and {\tt NONE}. In the latter 
case no wet path delay is computed. The MMF is the best model. The zenith 
path delay is to be multiplied by the mapping function which is defined in 
the keyword {\st WET\_MAPPING\_FUNCTION}. Alternatives are {\tt NMFW} for 
the Niell non-hydrostatic mapping function and MMF for the Mean Mapping 
Function. 

  Computation of path delay depends on surface meteorological parameters.
In the absence of measured meteorological parameters, default values are
computed on the basis of the model specified by the keyword {\sf METEO\_DEF}.
If the value of pressure are out of range [500\,00, 110\,000] Pa, all
meteorological parameters, air pressure, air temperature and relative
humidity are considered missing. Three values are supported {\tt IMA},
{\tt CALC} and {\tt NONE}. The value {\tt IMA} means if the surface \
meteorological parameters are missing they are computed according to the 
model of the International Meteorological Association. The value {\tt CALC} 
means that the surface meteorological parameters are computed in the mode 
of compatibility with Calc: $ P = 101325.0D0 \cdot (1 - \flo{6.5}{-3} 
\cdot h_{ort}/293.15)^{5.26}$. The origin of the model that Calc uses 
is obscure. The value {\tt NONE} means that any value of pressure, 
temperature and relatives are considered present, regardless whether 
they are out of range or not.

\subsubsection{Computation of delay ionosphere path delay.}


  Keyword {\sf IONOSPHERIC\_MODEL} specifies whether ionosphere path 
delay is to be computed. Supported values: {\sf NONE} or {\sf GNSS\_TEC\_MAP}.
Value {\sf GNSS\_TEC\_MAP} means that ionosphere path delay will be computed
using gridded time series of TEC from analysis of global navigation satellite
systems, such as GPS or GLONASS. The files with the TEC values in VIONO 
format are specified by keywords
{\sf IONOSPHERE\_DATA\_FILE}, {\sf IONOSPHERE\_DATA\_FILE\_2ND}
{\sf IONOSPHERE\_DATA\_FILE\_3RD}, {\sf IONOSPHERE\_DATA\_FILE\_4TH}.
Why more than one file? The data file is supposed to present TEC at a regular
grid with the same step, without gaps. The TEC model may have different grid
size and may have gaps. In that case the dataset is split into several 
files, each of them presents the TEC model at a regular grid without gaps.
If more than one TEC model outputs are specified in keywords
{\sf IONOSPHERE\_DATA\_FILE}, {\sf IONOSPHERE\_DATA\_FILE\_2ND}
{\sf IONOSPHERE\_DATA\_FILE\_3RD}, {\sf IONOSPHERE\_DATA\_FILE\_4TH},
then it is assumed that each file covers different time span.


\subsection{Partial derivatives and coupling effects.}

  Partial derivatives of time delay and delay rate with respect to parameters
of the product of the tilt of the symmetry axis of the atmosphere in the 
local topocentric coordinate system and the zenith path delay are computed 
in accordance with the expression specified by the keyword 
{\sf ATMOSPHERE\_TILT\_PARTIALS}. The following values are supported:

\begin{itemize}
   \item {\tt MACMILLAN\_1995} --- Expression according to  \cite{r:mm95}:
%
\beq
     \Der{\tau}{\tau_{grad,North}} = R_{zd} \Frac{\cos{A}}{\tan{E}} \\
     \Der{\tau}{\tau_{grad,East}}  = R_{zd} \Frac{\sin{A}}{\tan{E}}
\eeq{e:im3}
%
   \item {\tt TILT\_NMFH}      --- Expression according to \cite{r:cc97}:
%
\beq
     \Der{\tau}{\tau_{grad,North}} = R_{zd} \Frac{\cos{A}}{\tan{E}\sin{E} + C} \\
     \Der{\tau}{\tau_{grad,East}}  = R_{zd} \Frac{\sin{A}}{\tan{E}\sin{E} + C}
\eeq{e:im4}
%
   \item {\tt TILT\_NMFW}      --- Expression according to \cite{r:cc97}:
%
\beq
     \der{\tau}{\tau_{grad,North}} = R_{zw} \Frac{\cos{A}}{\tan{E}\sin{E} + C} \\
     \der{\tau}{\tau_{grad,East}}  = R_{zw }\Frac{\sin{A}}{\tan{E}\sin{E} + C}
\eeq{e:im5}
   \item {\tt NONE}      --- No partial derivative with respect to parameters
of the tilt to perform.
\end{itemize}
where $C = 0.0031$.

   Computation of the coupling term between the tropospheric path delay and
axis offset is controlled by the keyword {\sf TROP\_AXOF\_COUPLING}, which
supports two values: {\tt YES} and {\tt NONE}. Computation of the coupling
term between the tropospheric path delay and the geometric delay is controlled
by the keyword {\sf TROP\_GEOMETRIC\_COUPLING}, which supports two
values: {\tt YES} and {\tt NONE}.

   Computation of delay rate using an analytical expression can be
turned on or turned off in accordance with the value of the keyword
{\sf DELAY\_RATE}: {\tt YES} or {\tt NONE}.

  Keywords {\sf GEOM\_EXPR\_FAR\_ZONE},
{\sf GEOM\_EXPR\_FAR\_ZONE}, and {\sf SOURCE\_STRUCTURE} are reserved
for future. Currently, they should have the following values:
%
\begin{itemize}
   \item {\sf GEOM\_EXPR\_FAR\_ZONE} {\tt PK2001}
   \item {\sf GEOM\_EXPR\_FAR\_ZONE} {\tt LIGHT\_TIME}
   \item {\sf SOURCE\_STRUCTURE}   {\tt NONE}
\end{itemize}

   The keyword {\sf GRS\_METRIC} specifies the scaling parameter in the
metric tensor for the geocentric coordinate system. The following values
are supported:
%
\begin{itemize}
   \item {\tt ITRF2000} --- The ITRF2000 metric (the same as IERS1992):
                            $L_g =  \Frac{f M\earth}{\bar{R}\earth \, c^2} =
                                    \flo{6.969290134}{-10}$;
%
   \item {\tt IAU2000}  --- The IAU2000  metric: $L_g = 0$;
%
   \item {\tt IERS1996} --- The IERS1996 metric:
                            $L_g =  -\Frac{f M\earth}{\bar{R}\earth \, c^2} =
                                    -\flo{6.969290134}{-10}$;
\end{itemize}

\subsection{Computation of path delay when one of the station is on orbit}

   Theory of computation of path delay when one of the station is on the
orbit is given in section \ref{s:orbit}. The orbit ephemerise should be read
with routine {\sf VTD\_READ\_NZO} and then loaded with 
routine {\sf VTD\_LOAD\_NZO}. The satellite ephemeris should conform 
to Orbit Data Messsages CSSDS 502.0-B-2%
\footnote{http://public.ccsds.org/publications/archive/502x0b2c1.pdf} 
standard issued by the Consultative Committee for Space Data Systems.
Computation of path delay is done with the same routine as for computation 
of path delay between ground stations: {\sf VTD\_DELAY}. Effects of tides, 
antenna axis offsets, loadings, delay in the ionosphere and troposphere 
are not computed for the orbiting station. NB: {\sf VTD} cannot compute 
path delay if the orbit is not available.

   In the case of a satellite that does not have a sample counter, like
Radioastron, clock synchronization is done implicitly at the moment of 
receiving the first sample of a scan. The light time between the orbiting 
station and the control downlink station (which is usually {\it different} 
than the ground observing station) should be computed and added to the path
delay returned by {\sf VTD\_DELAY}. Routine {\sf VTD\_LT\_ORB} makes this
computation. The effective clock function of the orbiting station 
that should be used in data reduction is a sum of three terms: the clock 
function of the orbiting station that accounts for Hydrogen maser frequency 
variations, the clock station of the downlink station, and the light time
between the orbiting and downlink station. The first (unknown) term affects all 
the samples. The second and third term affect only the first sample of a scan.
The second term is computed from time comparison between the Hydrogen
maser at the downlink station and the GPS clock and it is approximated by a
polynomial of the first degree. 

  The term that describes time dilation (expression \ref{e:e16d})
is computed by routine {\sf VTD\_REL\_ORB}. It has {\bf two time arguments}:
time synchronization epoch and observation epoch. NB: both time arguments
should be given at a retarded moment of time, $t_s$. 
The difference $t_d - t_s$ is given by routine {\sf VTD\_LT\_ORB}.
This difference should be {\it subtracted } from time of observation.
The result of {\sf VTD\_REL\_ORB}, $\Delta t$, is {\it added} if the 
orbiting station is a reference station \#1 or {\it subtracted} 
if the orbiting station is a remote station \#2.

   Reduction for clock for the ground station is made by computing
a~priori clock function from Hydrogen maser comparison and the GPS clock
and applying this function to {\it every} sample. For reduction for time
dilation with a continuous on-board sample counter, the first time epoch 
for routine {\sf VTD\_REL\_ORB} is time of clock synchronization which 
usually takes place prior the observing session. For reduction for time 
dilation without a continuous on-board sample counter, the first epoch
of {\sf VTD\_REL\_ORB} is a time coordinate at the orbiting station 
at the moment {\it first sample of a given scan}, which is time of 
a downlink station minus light time $t_d - t_s$ given by routine 
{\sf VTD\_LT\_ORB}.

  The fundamental distinction of a case when an orbiting station
without a continuous on-board sample counter from other cases that
time delay and delay rate is a {\bf function of two time arguments:}
nominal scan start determined by a downlink clock and time of observation.
If the orbiting station has a continuous on-board sample counter, then
the first argument is time of clock synchronization at the beginning
of the experiment. Its precise value with accuracy significantly better 
then the window of fringe search is irrelevant, since it will be solved
for during parameter estimation anyway.

  Thus, time delay for an orbiting station is a sum of three terms:
delay returned by {\sf VTD\_DELAY}, light time returned by {\sf VTD\_LT\_ORB},
and clock dilation returned by {\sf VTD\_REL\_ORB} with an appropriate sign.

\begin{thebibliography}{21}

  \bibitem[Bowring, 1985]{r:bowring85}
      Bowring,~B.R., 1985, The Accuracy of Geodetic Latitude and Height
      Equations", Survey Review, vol.~28, pp. 202--206.

  \bibitem[Capitaine et~al.(2003a)]{r:Cap03a}
      Capitaine,~N., Chapront,~J., Lambert,~S., \& Wallace,~P.~T.,
        ``Expressions for the Celestial Intermediate Pole and Celestial
          Ephemeris Origin consistent with the IAU 2000A precession-nutation
          model'', {\it Astronomy and Astrophysics}, vol.~400, p.~1145, 2003

  \bibitem[Chen and Herring (1997)]{r:cc97}
      Chen~G., Herring~T.~A. ``Effects of atmospheric azumithal assymetry on
        the analysis of space geodetic data'', {\it Journal of Geophys. Res.},
        1997, vol.~102(B9), pp. 20,489--20,502.

  \bibitem[Davis (1985)]{r:davis85}
     Davis, J.L., et al., "Geodesy by radio interferometry:
          Effects of atmospheric modeling errors on estimates of
          baseline length", Radio Science, 20, 1593-1607, 1985.

  \bibitem[Dehant et~al., 1998]{r:ddw98}
     Dehant~V., Defraigne~P. and Wahr~J.~M, ``Tides for a convective Earth'',
       {\it Journal of Geophys. Res.}, 1998, vol.~104(B1), pp. 1035-1058.

  \bibitem[Dickman, 1993]{r:dickman1993}
      Dickman, S.R, ``Dynamic ocean-tide effects on Earth's rotation'',
         {\it Geophys. J. Int.}, vol. 112, pp. 448-470, 1993.

  \bibitem[Goff \& Gratch, 1946]{r:goff}
      Goff J.A., S.~Gratch.
         Low-pressure properties of water from {$-160^\circ$ to $212^\circ F$},
          {\em American society of heating and ventilating engeneers.
           Transactions}, {\bf 52}, pp. 95--129, 1946.

  \bibitem[Hartmann and Wenzel(1995)]{r:hw95}
      Hartmann, T. and H.-G.~Wenzel, The HW95 tidal potential catalogue,
        {\it Geophys. Res. Let.}, {\it 22}, 3353--3356, 1995.

  \bibitem[Hrgian, 1975]{r:hrgian}
      Hrgian, A.H, ed., International meteorological tables, Ser.~1--2,
          Transactions of the Soviet meteorological association, vol. 94,
          Obninsk, 1975.

  \bibitem[Kopeikin and  Shaeffer, 1999]{r:ks1999}
      Kopeikin,~S.~M. and Shaeffer~G., Physical Review, D, vol. 50, 124002, 1999.

  \bibitem[Lieske et~al., 1976]{r:lieske1977}
      Lieske, J.H. and Lederle, T. and Fricke, W. \& Morando, B.,
       ``Expressions for the precession quantaties based upon the {IAU} (1976)
         system of astronomical constants'', {\it Astronomy \& Astrophysics},
         vol.~58, N1, pp.~1--17, 1977.

  \bibitem[{\it Mathews et al.}(1995)]{r:mat95}
      Mathews, P.M., B.A.~Buffett, and I.I.~Shapiro, Love numbers for a
        rotating spheroidal Earth: New definitions and numerical values,
        {\it Geophys. Res. Let.}, {\it 22}(5), 579--582, 1995.

  \bibitem[Mathews et~al., 1997]{r:mdg97}
     Mathews~P.M., Dehant~V. and Gipson~J.~M. ``Tidal Station Displacements'',
       {\it Journal of Geophys. Res.}, 1997, vol.~102(B9), pp.~20469--20478

  \bibitem[{\it Mathews }(2001)]{r:mat01}
      Mathews, P.M., Love numbers and gravimetric factor for diurnal tides,
        {\it Journal of the Geodetic Society of Japan}, {\it 47}(1),
         231--236, 2001.

  \bibitem[McMillan and Ma (1997)]{r:mm95}
      McMaillan~D.S. and Ma~C. ``Atmospheric gradients and the VLBI
        terrestrical and celestial reference frames'', {\it Geophys. Res.
        Let.}, vol.~24, pp.~453--456, 1997.

  \bibitem[Matveenko et~al.(1965)]{r:mat65}
      Matveenko,~L.~I., Karadashev,~A.~C., Sholomitskij,~G.~B.
      1965, Izvestia VUZov. Radiofizika, 8, 651 (in Russian).

  \bibitem[Niell (1996)]{r:nmf}
      Niell, A.E., (1996). Global mapping functions for the atmosphere delay
        at radio wavelengths, {\it J. Geophys. Res.}, 101, No. B2,
        p.~3227--3246.

  \bibitem[Saastamoinen(1972a)]{r:saa72a}
     Saastamoinen~J (1972a) Contributions to the theory of 
       atmospheric refraction, Bull Geod, 105:279--298

  \bibitem[Saastamoinen(1972b)]{r:saa72b}
     Saastamoinen~J (1972b) Introduction to practical computation of 
       astronomical refraction. Part II Bull Geod, 106:383--397

  \bibitem[{\it Simon et al.}(1994)]{r:simon}
     Simon, J.L, P.~Bretagnon, J.~Chapront, M.~Chapront-Touze, G.,~Francou,
       and J.~Laskar, Numerical expressions for precession formulae and
       mean elements for the Moon and the planets, {\it Astronomy and
       Astrophysics}, {\it 282}, 663--684, 1994.

  \bibitem[Sovers et al.(1998)]{r:masterfit}
      Sovers,~O.J, J.L. Fanselow, and C.S. Jacobs, 1998, Review of Modern Physics,
      70, 1393

  \bibitem[ISO (1975)]{r:iso_isa}
      International Organization for Standardization, Standard Atmosphere,  
        ISO 2533:1975, 1975
        {\tt http://en.wikipedia.org/wiki/International\_Standard\_Atmosphere}

\end{thebibliography}

\end{document}
